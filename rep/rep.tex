
%\documentclass[12pt, utf8]{G7-32-kurs}
\documentclass[12pt]{article}
\usepackage[14pt]{extsizes}

%\usepackage[utf8]{inputenc}
%\usepackage[T2A]{fontenc}
%\usepackage[english, russian]{babel}

\usepackage[pdftex]{graphicx}
\usepackage{float}
\usepackage{pdflscape}
\usepackage{bytefield}
\usepackage{enumerate}

%\usepackage{indentfirst}
%\usepackage[multiple]{footmisc}

\usepackage{listingsutf8}
\lstloadlanguages{C}
\lstset
{
        language=C,
		breaklines,
		columns=fullflexible,
		flexiblecolumns,
		numbers=none,
        basicstyle=\tt\fontsize{12pt}{12pt}\selectfont,
        commentstyle=\bf,
        showtabs=false, 
        showspaces=false,
        showstringspaces=false,
        tabsize=2,
        inputencoding=utf8/cp1251,
		frame=single,
		showlines=true
}

%\usepackage{ltxtable}
%\usepackage{tabularx}
%\usepackage{longtable}
%\usepackage{array}
%\usepackage{multirow}
%\setlength{\extrarowheight}{2pt}
%\newcolumntype{Y}{>{\centering\arraybackslash}X}
%\newcolumntype{Z}{>{\small\centering\arraybackslash}p{1.5cm}}
%\newcolumntype{M}[1]{>{\small\centering\arraybackslash}p{#1}}
%\newcolumntype{N}{>{\small\centering\arraybackslash}X}
%\setlength{\LTcapwidth}{\textwidth}

\usepackage{amssymb,amsfonts,amsmath}
%\usepackage{amssymb,amsfonts,amsmath,mathtext,cite,enumerate}

%\EqInChapter
%\TableInChapter
%\PicInChapter

\usepackage{geometry}
\geometry{left=2cm}
\geometry{right=2cm}
\geometry{top=2cm}
\geometry{bottom=2cm}

\graphicspath{{image/}}

%\renewcommand{\theenumi}{\arabic{enumi}.}
%\renewcommand{\theenumii}{\arabic{enumii}}
%\renewcommand{\theenumiii}{.\arabic{enumiii}}
%\renewcommand{\theenumiv}{.\arabic{enumiv}}

%\renewcommand{\labelenumi}{\arabic{enumi}.}
%\renewcommand{\labelenumii}{\arabic{enumi}.\arabic{enumii}}
%\renewcommand{\labelenumiii}{\arabic{enumi}.\arabic{enumii}.\arabic{enumiii}}
%\renewcommand{\labelenumiv}{\arabic{enumi}.\arabic{enumii}.\arabic{enumiii}.\arabic{enumiv}}

%\renewcommand{\rmdefault}{mcr} % Courier New

\makeatletter 
\renewcommand\appendix
{
	\par
	\setcounter{chapter}{0}%
	\setcounter{section}{0}%
	\gdef\thechapter{\@Asbuk\c@chapter}
}
\makeatother

\newcommand\myappendix[1]
{
	\refstepcounter{chapter}
	
	\chapter*{\centering{Приложение~\thechapter}}
	\begin{center}
		\large \bf #1
	\end{center}
	
	\addcontentsline{toc}{chapter}{Приложение~\thechapter:~#1}
}

\newcommand{\mysource}[2]
{
        \refstepcounter{figure}
		\label{listing:#2}
		{
			\centering{Листинг~\thefigure~---~Содержимое~файла~<<#2>>}
			\nopagebreak
			\lstinputlisting[]{#1#2}
			\bigskip
		}
}

% \mimage{test}{test}{Тестовый рисунок}{width=\linewidth}

%\newcommand\mimage[4]
%{

%	\begin{center}
		
%		\begin{figure}[H]
	
%			\includegraphics[#4]{#2.png}
%			\caption{#3}
%			\label{image:#1}
	
%		\end{figure}

%	\end{center}
%}

\newcommand\mimage[4]
{
	\vbox
	{
		\begin{center}
			
			\refstepcounter{figure}
			\label{image:#1}
			\includegraphics[#4]{#2.png}

			{\noindent Рисунок~\thefigure~---~#3}

		\end{center}
	}
}



\newcommand{\linux}{ОС GNU/Linux}
\newcommand{\virtpo}{VMWare}

\newcommand{\first}{10.0.1.128}
\newcommand{\second}{10.0.2.128}
\newcommand{\midethf}{eth0}
\newcommand{\mideths}{eth1}
\newcommand{\midf}{10.0.1.1}
\newcommand{\mids}{10.0.2.1}

\newcommand{\mytitle}{<<Курс лабораторных работ\\по дисциплине <<Вычислительные сети>>\\для ОС GNU/Linux>>}
\newcommand{\tfirst}{Настройка подключения вычислительной системы к локальной вычислительной сети}
\newcommand{\tsecond}{Протоколы транспортного уровня стека протоколов TCP/IP}
\newcommand{\tthird}{Протоколы сетевого уровня стека протоколов TCP/IP}
\newcommand{\tfourth}{Перехват сетевого трафика. Межсетевой экран netfilter}

\begin{document}

% ############################################################################ 
% Вступление

\frontmatter

	
\NirOrgLongName{\textsc{Рязанский Государственный Радиотехнический Университет\\Кафедра ЭВМ}}

\NirManager{старший преподаватель кафедры ЭВМ}{С.И.Бабаев}

\NirTown{г. Рязань,}

\NirUdk{УДК \No~519.687.4, \No~519.688}	% Организация совместной работы нескольких вычислительных машин,
										% программы и алгоритмы для решения отдельных задач на вычислительных машинах
\NirGosNo{}

%\NirStage{Этап \No 5}{заключительный}{<<Записка по курсовому проекту>>}
% 1 - ТЗ
% 2 - Написание программы
% 3 - Отладка
% 4 - Тестирование
% 5 - Записка

\NirTitle{\textbf{\mytitle}}



	\Executors		% Исполнители

		\begin{longtable}{p{0.35\linewidth}p{0.2\linewidth}p{0.35\linewidth}}
		студент группы 740М, & & \\
		М.В. Акинин & \rule{1\linewidth}{0.1pt}& \\
		\end{longtable}

	\tableofcontents

% ############################################################################ 
% Основная часть

\mainmatter

	\chapter{Введение}

		
\section{Описание курса лабораторных работ}

	Данный курс лабораторных работ предназначен для студентов 3 - 5-го курсов, обучающихся по специальности <<Компьютерная безопасность>>
	(код специальности: 090102) и может быть также предложен студентам прочих специальностей, уровень подготовки которых удовлетворяет приведенным
	ниже требованиям.

	Данный курс лабораторных работ разработан мною как возможная альтернатива курсу лабораторных работ, рассчитанных на выполнение в ОС семейства
	Windows, поскольку использование ОС Windows для проведения лабораторных работ по дисциплине <<Вычислительные сети>>
	для подготовки специалистов по специальности 090102 имеет ряд недостатков:

	\begin{itemize}

		\item ОС GNU/Linux удобна в использовании, поскольку обладает дружественными и легкими в освоении консольным и графическими интерфейсами,
		в отличии от ОС Windows, консольный интерфейс управления которой существенно ограничивает возможности оператора по управлению ОС,
		а использование графического интерфейса управления вычислительной сетью представляется не всегда разумным;

		\item ОС Windows суть коммерческая ОС, отказ от использования которой при подготовке специалистов может существенным образом сэкономить средства
		ВУЗа;

		\item Развитие привычки работать с ОС Windows у будущих специалистов в области компьютерной безопасности может негативно сказаться на их
		карьерном будущем, так как решения, основанные на ОС Windows, не обладают достаточной степень защищенности и, следовательно, не могут использоваться
		для решения задач обеспечения безопасности информации в вычислительных системах и вычислительных сетях определенного назначения.

		Решения, позволяющие повысить уровень защищенности ОС Windows (SecretNet и прочие), также являются коммерческими, а, следовательно, ведут к
		дополнительным затратам организаций, их использующих, и необходимости их подробного освоения, что, очевидно, не может быть обеспечено в ходе подготовки
		специалистов по специальности 090102.

	\end{itemize}

	В свою очередь, ОС GNU/Linux обладает рядом достоинств:

	\begin{itemize}

		\item Удобные консольный и графические интерфейсы, позволяющие оператору ОС максимально удобным образом настроить интерфейс ОС под свои потребности;
		\item Огромный потенциал автоматизации рутинных действий с помощью разнообразных скриптовых языков, таких, как: Perl, Python, языки командных оболочек
		(Bash, Zsh);
		\item Широкий спектр сетевых возможностей ОС, включающий в себя, кроме всего прочего, поддержку большого количества разнообразного сетевого оборудования;
		\item Бесплатность;
		\item Распространение подавляющего большинства компонент системы в виде пакетов с файлами исходного кода компонент;
		\item Оперативное устранение уязвимостей в основных компонентах ОС таких, как ядро Linux и библиотека GLIBC;
		\item Возможность, для некоторых дистрибутивов, покупки некоторого срока техподдержки;
		\item Большое количество пользователей ОС, настроенных, в основном, дружелюбно и могущих дать ответ на возникший в процессе использования ОС вопрос.

	\end{itemize}

	Кроме того, необходимо отметить также факт того, что ОС GNU/Linux относится к классу Unix-подобных ОС, а, следовательно, навыки, полученные при использовании
	ОС GNU/Linux, студенты могут перенести и в прочие Unix-подобные ОС, такие, как FreeBSD, OpenBSD, NetBSD, Solaris и другие ОС.

	При выполнении предлагаемых лабораторных работ студенты приобретут следующие навыки:

	\begin{itemize}

		\item Навыки работы в командной строке ОС GNU/Linux;
		\item Навыки программирования на языке C в ОС GNU/Linux с применением средств стандартной библиотеки языка C;
		\item Навыки настройки подключения сетевого узла, управляемого ОС GNU/Linux, к вычислительной сети относительно сложной топологии;
		\item Навыки мониторинга состояния вычислительных сетей;
		\item Навыки сетевого программирования с использованием протоколов ARP, ICMP, TCP и UDP стека протоколов TCP/IPv4;
		\item Навыки перехвата сетевого трафика с помощью специальных программных комплексов - снифферов, навыки анализа перехваченного сетевого трафика;
		\item Навыки разработки собственных программных средств перехвата сетевого трафика.

	\end{itemize}

	Предлагаемый курс лабораторных работ состоит из четырех лабораторных работ из расчета один месяц на подготовку к лабораторной работе, на выполнение
	лабораторной работы и на защиту отчета по лабораторной работе. Курс лабораторных работ состоит из следующих работ:

	\begin{enumerate}[1.]

		\item {\bf<<Настройка подключения вычислительной системы к локальной вычислительной сети>>}.

			Данная лабораторная работы знакомит студентов с основными принципами настройки подключения сетевых узлов, управляемых ОС
			GNU/Linux, к вычислительным сетям достаточно сложной топологии. В данной лабораторной работе студенты получат необходимые базовые навыки
			программирования на языке C для ОС GNU/Linux, требующиеся для выполнения последующих лабораторных работ;

		\item {\bf<<Протоколы транспортного уровня стека протоколов TCP/IP>>}.

			Данная лабораторная работа посвящена протоколам TCP и UDP и принципам IP-адресации сетевых узлов и подсетей вычислительных сетей, использующих
			для сетевого обмена протоколы стека протоколов TCP/IPv4. В ходе выполнения данной лабораторной работы студенты ознакомятся с функционалом ОС
			GNU/Linux, позволяющим производить мониторинг сетевого обмена по протоколам TCP и UDP, позволяющим осуществлять простейший обмен между
			сетевыми узлами с использованием указанных протоколов, а также позволяющим решать прочие задачи по управлению сетевым обменом с использованием
			протоколов TCP и UDP. В рамках данной лабораторной работе студенты должны будут разработать многокомпонентный программный комплекс, организующий
			удаленное управление с одного из сетевых узлов вычислительной сети, используемой в ходе выполнения лабораторной работы, другими сетевыми узлами
			означенной вычислительной сети;

		\item {\bf<<Протоколы сетевого уровня стека протоколов TCP/IP>>}.

			Данная лабораторная работа посвящена протоколам IP, ICMP и ARP стека протоколов TCP/IP. В ходе выполнения данной лабораторной работы студенты
			ознакомятся с возможностями ОС GNU/Linux по определению факта доступности удаленного сетевого узла и определению некоторых характеристик возможного
			сетевого обмена с данным сетевым узлом. В ходе выполнения данной лабораторной работы студенты приобретут навыки работы с ARP-кэшами
			сетевых узлов, операторами которых они являются. В рамках данной лабораторной работы студенты должны будут разработать набор программ, использующих
			протокол ICMP для проверки доступности удаленного сетевого узла и протокол ARP для определения соответствия некоторому IP-адресу
			соответствующего MAC-адреса;
		
		\item {\bf<<Перехват сетевого трафика. Межсетевой экран netfilter>>}.

			Данная лабораторная работа посвящена снифферу tcpdump, межсетевому экрану netfilter и консольной утилитой управления межсетевым экраном netfilter - iptables, а также
			средствам ОС GNU/Linux, позволяющим программисту разработать собственный программный комплекс перехвата сетевого трафика.
			В ходе выполнения данной лабораторной работы студенты приобретут навыки использования сниффера tcpdump для интеллектуального перехвата сетевого
			трафика и получения информации о характере трафика и статистических закономерностях, прослеживающихся в содержании перехваченного трафика.
			В ходе выполнения данной лабораторной работы студенты получат навыки использования межсетевого экрана netfilter для оценки характера и объемов
			сетевого трафика, проходящего через сетевой узел. В ходе выполнения данной лабораторной работы студенты получат навыки разработки на
			языке программирования C собственных программных решений перехвата сетевого трафика, что является ценным опытом для будущего специалиста в области
			компьютерной безопасности.

	\end{enumerate}

	Студенты, приступающие к выполнению предлагаемого курса лабораторных работ, должны соответствовать следующим минимальным требованиям:

	\begin{itemize}

		\item Знание языка программирования C - на уровне одного семестра.

			Требования: знание основных компонент языка, умение работать с указателями, умение выделять и освобождать память с использованием
			функционала стандартной	библиотеки языка C, умение организовывать файловый ввод / вывод с использованием функционала стандартной
			библиотеки языка C;

		\item Навыки работы с консольным интерфейсом ОС GNU/Linux, Windows или DOS.

			Требование: знание основных команд (создание каталога, удаление каталога, переход в каталог и тому подобных).

	\end{itemize}

\section{Примеры выполнения заданий практических частей лабораторных работ}

	\subsection{Общая информация}

	В данном разделе приведены описания программ, разработанных мною по заданиям практических частей лабораторных работ.
	Каждая демонстрационная программа реализует, кроме основного задания
	практической части лабораторной работы, дополнительно одно из заданий, предназначенных для защиты отчета по лабораторной работе.
	
	Данный раздел демонстрирует практическую выполнимость заданий к лабораторным работам.
	
	Программа (программы) для каждой из лабораторных работ были
	разработаны (но не закомментированы) в течении сорока минут (лабораторная работа № 2: 52 минуты).
	Таким образом, имеем следующее временное расписание выполнения лабораторной работы студентом:

	\begin{enumerate}

		\item 1 - 2 дня по 4 - 5 часов - подготовка к лабораторной работе.

			Данный этап включает в себя изучение теоретической части методических указаний к лабораторной работе;

		\item Лабораторная работа:

			\begin{enumerate}

				\item 40 минут - выполнение практического части, не связанной с программированием;
				\item 1 час 20 минут - выполнение практической части, связанной с программированием.

					Данный этап подразумевает разработку и отладку программного обеспечения и демонстрацию преподавателю корректного выполнения разработанного программного обеспечения;

				\item 20 минут + 20 минут - защита отчета по предыдущей лабораторной работе.

			\end{enumerate}

	\end{enumerate}

	\subsection{Лабораторная работа № 1}

	В листинге \ref{listing:lab1.c} приведен исходный код программы, реализующей задание практической части лабораторной работы № 1 - получение содержимого
	таблицы маршрутизации сетевого узла с помощью файла /proc/net/route и вывод полученной таблицы в стандартный поток вывода.

	На рисунке \ref{image:demo-lab1} приведен выполнения демонстрационной программы.

	\mimage{demo-lab1}{demo/1}{Процесс выполнения демонстрационной программы}{width=0.95\textwidth}

	\subsection{Лабораторная работа № 2}

	В листинге \ref{listing:lab2.c} приведен исходный код программы, реализующей задание практической части лабораторной работы № 2 - удаленное управление
	серверными компонентами разработанного программного комплекса, запущенными на удаленных сетевых узлах, с помощью управляющей компоненты.

	На рисунке \ref{image:demo-lab2-1} приведен процесс выполнения управляющей компоненты.
	На рисунке \ref{image:demo-lab2-2} приведен процесс выполнения серверной компоненты, осуществляющей обмен с	управляющей компонентой по протоколу UDP.
	На рисунке \ref{image:demo-lab2-3} приведен процесс выполнения серверной компоненты, осуществляющей обмен с	управляющей компонентой по протоколу TCP.

	\mimage{demo-lab2-1}{demo/1000}{Процесс выполнения управляющей компоненты}{width=0.9\textwidth}
	\mimage{demo-lab2-2}{demo/1001}{Процесс выполнения серверной компоненты, осуществляющей обмен с управляющей компонентой по протоколу UDP}{width=\textwidth}
	\mimage{demo-lab2-3}{demo/1002}{Процесс выполнения серверной компоненты, осуществляющей обмен с управляющей компонентой по протоколу TCP}{width=\textwidth}

	\subsection{Лабораторная работа № 3}

	В листинге \ref{listing:lab3-arp.c} приведен исходный код программы, реализующей часть задания практической части лабораторной работы № 3 -
	отправление ARP-запроса и получение ARP-ответа. В листинге \ref{listing:lab3-icmp.c} приведен исходный код программы, реализующей часть
	задания практической части лабораторной работы № 3 - отправление пакетов типа ECHO протокола ICMP и получение ответных пакетов типа ECHO-REPLY
	протокола ICMP.

	На рисунке \ref{image:demo-lab3-arp} приведен выполнения демонстрационной программы, выполняющей отправление ARP-запроса и получение
	ARP-ответа. На рисунке \ref{image:demo-lab3-icmp} приведен выполнения демонстрационной программы, выполняющей отправление пакетов
	типа ECHO протокола ICMP и получение ответных пакетов типа ECHO-REPLY протокола ICMP. На рисунке \ref{image:demo-lab3-tcpdump} приведен вывод сниффера
	tcpdump, демонстрирующий корректность отправления и получения ответных пакетов соответствующих протоколов перечисленными демонстрационными программами.

	\mimage{demo-lab3-arp}{demo/1003}{Отправление ARP-запроса и получение ARP-ответа}{width=\textwidth}
	\mimage{demo-lab3-icmp}{demo/1004}{Отправление пакетов типа ECHO протокола ICMP и получение ответных пакетов типа ECHO-REPLY протокола ICMP}{width=\textwidth}
	\mimage{demo-lab3-tcpdump}{demo/1005}{Результат выполнения сниффера tcpdump}{width=\textwidth}

	\subsection{Лабораторная работа № 4}

	В листинге \ref{listing:lab4.c} приведен исходный код программы, реализующей задание практической части лабораторной работы № 4 - перехват пакетов
	протокола TCP.

	На рисунке \ref{image:demo-lab4} приведен выполнения демонстрационной программы. На рисунке \ref{image:demo-lab4-server}
	приведен процесс выполнения TCP-сервера на сетевом узле 10.0.1.128, на рисунке \ref{image:demo-lab4-client} приведен процесс выполнения клиента на сетевом
	узле 10.0.2.128, выполняющего подключение к TCP-серверу, запущенному на сетевом узле 10.0.1.128.

	\mimage{demo-lab4}{demo/704}{Процесс выполнения демонстрационной программы}{width=\textwidth}
	\mimage{demo-lab4-server}{demo/705}{Процесс выполнения TCP-сервера на сетевом узле 10.0.1.128}{width=\textwidth}
	\mimage{demo-lab4-client}{demo/706}{Процесс выполнения клиента на сетевом узле 10.0.2.128, выполняющего подключение к TCP-серверу, запущенному на сетевом узле 10.0.1.128}{width=\textwidth}



	\chapter{План выполнения лабораторных работ}

		\newcommand{\labplan}[2]
		{
			\section{Лабораторная работа № #1 <<#2>>}

				\subsection{Цель работы}

					
Получение навыков использования сниффера tcpdump для перехвата и анализа сетевого трафика.

Получение навыков разработки программного обеспечения, выполняющего перехват сетевого трафика.

Получение навыков использования межсетевого экрана netfilter и утилиты iptables,
предназначенной для конфигурирования netfilter.

Получение навыков использования средств командной строки \linux\ для выполнения всестороннего анализа
разнообразной текстовой информации.



				\subsection{Теоретическая часть}

					
\subsubsection{Протоколы сетевого уровня стека протоколов TCP/IP}

	Сетевой уровень модели OSI предназначен для образования единой транспортной системы, объединяющей несколько вычислительных сетей.

	К протоколам сетевого уровня стека протоколов TCP/IP относятся следующие протоколы:

	\begin{itemize}

		\item IP - протокол негарантированной доставки данных (в данной лабораторной работе используется версия IPv4 данного протокола);
		\item ICMP - сервисный протокол, предназначенный для выполнения различных служебных задач обеспечения функционирования сети, в частности,
		для передачи сообщений об ошибках;
		\item ARP - протокол, предназначенный для определения по известному адресу сетевого уровня адреса канального уровня (в случае сетей, построенных
		на базе технологии Ethernet и стеке протоколов TCP/IP, адресом канального уровня является MAC-адрес сетевого адаптера, адресом сетевого уровня -
		IP-адрес сетевого узла);
		\item RARP - протокол, предназначенный для определения по известному адресу канального уровня адреса сетевого уровня;
		\item IGMP - протокол управления групповой (multicast) передачей данных в сетях, основанных на протоколе IP;
		\item Прочие протоколы.

	\end{itemize}

	\linux\ предоставляет удобные программные средства работы с протоколами сетевого уровня стека протоколов TCP/IP.

	Использование протоколов сетевого уровня стека протоколов TCP/IP процессами системы для решения задач сетевого взаимодействия состоит в использовании
	сокетов соответствующих типов. Работа процессов ОС с протоколами сетевого уровня стека протоколов TCP/IP, в общем случае,
	подразумевает следущую последовательность действий:

	\begin{enumerate}

		\item Создание сокета с помощью системного вызова socket.

			В качестве идентификатора семейства протоколов данному системному вызову необходимо передать значение одной из следующих констант:

			\begin{itemize}

				\item AF\_INET - для работы с протоколами стека протоколов TCP/IPv4, пакеты которых при пересылке вкладываются в пакеты протокола IPv4;
				\item AF\_INET6 - для работы с протоколами стека протоколов TCP/IPv6, пакеты которых при пересылке вкладываются в пакеты протокола IPv6;
				\item AF\_PACKET - для работы с протоколами канального и сетевого уровней различных семейств протоколов.

					AF\_PACKET-сокеты используются для работы с протоколами сетевого уровня стека протоколов TCP/IP,
					пакеты которых не инкапсулируются в пакеты протоколов IPv4 и IPv6.

			\end{itemize}

			В качестве идентификатора режима передачи пакетов системному вызову socket необходимо передать значение одной из следующих констант:

			\begin{itemize}

				\item SOCK\_RAW - в случае, если в качестве идентификатора семейства протоколов системному вызову socket было передано значение констант
				AF\_INET или AF\_INET6.

					Использование сырых пакетных сокетов (сокетов, при создании которых системному вызову socket были переданы значения констант AF\_PACKET
					и SOCK\_RAW) предполагает работу с протоколами канального уровня;

				\item SOCK\_DGRAM - в случае, если в качестве идентификатора семейства протоколов системному вызову socket было передано значение константы
				AF\_PACKET.

			\end{itemize}

			Использовать пакетные (AF\_PACKET) и (или) сырые (SOCK\_RAW) сокеты имеют право только те процессы, которые обладают характеристикой CAP\_NET\_RAW.
			Характеристика CAP\_NET\_RAW по умолчанию устанавливается на процессы, имеющие эффективным владельцем суперпользователя.
			К процессам, имеющим эффективным владельцем суперпользователя, относятся процессы, запущенные суперпользователем;

		\item Отправление / получение данных с помощью системных вызовов sendto и recv соответственно.

			При отправлении и получении пакетов протоколов сетевого уровня необходимо помнить, что заголовки означенных пакетов
			считываются ОС из пользовательского буфера при отправлении пакетов и помещаются ОС в пользовательский буфер при получении пакетов. Это означает,
			что, в случае отправления пакета процессом ОС, заголовок пакета должен быть сформирован процессом и помещен в начало буфера,
			указатель на который передается в системный вызов sendto. При получении пакета операционная система записывает в пользовательский
			буфер пакет целиком - заголовок пакета в начало буфера и поле данных - следом за заголовком.
			Для корректого формирования и анализа заголовков пакетов необходимо использовать указатели на экземпляры следующих структур данных:

			\begin{itemize}

				\item Протокол IP - структура данных iphdr, описанная в заголовочном файле \linebreak <netinet/ip.h>;
				\item Протокол ICMP - структура данных icmphdr, описанная в заголовочном файле <netinet/ip\_icmp.h>;
				\item Протоколы ARP и RARP - структура данных arphdr, описанная в заголовочном файле <net/if\_arp.h>;
				\item Протокол IGMP - структура данных igmp, описанная в заголовочном файле \linebreak <netinet/igmp.h>.

			\end{itemize}

			Корректное формирование и анализ содержимого пакетов протоколов сетевого уровня состоит, таким образом, в выполнении следующих действий:

			\begin{enumerate}

				\item Описание переменной типа <<указатель на структуру данных, описывающую заголовок пакета>>;
				\item Присвоение указателю адреса первого байта буфера;
				\item Формирование / анализ содержимого заголовка пакета с помощью указателя на соответствующую структуру данных;
				\item Запись / считывание содержимого поля данных пакета, расположенного в буфере по смещению, равному размеру в байтах заголовка пакета.
				Размер в байтах заголовка пакета, в свою очередь, равен размеру в байтах экземпляра соответствующей структуры данных.

			\end{enumerate}

			Необходимо также помнить, что системному вызову sendto передается, кроме всего прочего, описатель адреса сетевого узла - получателя пакета.
			Данный описатель суть есть указатель на экземпляр одной из следующих структур данных, явно приведенный к указателю на экземпляр структуры данных
			sockaddr:

			\begin{itemize}

				\item sockaddr\_in - в случае использования протокола, пакеты которого инкапсулируются в пакеты протокола IPv4;
				\item sockaddr\_ll - в случае использования протокола, пакеты которого не инкапсулируются в пакеты протоколов IPv4 или IPv6.
				
				Структура данных sockaddr\_ll описана в заголовочном файле <netpacket/ packet.h>; 

			\end{itemize}

		\item Уничтожение сокета с помощью системного вызова close.
	
	\end{enumerate}

\subsubsection{Протокол IP}
	
	Протокол IPv4 (в дальнейшем просто протокол IP) - протокол негарантированной доставки данных, использующий формат IPv4 адресации сетевых узлов.

	Протокол IP обеспечивает доставку пользовательских данных на сетевом уровне взаимодействия узлов сети. Пакеты протоколов транспортного уровня
	стека протоколов TCP/IP и некоторые протоколы сетевого уровня стека протоколов TCP/IP вкладываются (инкапсулируются) в пакеты протокола IP.

	Формат пакета протокола IP приведен на рисунке \ref{image:lab3-ip-struct}.

	\vbox
	{
		\begin{center}
			
			\refstepcounter{figure}
			\label{image:lab3-ip-struct}

			\begin{bytefield}[bitwidth=12pt]{32}
				\bitheader{0,3,4,7,8,15,16,31}\\
				\begin{rightwordgroup}{\rotatebox{90}{Заголовок пакета}}
				\bitbox{4}{Version}\bitbox{4}{IHL}\bitbox{8}{ToS}\bitbox{16}{Total}\\
				\bitbox{16}{ID}\bitbox{1}{\rotatebox{90}{\small RF}}\bitbox{1}{\rotatebox{90}{\small DF}}\bitbox{1}{\rotatebox{90}{\small MF}}
				\bitbox{13}{Offset}\\
				\bitbox{8}{TTL}\bitbox{8}{Protocol}\bitbox{16}{Checksum}\\
				\bitbox{32}{Source}\\
				\bitbox{32}{Destination}\\
				\bitbox{32}{Options}\\
				\bitbox{32}{Padding}
				\end{rightwordgroup}\\
				\wordbox{1}{Data}
			\end{bytefield}

			{\noindent Рисунок~\thefigure~---~Формат пакета протокола IP}

		\end{center}
	}

	Заголовок пакета протокола IP состоит из следующих полей:

	\begin{itemize}

		\item <<Version>> - номер версии протокола IP. Для протокола IPv4 значение данного поля устанавливается в 4;
		\item <<IHL>> - размер в двойных словах (32 бита) заголовка пакета (IP Header Length).
		Обычно, значение данного поля равно 5;
		\item <<ToS>> - тип сервиса (Type of Service; ToS; байт дифференцированного обслуживания; DS-байт). Значение данного поля отражает требования отправителя
		к качеству обслуживания пакета;
		\item <<Total>> - размер пакета в байтах;
		\item <<ID>> - идентификатор пакета, используемый для идентификации исходного блока данных, фрагментированного для передачи на несколько IP-пакетов.
		Фрагменты одного и того же блока данных имеют один и тот же идентификатор;
		\item <<RF>> - зарезервированный бит;
		\item <<DF>> - флаг запрета фрагментации пакета (Do not Fragment);
		\item <<MF>> - флаг, установленный в 1 в том случае, если данный пакет не является последним в цепочке пакетов, содержащих фрагменты исходного
		блока данных (More Fragments);
		\item <<Offset>> - смещение в байтах поля <<Data>> (поля данных) пакета от начала заголовка пакета. Значение данного поля должно быть кратно 8;
		\item <<TTL>> - время жизни пакета (Time To Live; TTL) в переходах между маршрутизаторами (в хопах).
		
		Каждый маршрутизатор, пересылающий пакет, вычитает из TTL пакета единицу. В случае, если значение TTL пакета достигло нуля,
		то пакет отбрасывается очередным маршрутизатором;

		\item <<Protocol>> - идентификатор протокола, пакет которого вложен в пакет протокола IP.

		Идентификатор протокола ICMP равен 1, протокола IGMP - 2, протокола TCP - 6, протокола UDP - 17, протокола SCTP - 132;

		\item <<Checksum>> - контрольная сумма заголовка пакета;
		\item <<Source>> - IP-адрес сетевого узла - источника пакета;
		\item <<Destination>> - IP-адрес сетевого узла - получателя пакета;
		\item <<Options>> - произвольное число параметров, используемых при отладке вычислительной сети;
		\item <<Padding>> - заполнение заголовка пакета нулями до размера в байтах, кратного 32-м.

	\end{itemize}

	В поле <<Data>> (поле данных) IP-пакета помещаются данные, пересылаемые с помощью пакета протокола IP. В частности, в поле <<Data>> могут быть помещены
	пакеты протоколов ICMP, TCP, UDP или SCTP.

	При написании программ, работающих непосредственно с пакетами протокола IP, программист может
	воспользоваться структурой данных iphdr, определенной в заголовочном файле <netinet/ip.h> и удобным образом
	описывающей заголовок IP-пакета. Имеют место быть следующие соответствия полей структуры данных iphdr
	полям заголовка пакета протокола IP;

	\begin{itemize}

		\item Поле version - поле <<Version>>;
		\item Поле ihl - поле <<IHL>>;
		\item Поле tos - поле <<ToS>>;
		\item Поле tot\_len - поле <<Total>>;
		\item Поле id - поле <<ID>>;
		\item Поле frag\_off - поля <<RF>>, <<DF>>, <<MF>> и <<Offset>>.

			Для выделения из значения поля frag\_off значений составляющих его полей заголовка пакета протокола IP
			необходимо использовать следующие, определенные в заголовочном файле <netinet/ip.h>, маски:

				\begin{itemize}

					\item IP\_RF - поле <<RF>>;
					\item IP\_DF - поле <<DF>>;
					\item IP\_MF - поле <<MF>>;
					\item IP\_OFFMASK - поле <<Offset>>;

				\end{itemize}

		\item Поле ttl - поле <<TTL>;
		\item Поле protocol - поле <<Protocol>>;
		\item Поле check - поле <<Checksum>>;
		\item Поле saddr - поле <<Source>>;
		\item Поле daddr - поле <<Destination>>.

	\end{itemize}

\subsubsection{Протокол ICMP. Утилита ping}

	Протокол ICMP суть есть сервисный протокол, предназначенный для выполнения различных служебных задач обеспечения функционирования сети, в частности,
	для передачи сообщений об ошибках. При передаче пакеты протокола ICMP вкладываются в пакеты протокола IP.

	Формат пакета протокола ICMP приведен на рисунке \ref{image:lab3-icmp-struct}.

	\vbox
	{
		\begin{center}
			
			\refstepcounter{figure}
			\label{image:lab3-icmp-struct}

			\begin{bytefield}{32}
				\bitheader{0,7,8,15,16,31}\\
				\bitbox{8}{Type}\bitbox{8}{Code}\bitbox{16}{Checksum}\\
				\wordbox{1}{Data}
			\end{bytefield}

			{\noindent Рисунок~\thefigure~---~Формат пакета протокола ICMP}

		\end{center}
	}

	Пакет протокола ICMP состоит из следующих полей:

		\begin{itemize}
		
			\item Поле <<Type>> содержит идентификатор типа пакета и может принимать следующие значения:

				\begin{itemize}

					\item 0 — эхо-ответ (ECHO-REPLY);
					\item 4 — сдерживание источника (отключение источника при переполнении очереди);
					\item 5 — перенаправление;
					\item 8 — эхо-запрос (ECHO);
					\item 11 — превышение временного интервала (для дейтаграммы время жизни истекло);
					\item 12 — неверный параметр (проблема с параметрами дейтаграммы: ошибка в IP-заголовке или отсутствует необходимая опция);
					\item 30 — трассировка маршрута;
					\item Прочие значения из диапазона $[0, 255]$;

				\end{itemize}

			\item Поле <<Code>> заполняется для пакетов определенных типов и содержит, фактически, уточнение типа пакета;
			\item Поле <<Checksum>> содержит контрольную сумму пакета;
			\item Поле <<Data>> содержит дополнительные характеристики ICMP-пакета и данные, пересылаемые в ICMP-пакете.
			Формат данного поля зависит от типа пакета.

		\end{itemize}

	В данной лабораторной работе исследуется возможность применения пакетов типа ECHO и ECHO-REPLY для проверки доступности
	удаленного узла вычислительной сети.

	Алгоритм проверки доступности удаленного узла вычислительной сети с помощью пакетов означенных типов состоит из следующих шагов:

	\begin{enumerate}

		\item Проверяющий сетевой узел отправляет проверяемому сетевому узлу ECHO-пакет;
		\item Проверяющий сетевой узел ожидает получения ответного ECHO-REPLY-пакета от проверяемого сетевого узла.

		Если таковой пакет получен, то проверяемый сетевой узел доступен проверяющему сетевому узлу. Если ответный ECHO-REPLY-пакет не получен проверяющим
		сетевым узлом за определенный промежуток времени, то проверяемый сетевой узел не доступен проверяющему сетевому узлу
		или доступен, но с неудовлетворительными временными характеристиками сетевого обмена.

	\end{enumerate}

	Для проверки доступности удаленного сетевого узла с помощью пакетов типа ECHO и ECHO-REPLY оператор может воспользоваться утилитой ping.

	На рисунке \ref{image:lab3-1} приведен процесс проверки доступности сетевого узла с IP-адресом, равным 127.0.0.1 (что соответствует обратной петле -
	то есть узлу, на котором запущена утилита ping), выполненной с помощью утилиты ping. Без дополнительных ключей утилита ping работает до тех пор,
	пока не будет прервана оператором (на рисунке \ref{image:lab3-1} выполнение ping прервано отправлением ей сигнала SIGINT путем нажатия
	сочетания клавиш <<Ctrl + C>>).

	TTL пакетов на рисунке \ref{image:lab3-1} равен 64, что означает, что маршрут до целевого узла сети может содержать не более 64-х узлов, считая целевой узел.
			
	\mimage{lab3-1}{3/1}{Проверка доступности сетевого узла 127.0.0.1 с помощью утилиты ping (остановлена оператором)}{}

	Для предписания самостоятельного завершения после отправления определенного оператором количества пакетов утилите ping необходимо передать
	ключ <<-c COUNT>>, где COUNT - количество пакетов, после отправления которых утилита ping должна самостоятельно завершить свою работу.

	\mimage{lab3-2}{3/2}{Проверка доступности сетевого узла 127.0.0.1 с помощью утилиты ping (самостоятельно завершила работу после отправления 10-ти пакетов)}{}

	Размер блоков данных в пакетах, отправляемых утилитой ping, можно указать с помощью ключа <<-s SIZE>>, где SIZE - размер в байтах блока данных ECHO-пакета.

	\mimage{lab3-3}{3/3}{Проверка доступности сетевого узла 127.0.0.1 с помощью утилиты ping (размер блоков данных в отправляемых пакетах равен 777-и байтам)}{}

	Для изменения времени ожидания утилитой ping ответных пакетов от целевого узла сети необходимо воспользоваться ключом <<-W TIMEOUT>>, где TIMEOUT -
	максимальное время ожидания ответного пакета в секундах.

	На рисунке \ref{image:lab3-6} приведен результат выполнения утилиты ping к сетевому узлу с IP-адресом равным 192.168.91.128.

	\mimage{lab3-6}{3/6}{Проверка доступности сетевого узла 192.168.91.128 с помощью утилиты ping}{}

	Процессы ОС, использующие пакеты типов ECHO и ECHO-REPLY протокола ICMP для проверки доступности удаленных сетевых узлов,
	должны следовать следующим замечаниям:

	\begin{enumerate}

		\item Системный вызов socket, используемый для создания сокета, должен быть вызван со следующими параметрами:

			\begin{itemize}

				\item AF\_INET - идентификатор семейства протоколов (стек протоколов TCP/ IPv4);
				\item AF\_RAW - идентификатор режима передачи пакетов (сырой сокет);
				\item IPPROTO\_ICMP - идентификатор протокола сетевого взаимодействия (протокол ICMP);

			\end{itemize}

		\item Для отправления и получения пакетов необходимо использовать системные вызовы sendto и recv.
		
		При получении процессом пакета протокола ICMP сетевая подсистема ОС записывает в результирующий буфер содержимое заголовка пакета протокола IP перед
		заголовком пакета протокола ICMP - таким образом, заголовок пакета протокола ICMP находится в результирующем буфере по смещению, равному размеру в
		байтах заголовка пакета протокола IP. Размер в байтах заголовка пакета протокола IP можно считать, с некоторым допущением, равным размеру в байтах
		экземпляра структуры данных iphdr;

		\item Идентификаторы типов пакетов ECHO и ECHO-REPLY равны значениям констант ICMP\_ECHO и ICMP\_ECHOREPLY, определенным в заголовочном файле
		\linebreak <netinet/ip\_icmp.h>;
		\item Значения поля <<Code>> пакетов типов ECHO и ECHO-REPLY устанавливаются в ноль;
		\item Формат поля данных пакетов имеет вид, приведенный на рисунке \ref{image:lab3-icmp-data-struct}.

			\vbox
			{
				\begin{center}
					
					\refstepcounter{figure}
					\label{image:lab3-icmp-data-struct}

					\begin{bytefield}{32}
						\bitheader{0,15,16,31}\\
						\bitbox{16}{Identifier}\bitbox{16}{Sequence number}
					\end{bytefield}

					{\noindent Рисунок~\thefigure~---~Формат поля данных пакетов типа ECHO и ECHO-REPLY протокола ICMP}

				\end{center}
			}

			Здесь поле <<Identifier>> содержит 16-ти битовое беззнаковое целое число - идентификатор последовательности пакетов.
			Данное число идентифицирует процесс проверки доступности удаленного сетевого узла,
			состоящий из нескольких циклов обмена пакетами типов ECHO и ECHO-REPLY.

			Поле <<Sequence number>> содержит порядковый номер пакета в последовательности пакетов, составляющих процесс
			проверки доступности удаленного сетевого узла, считая с нуля.

			Поля <<Identifier>> и <<Sequence number>> записываются в сетевом порядке байт;

		\item Контрольная сумма пакета протокола ICMP вычисляется как дополненная до единицы сумма слов, составляющих содержимое пакета.
		При вычислении контрольной суммы пакета значение поля <<Checksum>> должно быть установлено в ноль.

		В листинге \ref{listing:lab3-checksum} приведен код функции checksum(), вычисляющей контрольную сумму содержимого буфера,
		указатель на который передается функции	в параметре buf, а размер в байтах которого - в параметре buf\_size. Функция checksum() возвращает
		16-ти битовое беззнаковое целое число - контрольную сумму содержимого буфера, указатель на который передан функции в параметре buf.

		\vbox
		{
			\medskip
			\begin{center}
					
				\refstepcounter{figure}
				\label{listing:lab3-checksum}
				\begin{lstlisting}

#include <stdint.h>

uint16_t checksum(uint16_t *buf, uint16_t buf_size)
{
	unsigned u;
	uint32_t sum = 0;
	buf_size /= 2;

	for(u = 0; u < buf_size; u++)
		sum += buf[u];

	return ~ ((sum & 0xFFFF) + (sum >> 16));
}

				\end{lstlisting}

				{\noindent Листинг~\thefigure~---~Функция checksum(), вычисляющая контрольную сумму содержимого целевого буфера}

			\end{center}
		}

		\item Структура данных icmphdr, описывающая пакет протокола ICMP, состоит из следующих полей:

			\begin{itemize}
					
				\item type - идентификатор типа пакета;
				\item code - код пакета (для пакетов типа ECHO и ECHO\_REPLY данное поле не используется и установлено в ноль);
				\item checksum - контрольная сумма пакета;
				\item un.echo.id - идентификатор последовательности пакетов;
				\item un.echo.sequence - номер пакета в последовательности.

			\end{itemize}

	\end{enumerate}

\subsubsection{Протокол ARP. Получение и редактирование содержимого кэша ARP}

	Протокол ARP предназначен для определения по известному адресу сетевого уровня соответствующего адреса канального уровня (в случае сетей, построенных
	на базе технологии Ethernet и стеке протоколов TCP/IP, адресом канального уровня является MAC-адрес сетевого адаптера, адресом сетевого уровня
	- IP-адрес сетевого узла).

	Формат пакета протокола ARP приведен на рисунке \ref{image:lab3-arp-struct}.

	\vbox
	{
		\begin{center}
			
			\refstepcounter{figure}
			\label{image:lab3-arp-struct}

			\begin{bytefield}{32}
				\bitheader{0,7,8,15,16,31}\\
				\bitbox{16}{Hardware type}\bitbox{16}{Protocol type}\\
				\bitbox{8}{Hardware len.}
				\bitbox{8}{Protocol len.}
				\bitbox{16}{Operation}\\
				\begin{rightwordgroup}{Размер полей зависит \\ от значений полей \\ <<Hardware length>> \\ и <<Protocol length>>}
				\bitbox{32}{SHA}\\
				\bitbox{32}{SPA}\\
				\bitbox{32}{THA}\\
				\bitbox{32}{TPA}
				\end{rightwordgroup}
			\end{bytefield}

			{\noindent Рисунок~\thefigure~---~Формат пакета протокола ARP}

		\end{center}
	}

	Пакет протокола ARP состоит из следующих полей:

	\begin{itemize}
			
		\item Поле <<Hardware type> - идентификатор протокола канального уровня;
		\item Поле <<Protocol type>> - идентификатор протокола сетевого уровня;
		\item Поле <<Hardware length>> - размер в байтах сетевого адреса, используемого в протоколе канального уровня;
		\item Поле <<Protocol length>> - размер в байтах сетевого адреса, используемого в протоколе сетевого уровня;
		\item Поле <<Operation>> - тип операции;
		\item Поле <<SHA>> - адрес узла отправителя, используемый в протоколе канального уровня;
		\item Поле <<SPA>> - адрес узла отправителя, используемый в протоколе сетевого уровня;
		\item Поле <<THA>> - адрес узла получателя, используемый в протоколе канального уровня;
		\item Поле <<TPA>> - адрес узла получателя, используемый в протоколе сетевого уровня.

	\end{itemize}

	Размеры полей <<SHA>>, <<SPA>>, <<THA>> и <<TPA>> зависят от протоколов, соответствие для которых устанавливается. 

	Сетевой обмен с использованием протокола ARP может проходить по следующему сценарию:

	\begin{enumerate}

		\item Сетевой узел - отправитель формирует ARP-запрос на установление соответствия MAC-адреса некоторому IP-адресу.

		ARP-запрос представляет из себя ARP-пакет со следующими значениями полей:

			\begin{itemize}
			
				\item Поле <<Hardware type> - 0x0001, что соответствует технологии Ethernet;
				\item Поле <<Protocol type>> - 0x0800, что соответствует протоколу IPv4;
				\item Поле <<Hardware length>> - 6;
				\item Поле <<Protocol length>> - 4;
				\item Поле <<Operation>> - 1;
				\item Поле <<SHA>> - MAC-адрес сетевого интерфейса, с которого сетевой узел - отправитель отправит пакет (размер поля - 6 байт);
				\item Поле <<SPA>> - IPv4-адрес сетевого узла - отправителя (размер поля - 4 байта);
				\item Поле <<THA>> - 0 (размер поля - 6 байт);
				\item Поле <<TPA>> - IPv4-адрес сетевого узла - получателя;

			\end{itemize}

		\item Сетевой узел - отправитель посылает ARP-запрос на широковещательный MAC-адрес (FF:FF:FF:FF:FF:FF);

		\item Удаленный сетевой узел, чей IP-адрес равен IP-адресу, помещенному в ARP-запрос, формирует ARP-ответ и отправляет ARP-ответ
		сетевому узлу - отправителю.

		Заполнение полей пакета ARP-ответа на ARP-запрос совпадает с заполнением полей пакета ARP-запроса со следующими исключениями:

			\begin{itemize}

				\item Поле <<Operation>> - 2;
				\item Поле <<THA>> - MAC-адрес сетевого интерфейса, с которого удаленный сетевой узел отправит ответный пакет (размер поля - 6 байт);

			\end{itemize}

		\item Сетевой узел - отправитель помещает запись о соответствии MAC-адреса IP-адресу в свой кэш ARP и в дальнейшем использует
		данную запись в тех случаях, когда возникает необходимость поставить в соответствие данному IP-адресу некоторый MAC-адрес
		(такая необходимость возникает, например, при выполнении вложения IP-пакета в Ethernet-кадр).

	\end{enumerate}

	Означенный кэш ARP сетевого узла суть есть важный элемент сетевого обмена, так как с помощью данного кэша сетевой узел ставит в соответствие
	IP-адресам (адресам сетевого уровня) MAC-адреса (адреса канального уровня). В случае гипотетической некорректной реализации кэша ARP вычислительная система
	не смогла бы в принципе осуществлять сетевой обмен по протоколам стека протоколов TCP/IP, так как связь между протоколами канального и сетевого уровней
	была бы нарушена.

	Записи о соответствиях IP-адресов MAC-адресам в кэше ARP бывают двух видов:

	\begin{itemize}

		\item Статические записи - добавляются оператором сетевого узла вручную (с помощью утилиты arp) и не выталкиваются из кэша;
		\item Динамические записи - могут быть добавлены как вручную оператором, так и динамически ОС и могут быть вытолкнуты из кэша
		по прошествии определенного промежутка времени (устаревание записи) или при переполнении кэша.

	\end{itemize}

	Получить содержимое кэша ARP можно следующими способами:

	\begin{enumerate}

		\item С помощью утилиты arp, для чего необходимо вызвать ее без аргументов, что проиллюстрировано рисунком \ref{image:lab3-7};
		\item Прочитав файл /proc/net/arp, что проиллюстрировано рисунком \ref{image:lab3-8}.

	\end{enumerate}

	\mimage{lab3-7}{3/7}{Вывод содержимого кэша ARP с помощью утилиты arp}{width=\textwidth}
	\mimage{lab3-8}{3/8}{Вывод содержимого кэша ARP посредством чтения файла /proc/net/arp}{width=\textwidth}

	Для удаления записи из кэша ARP необходимо вызвать утилиту arp с ключом <<-d IP>>, где IP - IP-адрес узла, запись о котором необходимо
	удалить из кэша ARP. Удаление записи из кэша ARP требует прав суперпользователя.
	Для добавления динамической записи в кэш ARP необходимо обратится к целевому узлу (например, с помощью утилиты ping).

	На рисунке \ref{image:lab3-9} приведено состояние кэша ARP до удаления записи об узле 192.168. 91.128.
	На рисунке \ref{image:lab3-10} приведены процесс удаления записи об узле 192.168.91.128 из кэша ARP и состояние кэша ARP после удаления записи
	об узле 192.168.91.128. На рисунке \ref{image:lab3-11} приведен процесс обращения к узлу 192.168.91.128 с помощью утилиты ping.
	И, наконец, на рисунке \ref{image:lab3-12} приведено состояние кэша ARP после добавления в него информации об узле 192.168.91.128 по результатам обращения
	к нему с помощью утилиты ping. Как видно из приведенных рисунков, состояние кэша ARP до удаления/добавления записи об узле 192.168.91.128
	идентично состоянию кэша ARP после удаления/добавления записи об узле 192.168.91.128.

	\mimage{lab3-9}{3/9}{Состояние кэша ARP до удаления записи об узле 192.168.91.128}{width=\textwidth}
	\mimage{lab3-10}{3/10}{Удаление записи об узле 192.168.91.128 из кэша ARP и состояние кэша после удаления записи из него}{width=\textwidth}
	\mimage{lab3-11}{3/11}{Запрос к узлу 192.168.91.128 с помощью утилиты ping}{}
	\mimage{lab3-12}{3/12}{Состояние кэша ARP после добавления в него записи об узле 192.168.91.128}{width=\textwidth}

	При разработке программного обеспечения на языке программирования C для \linux, использующего протокол ARP для определения соответствий IPv4-адресам
	MAC-адресов, необходимо учитывать следующие замечания:

	\begin{enumerate}

		\item Системный вызов socket, используемый для создания сокета, вызывается со следующими параметрами:

			\begin{itemize}

				\item AF\_PACKET - идентификатор семейства протоколов (протоколы канального уровня и протоколы сетевого уровня стека протоколов TCP/IP,
				пакеты которых не вкладываются в пакеты протокола IP);
				\item AF\_DGRAM - идентификатор режима передачи пакетов (негарантированный режим передачи);
				\item ETH\_P\_ARP - идентификатор протокола сетевого взаимодействия (протокол ARP; константа определена в заголовочном файле
				<linux/if\_ether.h>).

				Данный параметр необходимо передавать в системный вызов socket в сетевом порядке байт, тогда как значение константы ETH\_P\_ARP записано в порядке
				байт хоста;

			\end{itemize}

		\item Широковещательный MAC-адрес передается системному вызову sendto с помощью структуры данных sockaddr\_ll, содержащей следующие поля:

			\begin{itemize}
				
				\item sll\_family - идентификатор семейства протоколов (AF\_PACKET);
				\item sll\_protocol - идентификатор протокола сетевого взаимодействия\\(ETH\_P\_ARP), записанный в сетевом порядке байт;
				\item sll\_ifindex - идентификатор сетевого интерфейса, через который необходимо отправить ARP-запрос;
				\item sll\_hatype - (в случае протокола ARP) идентификатор типа операции (1 для ARP-запроса);
				\item sll\_pkttype - идентификатор типа сетевого адреса (PACKET\_BROADCAST для широковещательного адреса);
				\item sll\_halen - размер в байтах адреса (6);
				\item sll\_addr - адрес (размер поля - 8 байт, в случае MAC-адреса используются первые 6 байт; все байты данного поля должны быть установлены
				в 0xFF);

			\end{itemize}

		\item Для получения идентификатора сетевого интерфейса по его имении необходимо воспользоваться системным вызовом ioctl.

		Возможное применение означенного системного вызова для получения идентификатора сетевого интерфейса по его имени
		проиллюстрировано функцией get\_iface\_ id(), исходный код которой приведен в листинге \ref{listing:lab3-get-iface-id}.
		Функция get\_iface\_id() принимает следующие параметры:
		
			\begin{itemize}
			
				\item sock - номер дескриптора сокета, открытого для обмена по протоколу ARP;
				\item device\_name - имя целевого сетевого интерфейса.
				
			\end{itemize}
			
		Функция get\_iface\_id() возвращает идентификатор целевого сетевого интерфейса или -1 в случае ошибки.

		\vbox
		{
			\medskip
			\begin{center}
					
				\refstepcounter{figure}
				\label{listing:lab3-get-iface-id}
				\begin{lstlisting}

#include <string.h>
#include <sys/ioctl.h>
#include <net/if.h>

int get_iface_id(int sock, const char* device_name)
{
	struct ifreq req;
	strcpy(req.ifr_name, device_name);

	if(ioctl(sock, SIOCGIFINDEX, & req))
		return -1;
	
	return req.ifr_ifindex;
}

				\end{lstlisting}

				{\noindent Листинг~\thefigure~---~Функция get\_iface\_id(), определяющая идентификатор целевого сетевого интерфейса}

			\end{center}
		}

		\item Структура данных arphdr, описывающая пакет протокола ARP, состоит из следующих полей:

			\begin{itemize}
					
				\item ar\_hrd - идентификатор протокола канального уровня - 0x0001, что соответствует технологии Ethernet;
				\item ar\_pro - идентификатор протокола сетевого уровня - 0x0800, что соответствует протоколу IPv4;
				\item ar\_hln - размер в байтах сетевого адреса, используемого в протоколе канального уровня - 6;
				\item ar\_pln - размер в байтах сетевого адреса, используемого в протоколе сетевого уровня - 4;
				\item ar\_op - тип операции - 1 (ARP-запрос) или 2 (ARP-ответ).

			\end{itemize}

			Поля ar\_hdr, ar\_pro и ar\_op должны быть записаны в сетевом порядке байт;

		\item Поля пакета протокола ARP, содержащие адреса сетевого узла - отправителя и сетевого узла - получателя, указываются в буфере, содержащем
		означенный пакет, непосредственно после заголовка пакета.

	\end{enumerate}



				\subsection{Практическая часть}

					
\subsubsection{Физическая организация локальной вычислительной сети, используемой в данной лабораторной работе}
	\label{lab1:lab1-net}

	Для выполнения данной лабораторной работы используется вычислительная сеть, состоящая из трех сетевых узлов.

	Сетевой узел, в дальнейшем называемый <<центральным>> сетевым узлом, и два сетевых узла, в дальнейшем называемые <<периферийными>> сетевыми узлами,
	должны быть организованы на основе виртуальных вычислительных систем, функционирующих с помощью программного комплекса
	виртуализации вычислительных систем \virtpo.

	Вычислительные системы, на которых должны быть организованы центральный и периферийные сетевые узлы,
	обладают следующими характеристиками:

	\begin{itemize}

		\item Аппаратная часть - вычислительная система на базе микропроцессора архитектуры ia32;
		\item Программная часть - \linux\ с ядром Linux версии 2.6.20 и выше и библиотекой GLIBC версии 2.11 и выше.

	\end{itemize}

	Центральный сетевой узел должен быть объединен с каждым из периферийных сетевых узлов в вычислительную сеть
	с помощью отдельного виртуального сетевого Ethernet-адаптера, функционирование которого обеспечивает \virtpo.
	Таким образом, вычислительная сеть, объединяющая три указанных сетевых узла, должна состоять из двух подсетей,
	каждая из которых должна состоять из центрального сетевого узла и одного из периферийных сетевых узлов.
	Структурная схема вычислительной сети приведена на рисунке \ref{image:lab1-struct}.

	\mimage{lab1-struct}{1/struct}{Структурная схема используемой вычислительной сети}{width=\textwidth}

\subsubsection{Настройка вычислительной сети}
	\label{lab1:lab1-p1}

	В соответствии с пунктом \ref{lab1:lab1-net} необходимо создать и настроить вычислительную сеть, состоящую из трех сетевых узлов,
	организованных на основе вычислительных систем, виртуализируемых с помощью программного комплекса \virtpo\ и
	использующих \linux, соответствующую заявленным в пункте \ref{lab1:lab1-net} требованиям.
	
	Для получения виртуальных машин, объединяемых в вычислительную сеть, необходимо выполнить следующие действия:

	\begin{enumerate}

		\item Получить у преподавателя эталонную виртуальную машину с предустановленной на ней \linux, соответствующей
		приведенным в пункте \ref{lab1:lab1-net} требованиям;

		\item Сделать две копии эталонной виртуальной машины.

	\end{enumerate}

	В дальнейшем, при запуске копий эталонной виртуальной машины на возможный запрос программного комплекса виртуализации о
	подтверждении получения виртуальной машины путем копирования эталонной виртуальной машины необходимо отвечать утвердительно.

	Полученные три идентичные виртуальные машины необходимо использовать следующим образом:

	\begin{enumerate}

		\item Эталонную виртуальную машину - в качестве вычислительной системы, на базе которой организуется
		центральный сетевой узел;

		\item Копии эталонной виртуальной машины - в качестве вычислительных систем, на базе которых
		организуются периферийные сетевые узлы.

	\end{enumerate}

	Перед выполнением непосредственной настройки вычислительной сети, необходимо выполнить следующие действия с виртуальными сетевыми адаптерами \virtpo:

	\begin{enumerate}

		\item Убедиться, что центральный сетевой узел физически подключен к обеим подсетям вычислительной сети,
		для чего необходимо открыть окно настроек виртуальной машины, на базе которой организуется
		центральный сетевой узел, и убедиться в том, что в данном окне присутствуют записи о
		подключенных к виртуальной машине двух виртуальных Ethernet-адаптерах;

		\item Для каждой из двух виртуальных машин, на базе которых организуются периферийные сетевые узлы,
		в окне настроек виртуальной машины требуется удалить одно из подключений к виртуальным сетевым адаптерам,
		для чего в окне настроек виртуальной машины необходимо выбрать
		соответствующее подключение и щелкнуть по кнопке <<Удалить>> (<<Delete>>) означенного окна.
		В конце концов, окно настроек виртуальной машины, на базе которой организуется периферийный сетевой узел,
		с точностью до имени сетевого адаптера примет вид, приведенный на рисунке \ref{image:lab1-vmsetup}.

		Необходимо помнить, что удаляемые подключения виртуальных машин к сетевым адаптерам
		не должны соответствовать одному и тому же виртуальному сетевому адаптеру,
		иначе создать сеть, состоящую из двух физически разделенных сетей, не получится.

		\mimage{lab1-vmsetup}{1/vmsetup}{Окно настроек виртуальной машины, на базе которой должен быть организован
		периферийный сетевой узел}{width=\textwidth}

	\end{enumerate}

	Непосредственная настройка вычислительной сети заключается в выполнении следующих действий:

	\begin{enumerate}

		\item На центральном сетевом узле:

			\begin{enumerate}

				\item Настройка сетевых интерфейсов в соответствии со схемой, приведенной на рисунке
				\ref{image:lab1-struct}.

				При выполнении настройки сетевых интерфейсов указывать широковещательные адреса сетей не нужно.
				В качестве масок сетей необходимо указывать 255.255.255.0;

				\item Настройка проброса пакетов между сетями.

				Для включения проброса пакетов между сетями необходимо выполнить команду <<echo 1 > /proc/sys/net/ipv4/ip\_forward>>.
				Данная команда запишет единицу в файл /proc/sys/net/ipv4/ip\_forward, содержимое которого (0 или 1) выступает в роли флага при
				принятии решения сетевым узлом о возможности проброса пакетов между сетями;

			\end{enumerate}

		\item На каждом из периферийных сетевых узлов:

			\begin{enumerate}

				\item Настройка сетевого интерфейса в соответствии со схемой, приведенной на рисунке
				\ref{image:lab1-struct}.

				При выполнении настройки сетевых интерфейсов указывать широковещательные адреса сетей не нужно.
				В качестве масок сетей необходимо указывать 255.255.255.0;

				\item Добавление в таблицу маршрутизации записи о центральном сетевом узле, как о шлюзе по умолчанию.

			\end{enumerate}

	\end{enumerate}

	Для проверки корректности настройки вычислительной сети необходимо, используя утилиту ping, выполнить следующие действия:

	\begin{enumerate}

		\item Проверить доступность центральному сетевому узлу каждого из периферийных сетевых узлов,
		послав каждому из периферийных сетевых узлов 10 пакетов типа ECHO протокола ICMP;

		\item Проверить доступность одному из периферийных узлов другого периферийного узла,
		послав тому 10 пакетов типа ECHO протокола ICMP.

	\end{enumerate}

\subsubsection{Получение содержимого таблицы маршрутизации}
	\label{lab1:lab1-p2}

	Необходимо разработать программу на языке C, которая, будучи запущена на центральном сетевом узле, выведет в стандартный поток вывода
	список маршрутов, присутствующих в таблице маршрутизации центрального сетевого узла, в виде строк вида: <<IFACE DEST GATEWAY MASK>>, где:

	\begin{itemize}

		\item IFACE - имя сетевого интерфейса, через который отправляются пакеты по соответствующему маршруту;
		\item DEST - IP-адрес целевой сети, записанный в традиционной нотации (четыре числа в десятичной системе счисления, разделенные точками);
		\item GATEWAY - IP-адрес шлюза, записанный в традиционной нотации;
		\item MASK - маска сети, записанная в традиционной нотации.

	\end{itemize}

	Существуют несколько способов получения содержимого таблицы маршрутизации в \linux\ в программах, написанных на языке программирования C.
	Одним из таковых способов является чтение содержимого файла /proc/net/route, в котором текущая таблица маршрутизации представлена в виде ASCII-текста.
	Пример содержимого файла \linebreak /proc/net/route некоторой вычислительной системы приведен на рисунке
	\ref{image:lab1-flroute}.

	\mimage{lab1-flroute}{1/100}{Содержимое файла /proc/net/route}{width=\textwidth}

	Содержимое файла /proc/net/route представляет из себя таблицу, состоящую из нескольких строк,
	каждая из которых, кроме первой строки и пустых строк, описывает один из доступных маршрутов.
	Первая строка содержит заголовок таблицы. Колонки таблицы, хранимой в файле /proc/net/route,
	описывают определенные параметры маршрутов:

	\begin{itemize}
		
		\item Iface - имя сетевого интерфейса;
		\item Destination - IP-адрес целевой сети;
		\item Gateway - IP-адрес шлюза;
		\item Mask - маска сети.

	\end{itemize}

	Все IP-адреса, записанные в файле /proc/net/route, представлены в виде 32-х битовых шестнадцатеричных чисел в сетевом порядке байт.
	Необходимо помнить, что порядок байт хоста в случае вычислительных систем, построенных на базе процессов архитектур ia32 и ia64,
	обратен сетевому порядку байт.

	Для редактирования файла исходного кода программы необходимо воспользоваться консольным текстовым редактором nano:

	\begin{enumerate}

		\item Открыть файл исходного кода на редактирование.
		
		Для открытия файла исходного кода на редактирование необходимо запустить на выполнение текстовый редактор nano,
		передав тому в качестве аргумента командной строки имя файла исходного кода. В том случае, если на момент открытия файл исходного кода не существовал,
		он будет создан, если оператор в процессе редактирования файла сохранит изменения, внесенные в файл.

		Файл исходного кода, очевидно, должен иметь расширение <<.c>>;

		\item Записать в файл исходного кода исходный код программы.

		Для переключения между раскладками (английской и русской) можно воспользоваться сочетанием клавиш <<Shift + Ctrl>>;

		\item Выйти из текстового редактора с сохранением изменений, внесенных в файл исходного кода.

		Для выхода из текстового редактора необходимо нажать сочетание клавиш <<Ctrl + X>>.
		Для подтверждения сохранения изменений при выходе из текстового редактора оператор должен ввести с
		клавиатуры символ <<Y>> в ответ на соответствующий запрос.
		
		Символ <<\verb|^|>>, присутствующий в большинстве указаний сочетаний клавиш для ввода команд управления редактором, означает клавишу <<Ctrl>>.
		Например, строка <<\verb|^|X>> соответствует сочетанию клавиш <<Ctrl + X>>.

	\end{enumerate}

	Компиляцию программы необходимо осуществить с помощью компилятора GNU C Compiler, выполнив в каталоге с файлом исходного кода следующую команду:
	<<gcc FILE -o program.out>>. Здесь:

	\begin{itemize}

		\item FILE - имя файла исходного кода;
		\item <<-o program.out>> - указание компилятору генерировать на выходе исполняемый файл program.out,
		сохраняемый в текущем каталоге.

	\end{itemize}

	Для запуска на выполнение полученного исполняемого файла необходимо выполнить команду <<PATH/program.out>>,
	где PATH - относительный или абсолютный путь к каталогу, содержащему исполняемый файл.
	PATH равен <<.>>, если исполняемый файл находится в текущем каталоге.

	Для принудительного завершения программы ее главному процессу необходимо отправить сигнал SIGINT нажатием сочетания клавиш <<Ctrl + C>> в том терминале
	вычислительной системы, в котором программа запущена.

\subsubsection{Подготовка отчета по лабораторной работе}

	Отчет по лабораторной работе должен содержать подробные описания процессов выполнения заданий пунктов
	\ref{lab1:lab1-p1} и \ref{lab1:lab1-p2}, проиллюстрированные достаточным количеством снимков экрана.

	В отчете по лабораторной работе должен быть приведен исходный код разработанного программного обеспечения с
	соответствующими комментариями.



				\subsection{Защита отчета по лабораторной работе}

					
Одно из следующих заданий может быть предложено в качестве задания для защиты отчета по лабораторной работе:

\begin{enumerate}

	\item Изменение IP-адресов сетевых узлов, IP-адресов подсетей вычислительной сети, масок подсетей вычислительной
	сети;
	
	\item Приведение структуры сети к структуре, заданной преподавателем (например, объединение подсетей вычислительной
	сети в сеть с единым IP-адресом сети и единой маской сети с отключением проброса пакетов на центральном
	сетевом узле);
	
	\item Добавление сетевого узла в вычислительную сеть с приведением структуры сети к структуре,
	заданной преподавателем;

	\item Подключение одного из сетевых узлов вычислительной сети, созданной в процессе выполнения лабораторной работы,
	к вычислительной сети, организованной в лаборатории, путем подключения сетевого адаптера вычислительной системы,
	в которой эмулируется данный сетевой узел, к виртуальной машине, в которой эмулируется данный сетевой узел,
	и организация на основе данного сетевого узла шлюза из вычислительной сети лаборатории в вычислительную сеть,
	созданную в процессе выполнения лабораторной работы.

\end{enumerate}


		}

		\labplan{1}{\tfirst}
		\labplan{2}{\tsecond}
		\labplan{3}{\tthird}
		\labplan{4}{\tfourth}

% ############################################################################ 
% Заключительная часть

\backmatter

	
\newcommand{\book}[6]{\bibitem{#1} #2~#3.~---~#4.:~#5~---~#6.}
\newcommand{\journal}[6]{\bibitem{#1} #2~#3.~//~#4~---~#5~---~№~#6.}
\newcommand{\url}[3]{\bibitem{#1} #2.~[Электронный~ресурс]~---~URL:~#3.~Дата~обращения:~\today.}

\begin{thebibliography}{0}

	% \book{nazarov}{Назаров А.С.}{Фотограмметрия: Учебное пособие}{Мн}{ТетраСистемс}{2006}
	% \journal{nikol}{Никольский Д.Б.}{Сравнительный обзор современных радиолокационных систем}{Геоматика}{2008}{1}
	% \url{srtm}{Описание и получение данных SRTM}{http://gis-lab.info/qa/srtm.html}

	\book{beresnev}{Береснев А.Л.}{Администрирование GNU/Linux с нуля}{СПб}{БХВ-Петербург}{2007}

	\url{wikipedia}{Википедия (англоязычный раздел)}{http://en.wikipedia.org}
	\url{wikipedia}{Википедия (русскоязычный раздел)}{http://ru.wikipedia.org}

	\book{olifer}{Олифер В.Г., Олифер Н.А.}{Компьютерные сети. Принципы, технологии, протоколы}{СПб}{Питер}{2010}

	\bibitem{labs}
		Гончаров В.А. Курс лабораторных работ по дисциплине <<Сети ЭВМ и системы телекоммуникаций>> --- Рязань: РГРТА
		--- 2000.

	\bibitem{lection}
		Калинкина Т.И. Курс лекций по дисциплине <<Вычислительные сети>> --- Рязань: РГРТУ --- 2010.

	\bibitem{oskurs}
		Акинин М.В. Системное программирование в \linux. Дистрибутив openSUSE 11.1 \linux. // Курсовой проект по
		дисциплине <<Операционные системы>>; научный руководитель: Засорин С.В. --- Рязань: РГРТУ --- 2009.

	\url{url-tcpdump}{Сниффер tcpdump - официальный сайт}{http://www.tcpdump.org}
	\url{url-wireshark}{Сниффер Wireshark - официальный сайт}{http://www.wireshark.org}
	\book{shildt}{Шилдт Г.}{Справочник программиста по C/C++}{М}{ООО <<И.Д. Вильямс>>}{2006}
	\url{public-data-network-numbers}{IANA IPv4 Address Space Registry}{http://www.iana.org/assignments/ipv4-address-space/ipv4-address-space. xhtml}

	\bibitem{rfc792}
		Internet Engineering Task Force. Internet Control Message Protocol // RFC 792 --- 1981.

	\bibitem{rfc5735}
		Internet Engineering Task Force. Special Use IPv4 Addresses // RFC 5735 --- 2010.

	\bibitem{rfc793}
		Internet Engineering Task Force. Transmission Control Protocol // RFC 793 --- 1981.

	\bibitem{rfc768}
		Internet Engineering Task Force. User Datagram Protocol // RFC 768 --- 1980.

	\book{granneman}{Граннеман С.}{Linux. Карманный справочник}{М}{ООО <<И.Д. Вильямс>>}{2008}

\end{thebibliography}



	\appendix
	
%		\myappendix{Описание прилагаемого DVD-диска} % TODO раскомментировать

%			
На DVD-диске, прилагаемом к записке по курсовому проекту, находятся следующие файлы и каталоги:

\begin{itemize}

	\item Каталог books - электронные версии некоторых источников из списка литературы:

		\begin{itemize}

			\item Файл akinin.pdf - \cite{oskurs};
			\item Файл granneman.djvu - \cite{granneman};
			\item Файл olifer.djvu - \cite{olifer};

		\end{itemize}

	\item Каталог soft - подборка программного обеспечения:

		\begin{itemize}

			\item Файл <<Foxit PDF.exe>> - утилита <<Foxit PDF>>, предназначенная для просмотра PDF-файлов;
			\item Каталог WinDjView и файлы WinDjView.exe и WinDjViewRU.dll, находящиеся в нем - утилита <<WinDjView>>, предназначенная для просмотра DjView-файлов;

		\end{itemize}

	\item Каталог src - файлы исходного кода демонстрационных программ.

		Все файлы, находящиеся в данном каталоге, суть есть текстовые файлы, сохраненные в кодировке UTF-8 с использованием Unix-соглашения по расстановке переносов строк (перенос строки - одиночный
		ASCII-символ с кодом 0xA).

		В каталоге src находятся следующие файлы:

		\begin{itemize}
		
			\item Makefile - файл автоматизации сборки демонстрационных программ;
			\item lab1.c - файл исходного кода демонстрационной программы к лабораторной работе № 1;
			\item lab2.c - файл исходного кода демонстрационной программы к лабораторной работе № 2;
			\item lab3-arp.c и lab3-icmp.c - файлы исходного кода демонстрационных программ к лабораторной работе № 3;
			\item lab4.c - файл исходного кода демонстрационной программы к лабораторной работе № 4;

		\end{itemize}

	\item Каталог vm, его подкаталог vmware и файлы, находящиеся в подкаталоге vmware каталога vm - эталонная виртуальная машина с предустановленной ОС GNU/Linux.

		Для функционирования эталонной виртуальной машины необходим программный комплекс \virtpo\ версии от 7 и выше.

		Имеющиеся пользователи в предустановленной ОС GNU/Linux:

		\begin{itemize}

			\item Пользователь root (суперпользователь) - пароль: toor;
			\item Пользователь user - пароль: user.

		\end{itemize}

		Для расшифровки контейнера с ключом шифрования к корневому разделу в ответ на соответствующий запрос программного комплекса LUKS при загрузке
		системы необходимо ввести пароль: <<djqyf b vbh>> (фраза <<война и мир>>, набранные без переключения клавиатуры в английскую раскладку);

	\item Файл report.pdf - электронная версия в формате PDF записки к курсовому проекту.

\end{itemize}

 % TODO раскомментировать

		\myappendix{Исходный код разработанного программного обеспечения}

			
% TODO раскоментировать

\mysource{../src/}{Makefile}
\mysource{../src/}{lab1.c}
\mysource{../src/}{lab2.c}
\mysource{../src/}{lab3-arp.c}
\mysource{../src/}{lab3-icmp.c}
\mysource{../src/}{lab4.c}

 % TODO раскомментировать

\end{document}

