
%\documentclass[12pt, utf8]{G7-32-kurs}
\documentclass[12pt]{article}
\usepackage[14pt]{extsizes}

%\usepackage[utf8]{inputenc}
%\usepackage[T2A]{fontenc}
%\usepackage[english, russian]{babel}

\usepackage[pdftex]{graphicx}
\usepackage{float}
\usepackage{pdflscape}
\usepackage{bytefield}
\usepackage{enumerate}

%\usepackage{indentfirst}
%\usepackage[multiple]{footmisc}

\usepackage{listingsutf8}
\lstloadlanguages{C}
\lstset
{
        language=C,
		breaklines,
		columns=fullflexible,
		flexiblecolumns,
		numbers=none,
        basicstyle=\tt\fontsize{12pt}{12pt}\selectfont,
        commentstyle=\bf,
        showtabs=false, 
        showspaces=false,
        showstringspaces=false,
        tabsize=2,
        inputencoding=utf8/cp1251,
		frame=single,
		showlines=true
}

%\usepackage{ltxtable}
%\usepackage{tabularx}
%\usepackage{longtable}
%\usepackage{array}
%\usepackage{multirow}
%\setlength{\extrarowheight}{2pt}
%\newcolumntype{Y}{>{\centering\arraybackslash}X}
%\newcolumntype{Z}{>{\small\centering\arraybackslash}p{1.5cm}}
%\newcolumntype{M}[1]{>{\small\centering\arraybackslash}p{#1}}
%\newcolumntype{N}{>{\small\centering\arraybackslash}X}
%\setlength{\LTcapwidth}{\textwidth}

\usepackage{amssymb,amsfonts,amsmath}
%\usepackage{amssymb,amsfonts,amsmath,mathtext,cite,enumerate}

%\EqInChapter
%\TableInChapter
%\PicInChapter

\usepackage{geometry}
\geometry{left=2cm}
\geometry{right=2cm}
\geometry{top=2cm}
\geometry{bottom=2cm}

\graphicspath{{image/}}

%\renewcommand{\theenumi}{\arabic{enumi}.}
%\renewcommand{\theenumii}{\arabic{enumii}}
%\renewcommand{\theenumiii}{.\arabic{enumiii}}
%\renewcommand{\theenumiv}{.\arabic{enumiv}}

%\renewcommand{\labelenumi}{\arabic{enumi}.}
%\renewcommand{\labelenumii}{\arabic{enumi}.\arabic{enumii}}
%\renewcommand{\labelenumiii}{\arabic{enumi}.\arabic{enumii}.\arabic{enumiii}}
%\renewcommand{\labelenumiv}{\arabic{enumi}.\arabic{enumii}.\arabic{enumiii}.\arabic{enumiv}}

%\renewcommand{\rmdefault}{mcr} % Courier New

\makeatletter 
\renewcommand\appendix
{
	\par
	\setcounter{chapter}{0}%
	\setcounter{section}{0}%
	\gdef\thechapter{\@Asbuk\c@chapter}
}
\makeatother

\newcommand\myappendix[1]
{
	\refstepcounter{chapter}
	
	\chapter*{\centering{Приложение~\thechapter}}
	\begin{center}
		\large \bf #1
	\end{center}
	
	\addcontentsline{toc}{chapter}{Приложение~\thechapter:~#1}
}

\newcommand{\mysource}[2]
{
        \refstepcounter{figure}
		\label{listing:#2}
		{
			\centering{Листинг~\thefigure~---~Содержимое~файла~<<#2>>}
			\nopagebreak
			\lstinputlisting[]{#1#2}
			\bigskip
		}
}

% \mimage{test}{test}{Тестовый рисунок}{width=\linewidth}

%\newcommand\mimage[4]
%{

%	\begin{center}
		
%		\begin{figure}[H]
	
%			\includegraphics[#4]{#2.png}
%			\caption{#3}
%			\label{image:#1}
	
%		\end{figure}

%	\end{center}
%}

\newcommand\mimage[4]
{
	\vbox
	{
		\begin{center}
			
			\refstepcounter{figure}
			\label{image:#1}
			\includegraphics[#4]{#2.png}

			{\noindent Рисунок~\thefigure~---~#3}

		\end{center}
	}
}



\newcommand{\linux}{ОС GNU/Linux}
\newcommand{\virtpo}{VMWare}

\newcommand{\first}{10.0.1.128}
\newcommand{\second}{10.0.2.128}
\newcommand{\midethf}{eth0}
\newcommand{\mideths}{eth1}
\newcommand{\midf}{10.0.1.1}
\newcommand{\mids}{10.0.2.1}

\newcommand{\mytitle}{<<Курс лабораторных работ\\по дисциплине <<Вычислительные сети>>\\для ОС GNU/Linux>>}
\newcommand{\tfirst}{Настройка подключения вычислительной системы к локальной вычислительной сети}
\newcommand{\tsecond}{Протоколы транспортного уровня стека протоколов TCP/IP}
\newcommand{\tthird}{Протоколы сетевого уровня стека протоколов TCP/IP}
\newcommand{\tfourth}{Перехват сетевого трафика. Межсетевой экран netfilter}

\begin{document}

% ############################################################################ 
% Вступление

\frontmatter

	
\NirOrgLongName{\textsc{Рязанский Государственный Радиотехнический Университет\\Кафедра ЭВМ}}

\NirManager{старший преподаватель кафедры ЭВМ}{С.И.Бабаев}

\NirTown{г. Рязань,}

\NirUdk{УДК \No~519.687.4, \No~519.688}	% Организация совместной работы нескольких вычислительных машин,
										% программы и алгоритмы для решения отдельных задач на вычислительных машинах
\NirGosNo{}

%\NirStage{Этап \No 5}{заключительный}{<<Записка по курсовому проекту>>}
% 1 - ТЗ
% 2 - Написание программы
% 3 - Отладка
% 4 - Тестирование
% 5 - Записка

\NirTitle{\textbf{\mytitle}}



	\Executors		% Исполнители

		\begin{longtable}{p{0.35\linewidth}p{0.2\linewidth}p{0.35\linewidth}}
		студент группы 740М, & & \\
		М.В. Акинин & \rule{1\linewidth}{0.1pt}& \\
		\end{longtable}

	\tableofcontents

% ############################################################################ 
% Основная часть

\mainmatter

	\chapter{Введение}

		
\section{Описание курса лабораторных работ}

	Данный курс лабораторных работ предназначен для студентов 3 - 5-го курсов, обучающихся по специальности <<Компьютерная безопасность>>
	(код специальности: 090102) и может быть также предложен студентам прочих специальностей, уровень подготовки которых удовлетворяет приведенным
	ниже требованиям.

	Данный курс лабораторных работ разработан мною как возможная альтернатива курсу лабораторных работ, рассчитанных на выполнение в ОС семейства
	Windows, поскольку использование ОС Windows для проведения лабораторных работ по дисциплине <<Вычислительные сети>>
	для подготовки специалистов по специальности 090102 имеет ряд недостатков:

	\begin{itemize}

		\item ОС GNU/Linux удобна в использовании, поскольку обладает дружественными и легкими в освоении консольным и графическими интерфейсами,
		в отличии от ОС Windows, консольный интерфейс управления которой существенно ограничивает возможности оператора по управлению ОС,
		а использование графического интерфейса управления вычислительной сетью представляется не всегда разумным;

		\item ОС Windows суть коммерческая ОС, отказ от использования которой при подготовке специалистов может существенным образом сэкономить средства
		ВУЗа;

		\item Развитие привычки работать с ОС Windows у будущих специалистов в области компьютерной безопасности может негативно сказаться на их
		карьерном будущем, так как решения, основанные на ОС Windows, не обладают достаточной степень защищенности и, следовательно, не могут использоваться
		для решения задач обеспечения безопасности информации в вычислительных системах и вычислительных сетях определенного назначения.

		Решения, позволяющие повысить уровень защищенности ОС Windows (SecretNet и прочие), также являются коммерческими, а, следовательно, ведут к
		дополнительным затратам организаций, их использующих, и необходимости их подробного освоения, что, очевидно, не может быть обеспечено в ходе подготовки
		специалистов по специальности 090102.

	\end{itemize}

	В свою очередь, ОС GNU/Linux обладает рядом достоинств:

	\begin{itemize}

		\item Удобные консольный и графические интерфейсы, позволяющие оператору ОС максимально удобным образом настроить интерфейс ОС под свои потребности;
		\item Огромный потенциал автоматизации рутинных действий с помощью разнообразных скриптовых языков, таких, как: Perl, Python, языки командных оболочек
		(Bash, Zsh);
		\item Широкий спектр сетевых возможностей ОС, включающий в себя, кроме всего прочего, поддержку большого количества разнообразного сетевого оборудования;
		\item Бесплатность;
		\item Распространение подавляющего большинства компонент системы в виде пакетов с файлами исходного кода компонент;
		\item Оперативное устранение уязвимостей в основных компонентах ОС таких, как ядро Linux и библиотека GLIBC;
		\item Возможность, для некоторых дистрибутивов, покупки некоторого срока техподдержки;
		\item Большое количество пользователей ОС, настроенных, в основном, дружелюбно и могущих дать ответ на возникший в процессе использования ОС вопрос.

	\end{itemize}

	Кроме того, необходимо отметить также факт того, что ОС GNU/Linux относится к классу Unix-подобных ОС, а, следовательно, навыки, полученные при использовании
	ОС GNU/Linux, студенты могут перенести и в прочие Unix-подобные ОС, такие, как FreeBSD, OpenBSD, NetBSD, Solaris и другие ОС.

	При выполнении предлагаемых лабораторных работ студенты приобретут следующие навыки:

	\begin{itemize}

		\item Навыки работы в командной строке ОС GNU/Linux;
		\item Навыки программирования на языке C в ОС GNU/Linux с применением средств стандартной библиотеки языка C;
		\item Навыки настройки подключения сетевого узла, управляемого ОС GNU/Linux, к вычислительной сети относительно сложной топологии;
		\item Навыки мониторинга состояния вычислительных сетей;
		\item Навыки сетевого программирования с использованием протоколов ARP, ICMP, TCP и UDP стека протоколов TCP/IPv4;
		\item Навыки перехвата сетевого трафика с помощью специальных программных комплексов - снифферов, навыки анализа перехваченного сетевого трафика;
		\item Навыки разработки собственных программных средств перехвата сетевого трафика.

	\end{itemize}

	Предлагаемый курс лабораторных работ состоит из четырех лабораторных работ из расчета один месяц на подготовку к лабораторной работе, на выполнение
	лабораторной работы и на защиту отчета по лабораторной работе. Курс лабораторных работ состоит из следующих работ:

	\begin{enumerate}[1.]

		\item {\bf<<Настройка подключения вычислительной системы к локальной вычислительной сети>>}.

			Данная лабораторная работы знакомит студентов с основными принципами настройки подключения сетевых узлов, управляемых ОС
			GNU/Linux, к вычислительным сетям достаточно сложной топологии. В данной лабораторной работе студенты получат необходимые базовые навыки
			программирования на языке C для ОС GNU/Linux, требующиеся для выполнения последующих лабораторных работ;

		\item {\bf<<Протоколы транспортного уровня стека протоколов TCP/IP>>}.

			Данная лабораторная работа посвящена протоколам TCP и UDP и принципам IP-адресации сетевых узлов и подсетей вычислительных сетей, использующих
			для сетевого обмена протоколы стека протоколов TCP/IPv4. В ходе выполнения данной лабораторной работы студенты ознакомятся с функционалом ОС
			GNU/Linux, позволяющим производить мониторинг сетевого обмена по протоколам TCP и UDP, позволяющим осуществлять простейший обмен между
			сетевыми узлами с использованием указанных протоколов, а также позволяющим решать прочие задачи по управлению сетевым обменом с использованием
			протоколов TCP и UDP. В рамках данной лабораторной работе студенты должны будут разработать многокомпонентный программный комплекс, организующий
			удаленное управление с одного из сетевых узлов вычислительной сети, используемой в ходе выполнения лабораторной работы, другими сетевыми узлами
			означенной вычислительной сети;

		\item {\bf<<Протоколы сетевого уровня стека протоколов TCP/IP>>}.

			Данная лабораторная работа посвящена протоколам IP, ICMP и ARP стека протоколов TCP/IP. В ходе выполнения данной лабораторной работы студенты
			ознакомятся с возможностями ОС GNU/Linux по определению факта доступности удаленного сетевого узла и определению некоторых характеристик возможного
			сетевого обмена с данным сетевым узлом. В ходе выполнения данной лабораторной работы студенты приобретут навыки работы с ARP-кэшами
			сетевых узлов, операторами которых они являются. В рамках данной лабораторной работы студенты должны будут разработать набор программ, использующих
			протокол ICMP для проверки доступности удаленного сетевого узла и протокол ARP для определения соответствия некоторому IP-адресу
			соответствующего MAC-адреса;
		
		\item {\bf<<Перехват сетевого трафика. Межсетевой экран netfilter>>}.

			Данная лабораторная работа посвящена снифферу tcpdump, межсетевому экрану netfilter и консольной утилитой управления межсетевым экраном netfilter - iptables, а также
			средствам ОС GNU/Linux, позволяющим программисту разработать собственный программный комплекс перехвата сетевого трафика.
			В ходе выполнения данной лабораторной работы студенты приобретут навыки использования сниффера tcpdump для интеллектуального перехвата сетевого
			трафика и получения информации о характере трафика и статистических закономерностях, прослеживающихся в содержании перехваченного трафика.
			В ходе выполнения данной лабораторной работы студенты получат навыки использования межсетевого экрана netfilter для оценки характера и объемов
			сетевого трафика, проходящего через сетевой узел. В ходе выполнения данной лабораторной работы студенты получат навыки разработки на
			языке программирования C собственных программных решений перехвата сетевого трафика, что является ценным опытом для будущего специалиста в области
			компьютерной безопасности.

	\end{enumerate}

	Студенты, приступающие к выполнению предлагаемого курса лабораторных работ, должны соответствовать следующим минимальным требованиям:

	\begin{itemize}

		\item Знание языка программирования C - на уровне одного семестра.

			Требования: знание основных компонент языка, умение работать с указателями, умение выделять и освобождать память с использованием
			функционала стандартной	библиотеки языка C, умение организовывать файловый ввод / вывод с использованием функционала стандартной
			библиотеки языка C;

		\item Навыки работы с консольным интерфейсом ОС GNU/Linux, Windows или DOS.

			Требование: знание основных команд (создание каталога, удаление каталога, переход в каталог и тому подобных).

	\end{itemize}

\section{Примеры выполнения заданий практических частей лабораторных работ}

	\subsection{Общая информация}

	В данном разделе приведены описания программ, разработанных мною по заданиям практических частей лабораторных работ.
	Каждая демонстрационная программа реализует, кроме основного задания
	практической части лабораторной работы, дополнительно одно из заданий, предназначенных для защиты отчета по лабораторной работе.
	
	Данный раздел демонстрирует практическую выполнимость заданий к лабораторным работам.
	
	Программа (программы) для каждой из лабораторных работ были
	разработаны (но не закомментированы) в течении сорока минут (лабораторная работа № 2: 52 минуты).
	Таким образом, имеем следующее временное расписание выполнения лабораторной работы студентом:

	\begin{enumerate}

		\item 1 - 2 дня по 4 - 5 часов - подготовка к лабораторной работе.

			Данный этап включает в себя изучение теоретической части методических указаний к лабораторной работе;

		\item Лабораторная работа:

			\begin{enumerate}

				\item 40 минут - выполнение практического части, не связанной с программированием;
				\item 1 час 20 минут - выполнение практической части, связанной с программированием.

					Данный этап подразумевает разработку и отладку программного обеспечения и демонстрацию преподавателю корректного выполнения разработанного программного обеспечения;

				\item 20 минут + 20 минут - защита отчета по предыдущей лабораторной работе.

			\end{enumerate}

	\end{enumerate}

	\subsection{Лабораторная работа № 1}

	В листинге \ref{listing:lab1.c} приведен исходный код программы, реализующей задание практической части лабораторной работы № 1 - получение содержимого
	таблицы маршрутизации сетевого узла с помощью файла /proc/net/route и вывод полученной таблицы в стандартный поток вывода.

	На рисунке \ref{image:demo-lab1} приведен выполнения демонстрационной программы.

	\mimage{demo-lab1}{demo/1}{Процесс выполнения демонстрационной программы}{width=0.95\textwidth}

	\subsection{Лабораторная работа № 2}

	В листинге \ref{listing:lab2.c} приведен исходный код программы, реализующей задание практической части лабораторной работы № 2 - удаленное управление
	серверными компонентами разработанного программного комплекса, запущенными на удаленных сетевых узлах, с помощью управляющей компоненты.

	На рисунке \ref{image:demo-lab2-1} приведен процесс выполнения управляющей компоненты.
	На рисунке \ref{image:demo-lab2-2} приведен процесс выполнения серверной компоненты, осуществляющей обмен с	управляющей компонентой по протоколу UDP.
	На рисунке \ref{image:demo-lab2-3} приведен процесс выполнения серверной компоненты, осуществляющей обмен с	управляющей компонентой по протоколу TCP.

	\mimage{demo-lab2-1}{demo/1000}{Процесс выполнения управляющей компоненты}{width=0.9\textwidth}
	\mimage{demo-lab2-2}{demo/1001}{Процесс выполнения серверной компоненты, осуществляющей обмен с управляющей компонентой по протоколу UDP}{width=\textwidth}
	\mimage{demo-lab2-3}{demo/1002}{Процесс выполнения серверной компоненты, осуществляющей обмен с управляющей компонентой по протоколу TCP}{width=\textwidth}

	\subsection{Лабораторная работа № 3}

	В листинге \ref{listing:lab3-arp.c} приведен исходный код программы, реализующей часть задания практической части лабораторной работы № 3 -
	отправление ARP-запроса и получение ARP-ответа. В листинге \ref{listing:lab3-icmp.c} приведен исходный код программы, реализующей часть
	задания практической части лабораторной работы № 3 - отправление пакетов типа ECHO протокола ICMP и получение ответных пакетов типа ECHO-REPLY
	протокола ICMP.

	На рисунке \ref{image:demo-lab3-arp} приведен выполнения демонстрационной программы, выполняющей отправление ARP-запроса и получение
	ARP-ответа. На рисунке \ref{image:demo-lab3-icmp} приведен выполнения демонстрационной программы, выполняющей отправление пакетов
	типа ECHO протокола ICMP и получение ответных пакетов типа ECHO-REPLY протокола ICMP. На рисунке \ref{image:demo-lab3-tcpdump} приведен вывод сниффера
	tcpdump, демонстрирующий корректность отправления и получения ответных пакетов соответствующих протоколов перечисленными демонстрационными программами.

	\mimage{demo-lab3-arp}{demo/1003}{Отправление ARP-запроса и получение ARP-ответа}{width=\textwidth}
	\mimage{demo-lab3-icmp}{demo/1004}{Отправление пакетов типа ECHO протокола ICMP и получение ответных пакетов типа ECHO-REPLY протокола ICMP}{width=\textwidth}
	\mimage{demo-lab3-tcpdump}{demo/1005}{Результат выполнения сниффера tcpdump}{width=\textwidth}

	\subsection{Лабораторная работа № 4}

	В листинге \ref{listing:lab4.c} приведен исходный код программы, реализующей задание практической части лабораторной работы № 4 - перехват пакетов
	протокола TCP.

	На рисунке \ref{image:demo-lab4} приведен выполнения демонстрационной программы. На рисунке \ref{image:demo-lab4-server}
	приведен процесс выполнения TCP-сервера на сетевом узле 10.0.1.128, на рисунке \ref{image:demo-lab4-client} приведен процесс выполнения клиента на сетевом
	узле 10.0.2.128, выполняющего подключение к TCP-серверу, запущенному на сетевом узле 10.0.1.128.

	\mimage{demo-lab4}{demo/704}{Процесс выполнения демонстрационной программы}{width=\textwidth}
	\mimage{demo-lab4-server}{demo/705}{Процесс выполнения TCP-сервера на сетевом узле 10.0.1.128}{width=\textwidth}
	\mimage{demo-lab4-client}{demo/706}{Процесс выполнения клиента на сетевом узле 10.0.2.128, выполняющего подключение к TCP-серверу, запущенному на сетевом узле 10.0.1.128}{width=\textwidth}



	\chapter{План выполнения лабораторных работ}

		\newcommand{\labplan}[2]
		{
			\section{Лабораторная работа № #1 <<#2>>}

				\subsection{Цель работы}

					
Ознакомление с протоколами транспортного уровня стека протоколов TCP/IP (в частности, с протоколами TCP и UDP).

Ознакомление с принципами IP-адресации сетевых узлов в вычислительных сетях, обмен в которых осуществляется с помощью протоколов стека протоколов TCP/IPv4.

Получение навыков использования утилиты netstat для определения состояния виртуальных TCP и UDP-портов сетевых интерфейсов вычислительной системы.

Получение навыков использования утилиты ncat для организации простейшего обмена данными по протоколам TCP и UDP
между операторами удаленных друг от друга вычислительных систем.

Ознакомление с содержимых файлов /etc/services и /etc/hosts. Получение навыков использования символьных псевдонимов IP-адресов сетевых узлов и
символьных псевдонимов виртуальных портов сетевых интерфейсов.

Получение навыков сетевого программирования с использованием протоколов TCP и UDP.



				\subsection{Теоретическая часть}

					
\subsubsection{Общие сведения}

	\linux\ является сетевой ОС и позволяет вычислительной системе, работающей под ее управлением,
	взаимодействовать с прочими вычислительными системами в сетях самой разнообразной организации.

	Так, например, \linux\ позволяет организовать функционирование вычислительной системы в составе локальной вычислительной
	сети, построенной по одной из вариаций технологии Ethernet и использующей протоколы стека протоколов
	TCP/IP\footnote{Допускается использование как протокола IPv4, так и протокола IPv6.}
	для взаимодействия узлов сети на сетевом и транспортном уровнях сетевой модели OSI.

	В общем случае, процесс функционирования вычислительной системы, работающей под управлением \linux,
	в качестве сетевого узла локальной вычислительной сети, основанной на технологии Ethernet и стеке протоколов
	TCP/IPv4, состоит из следующих этапов:

	\begin{enumerate}

		\item Физическое подключение сетевого адаптера к материнской плате вычислительной системы;

		\item Физическое подключение сетевого адаптера к каналу связи локальной вычислительной сети;

		\item Загрузка ОС вычислительной системы;

		\item Вход в ОС оператора в качестве суперпользователя;

		\item Настройка подключения сетевого адаптера к локальной вычислительной сети - создание сетевого узла на базе вычислительной системы. 
		
		Вычислительная система обеспечивает функционирование сетевого узла, подключенного через сетевой адаптер к локальной вычислительной сети.
		Сетевой адаптер доступен процессам пользовательского уровня ОС через соответствующий сетевой интерфейс.

		Данный этап предполагает:

			\begin{itemize}

				\item Задание IPv4-адреса настраиваемого сетевого узла;
				\item Задание маски сети\footnote{Маска сети суть есть целое беззнаковое число, позволяющее определить адрес сети и широковещательный адрес
				сети по адресу любого сетевого узла, состоящего в сети.} и широковещательного адреса сети\footnote{Широковещательный адрес сети используется
				при сетевом обмене для массовой рассылки пакетов всем сетевым узлам, состоящим в сети.}, в которую включен настраиваемый сетевой узел;
				\item Настройка дополнительных параметров сетевого узла.
				
				К числу дополнительных параметров сетевого узла относится, например, Maximum Transmission Unit
				(MTU) - максимальный размер пакета, могущего быть переданным через данный сетевым узлом или данному сетевому узлу
				по протоколу канального уровня сетевой модели OSI;

			\end{itemize}

		\item Настройка таблицы маршрутизации.

		С помощью таблицы маршрутизации вычислительная система принимает решение, по какому маршруту
		(какому сетевому узлу через какой сетевой интерфейс) необходимо передать очередной пакет данных, чтобы
		означенный пакет достиг сетевого узла - приемника;

		\item Запуск и настройка дополнительных сервисов.

		В число дополнительных сервисов входят, например, DNS-сервер, DHCP-сервер, WEB-сервер, FTP-сервер и прочие сервисы;

		\item Обмен данными между процессами вычислительной системы и процессами удаленных вычислительных систем;

		\item Останов дополнительных сервисов;

		\item Очистка таблицы маршрутизации;

		\item Сброс настроек сетевого интерфейса;

		\item Останов ОС.

	\end{enumerate}

	В данной лабораторной работе будет рассмотрен способ подключения вычислительной системы, работающей
	под управлением \linux, к локальной вычислительной сети, основанной на технологии Ethernet и
	стеке протоколов TCP/IPv4.

\subsubsection{Настройка подключения вычислительной системы к локальной вычислительной сети}

	Настройка подключения вычислительной системы к локальной вычислительной сети может быть произведена
	различными способами. Каждый из способов предполагает определенную степень автоматизации процесса подключения -
	от полностью ручного способа, рассматриваемого в данной лабораторной работе, до полностью автоматического,
	заключающегося в запуске операционной системой определенного набора программ с определенными ключами по
	составленному администратором ОС скрипту на некотором языке программирования\footnote{Чаще всего в качестве языка автоматизации
	выбирается язык командного интерпретатора, используемого администратором.}.

	Ручной способ настройки подключения вычислительной системы к локальной вычислительной сети предполагает следующую последовательность
	действий:

	\begin{enumerate}

		\item Вход оператора ОС в ОС в качестве суперпользователя;

		\item Настройка ядра ОС.

		Настройка ядра ОС предполагает загрузку в пространство ядра кода сетевой подсистемы ядра
		и драйвера используемого сетевого адаптера.
		
		Как правило, хотя и необязательно, код сетевой подсистемы ядра находится в файле ядра и,
		соответственно, загружается в ядерное пространство непосредственно в ходе загрузки ядра ОС на
		ранней стадии загрузки ОС. В том случае, если код сетевой подсистемы ядра скомпилирован в виде отдельных
		модулей ядра, подгружаемых в ходе работы ОС, данные модули загружаются в ходе запуска графической подсистемы
		ОС, поскольку она не может функционировать без сетевой подсистемы, или загружаются ОС автоматически,
		при выполнении попыток настройки сетевого адаптера. В крайне редких случаях, когда код сетевой подсистемы
		отсутствует в ядре, требуется перекомпилировать ядро ОС, что выходит за рамки данной лабораторной работы.

		Драйвер используемого сетевого адаптера может находится в файле ядра ОС, в отдельном модуле ядра,
		загружаемым ядром при выполнении попытки настройки сетевого адаптера, или отсутствовать в ядре вовсе.

		Определить, присутствует ли драйвер сетевого адаптера в пространстве ядра ОС, можно с помощью утилиты
		ifconfig - для этого необходимо запустить данную утилиту с ключом <<-a>>, который предписывает утилите
		ifconfig вывести в стандартный поток вывода информацию о всех присутствующих в вычислительной системе
		(но, возможно, ненастроенных) сетевых адаптерах.
		На рисунке \ref{image:lab1-5} приведен возможный вывод утилиты ifconfig, запущенной с ключом <<-a>>.

		\mimage{lab1-5}{1/5}{Список сетевых адаптеров, присутствующих в вычислительной системе}{}

		Как видно из рисунка \ref{image:lab1-5} к вычислительной системе подключены следующие сетевые адаптеры:

		\begin{itemize}

			\item eth0 - Ethernet-адаптер. IP-адрес отсутствует => адаптер не настроен;
			\item lo - виртуальный интерфейс обратной петли;
			\item vboxnet0 - виртуальный Ethernet-адаптер, используемый виртуальными машинами VirtualBox
				  (ни одна из таковых не запущена => vboxnet0 не настроен (IP-адрес отсутствует));
			\item vmnet1, vmnet8 - виртуальные Ethernet-адаптеры, используемые виртуальными машинами VMWare.
			Оба адаптера настроены и имеют следующие IP-адреса:

				\begin{itemize}

					\item vmnet1 - 192.168.126.1;
					\item vmnet8 - 192.168.91.1;

				\end{itemize}

			\item wlan0 - Wi-Fi карта (не настроена);

		\end{itemize}

		\item Настройка подключения сетевого интерфейса сетевого адаптера, с помощью которого выполняется подключение к локальной
		вычислительной сети, к локальной вычислительной сети.

		Означенная настройка предполагает:

		\begin{itemize}

			\item Задание IP-адреса настраиваемого сетевого узла;
			\item Задание маски сети и широковещательного адреса сети, в которую включен сетевой узел;
			\item Настройка дополнительных параметров сетевого узла.

		\end{itemize}

		Настройка сетевого интерфейса выполняется с помощью утилиты ifconfig и предполагает следующую последовательность действий.

		\begin{enumerate}

			\item Определение имени сетевого интерфейса.

			Интерфейсы Ethernet-адаптеров именуются в \linux\ в виде последовательности символов <<ethN>>, где N -
			порядковый номер сетевого адаптера (считая с 0), зависящий от его физического подключения к
			материнской плате вычислительной системы.

			Узнать имя сетевого интерфейса можно с помощью утилиты ifconfig, запущенной с ключом <<-a>>;

			\item Сброс настроек сетевого интерфейса.

			Для сброса настроек сетевого интерфейса необходимо выполнить команду <<ifconfig ethN down>>;

			\item Настройка сетевого интерфейса.

			Для настройки сетевого интерфейса необходимо выполнить команду <<ifconfig ethN IP broadcast BROADCAST netmask NETMASK>>, где:
			
			\begin{itemize}
			
				\item IP - IP-адрес сетевого узла;
				\item BROADCAST - широковещательный адрес сети;
				\item NETMASK - маска сети.

			\end{itemize}

			Передав утилите ifconfig дополнительные ключи, можно произвести настройку прочих параметров
			сетевого адаптера. Так, например, с помощью ключа <<mtu MTU\_SIZE>> оператор может настроить
			максимальный размер пакета (MTU\_SIZE; указывается в байтах),
			могущего быть переданным через сетевой адаптер по протоколу канального уровня сетевой модели OSI,
			что полезно для некоторых сетевых адаптеров отдельных производителей, требующих для своего корректного функционирования
			значение MTU меньшее, чем традиционное значение MTU, равное 1500 байтам.

		\end{enumerate}

		В случае успешного завершения процесса настройки сетевого интерфейса, запись о данном сетевом интерфейсе появится
		в списке корректно настроенных сетевых интерфейсов, могущего быть полученным с помощью утилиты ifconfig, запущенной без ключей,
		что проиллюстрировано рисунком \ref{image:lab1-4};

		\mimage{lab1-4}{1/4}{Список корректно настроенных сетевых интерфейсов вычислительной системы}{width=\textwidth}

		\item Настройка таблицы маршрутизации.

		После осуществления настройки сетевого интерфейса необходимо выполнить настройку таблицы маршрутизации, для чего используется утилита route.

		Одна запись, связанная с настраиваемым сетевым интерфейсом, в таблице маршрутизации уже есть -
		это указание направлять все пакеты, следующие в соответствующую сеть, через настроенный сетевой интерфейс.

		Необходимо добавить в таблицу маршрутизации запись о маршруте по умолчанию -
		то есть запись о маршруте, по которому будут отправляться пакеты в случае, если других маршрутов,
		по которым можно было бы отправить пакет, нет.
		
		Для добавления записи в таблицу маршрутизации о маршруте по умолчанию необходимо выполнить следующие действия:

		\begin{enumerate}

			\item Удалить запись о маршруте по умолчанию из таблицы маршрутизации.

			Для удаления записи о маршруте по умолчанию из таблицы маршрутизации необходимо выполнить команду
			<<route del default>>;
			
			\item Добавить запись о маршруте по умолчанию в таблицу маршрутизации
			с предписанием использовать целевой сетевой узел в качестве шлюза между сетями.

			Для добавления записи о маршруте по умолчанию в таблицу маршрутизации необходимо выполнить команду
			<<route add default gw IP ethN>>, где IP - IP-адрес сетевого узла - шлюза.

		\end{enumerate}

		Содержимое таблицы маршрутизации можно получить с помощью утилиты route, запустив данную утилиту без каких бы то ни было ключей.
		Пример получения содержимого таблицы маршрутизации с помощью утилиты route приведен на рисунке \ref{image:lab1-route}.

		\mimage{lab1-route}{1/route}{Состояние таблицы маршрутизации}{width=\textwidth}

	\end{enumerate}

\subsubsection{Проверка корректности функционирования локальной вычислительной сети}

	По завершению настройки вычислительной сети разумно проверить корректность ее функционирования,
	для чего можно использовать утилиту ping.
		
	Утилита ping предназначена для проверки доступности целевого сетевого узла и позволяет оценить
	некоторые параметры процесса обмена данными с целевым удаленным сетевым узлом.

	Для проверки доступности сетевого узла, объединенного с сетевым узлом, на котором производится
	запуск утилиты ping, в вычислительную сеть, утилите ping необходимо передать IP-адрес сетевого узла,
	доступность которого проверяется.

	На рисунке \ref{image:lab1-ping} приведен результат выполнения утилиты ping,
	запущенной на некотором сетевом узле последовательно с ключами <<-c 5 10.0.1.128>> и
	<<-c 5 10.0.2.128>>\footnote{Ключ <<-c 5>> суть есть указание использовать при проверки доступности
	целевого сетевого узла только 5 циклов обмена пакетами типов ECHO и ECHO-REPLY протокола ICMP.}.
	Как видно из рисунка \ref{image:lab1-ping}, сетевые узлы 10.0.1.128 и 10.0.2.128 доступны сетевому узлу, на котором запущена утилита ping,
	что позволяет говорить о корректности настроек локальной вычислительной сети.

	\mimage{lab1-ping}{1/ping}{Проверка корректности функционирования вычислительной сети}{width=\textwidth}



				\subsection{Практическая часть}

					
\subsubsection{Предварительная подготовка}

Для выполнения данной лабораторной работы должна быть использована вычислительная сеть, созданная и настроенная в ходе выполнения лабораторной работы № 1.
Структурная схема используемой вычислительной сети приведена на рисунке \ref{image:struct3}.

\mimage{struct3}{1/struct}{Структурная схема вычислительной сети}{width=\textwidth}

\subsubsection{Сетевое программирование с использованием протоколов ICMP и ARP. Утилиты ping и arp. Файл /proc/net/arp}
\label{task:l3t1}

	В рамках данной лабораторной работы каждому студенту группы необходимо выполнить на центральном сетевом узле
	следующие задания:

	\begin{enumerate}

		\item Проверить доступность каждого из периферийных сетевых узлов с помощью N циклов обмена с периферийными
		сетевыми узлами пакетами типа ECHO и ECHO-REPLY протокола ICMP;

		\item Получить содержимое ARP-кэша с помощью файла /proc/net/arp;

		\item Удалить из ARP-кэша записи об обоих периферийных сетевых узлах;

		\item Проверить корректность удаления записей из ARP-кэша с помощью утилиты arp;

		\item Разработать программу, выполняющую ARP-запрос о периферийном сетевом узле \linebreak \first;\label{lab3:pp-arp}

		\item Разработать программу, выполняющую обмен пакетами типов ECHO и ECHO-REPLY протокола ICMP с периферийным
		узлом \second.\label{lab3:pp-icmp}
		
		Программа должна выполнять последовательно N циклов обмена пакетами означенного типа с
		периферийным узлом \second\ и для каждого принятого или отправленного пакета программа должна выводить
		содержимое заголовка пакета;

		\item Запустить на выполнение программы, разработанные в пунктах \ref{lab3:pp-arp} и \ref{lab3:pp-icmp};

		\item По завершению выполнения программ, разработанных в пунктах \ref{lab3:pp-arp} и \ref{lab3:pp-icmp},
		получить содержимое ARP-кэша с помощью файла /proc/net/arp;

		\item Получить содержимое ARP-кэша с помощью утилиты arp;

		\item Сделать выводы о принципах функционирования ARP-кэша.

	\end{enumerate}

	Здесь N суть есть: $N = (M~mod~5 + 1) * 4$, где M - порядковый номер студента в журнале группы.

\subsubsection{Подготовка отчета по лабораторной работе}

	Подготовить отчет, в котором необходимо привести подробное описание процесса выполнения заданий пункта
	\ref{task:l3t1}, проиллюстрированное достаточным количеством снимков экрана.

	В отчете по лабораторной работе должен быть приведен исходный код разработанного программного обеспечения с
	соответствующими комментариями.



				\subsection{Защита отчета по лабораторной работе}

					
Одно из следующих заданий может быть предложено в качестве задания для защиты отчета по лабораторной работе:

\begin{enumerate}

	\item Переработка серверной компоненты, осуществлявшей обмен информацией с управляющей компонентой по протоколу
	UDP, для использования при означенном обмене протокола TCP;

	\item Переработка серверной компоненты, осуществлявшей обмен информацией с управляющей компонентой по протоколу
	TCP, для использования при означенном обмене протокола UDP;

	\item Добавление функционала обработки управляющей компонентой одной из следующих команд:

		\begin{itemize}

			\item <<get NODE FILE\_IN FILE\_OUT>> - команда запроса у серверной компоненты, запущенной на периферийном
			сетевом узле с номером NODE, содержимого файла FILE\_IN и сохранение полученной информации на центральном
			сетевом узле в файле FILE\_OUT;

			\item <<put NODE FILE\_IN FILE\_OUT>> - копирование файла FILE\_IN центрального сетевого узла в файл FILE\_OUT
			периферийного сетевого узла с номером NODE;

			\item <<route NODE>> - запрос у серверной компоненты, функционирующей на периферийном сетевом узле с номером
			NODE, таблицы маршрутизации данного периферийного сетевого узла;

		\end{itemize}

	\item Добавление поддержки обмена сообщениями между оператором управляющей компоненты и операторами сетевых узлов,
	на которых запущены серверные компоненты;

	\item Добавление поддержки симметричного шифрования данных, пересылаемых между серверными и управляющей компонентами.

		В качестве простейшего алгоритма симметричного шифрования можно использовать <<затенение>> байт с помощью
		операции побитового исключающего или.

		Ключи шифрования должны быть сгенерированы управляющей компонентой и разосланы серверным компонентам
		непосредственно после установа соединения между серверными и управляющей компонентами.

\end{enumerate}


		}

		\labplan{1}{\tfirst}
		\labplan{2}{\tsecond}
		\labplan{3}{\tthird}
		\labplan{4}{\tfourth}

% ############################################################################ 
% Заключительная часть

\backmatter

	
\newcommand{\book}[6]{\bibitem{#1} #2~#3.~---~#4.:~#5~---~#6.}
\newcommand{\journal}[6]{\bibitem{#1} #2~#3.~//~#4~---~#5~---~№~#6.}
\newcommand{\url}[3]{\bibitem{#1} #2.~[Электронный~ресурс]~---~URL:~#3.~Дата~обращения:~\today.}

\begin{thebibliography}{0}

	% \book{nazarov}{Назаров А.С.}{Фотограмметрия: Учебное пособие}{Мн}{ТетраСистемс}{2006}
	% \journal{nikol}{Никольский Д.Б.}{Сравнительный обзор современных радиолокационных систем}{Геоматика}{2008}{1}
	% \url{srtm}{Описание и получение данных SRTM}{http://gis-lab.info/qa/srtm.html}

	\book{beresnev}{Береснев А.Л.}{Администрирование GNU/Linux с нуля}{СПб}{БХВ-Петербург}{2007}

	\url{wikipedia}{Википедия (англоязычный раздел)}{http://en.wikipedia.org}
	\url{wikipedia}{Википедия (русскоязычный раздел)}{http://ru.wikipedia.org}

	\book{olifer}{Олифер В.Г., Олифер Н.А.}{Компьютерные сети. Принципы, технологии, протоколы}{СПб}{Питер}{2010}

	\bibitem{labs}
		Гончаров В.А. Курс лабораторных работ по дисциплине <<Сети ЭВМ и системы телекоммуникаций>> --- Рязань: РГРТА
		--- 2000.

	\bibitem{lection}
		Калинкина Т.И. Курс лекций по дисциплине <<Вычислительные сети>> --- Рязань: РГРТУ --- 2010.

	\bibitem{oskurs}
		Акинин М.В. Системное программирование в \linux. Дистрибутив openSUSE 11.1 \linux. // Курсовой проект по
		дисциплине <<Операционные системы>>; научный руководитель: Засорин С.В. --- Рязань: РГРТУ --- 2009.

	\url{url-tcpdump}{Сниффер tcpdump - официальный сайт}{http://www.tcpdump.org}
	\url{url-wireshark}{Сниффер Wireshark - официальный сайт}{http://www.wireshark.org}
	\book{shildt}{Шилдт Г.}{Справочник программиста по C/C++}{М}{ООО <<И.Д. Вильямс>>}{2006}
	\url{public-data-network-numbers}{IANA IPv4 Address Space Registry}{http://www.iana.org/assignments/ipv4-address-space/ipv4-address-space. xhtml}

	\bibitem{rfc792}
		Internet Engineering Task Force. Internet Control Message Protocol // RFC 792 --- 1981.

	\bibitem{rfc5735}
		Internet Engineering Task Force. Special Use IPv4 Addresses // RFC 5735 --- 2010.

	\bibitem{rfc793}
		Internet Engineering Task Force. Transmission Control Protocol // RFC 793 --- 1981.

	\bibitem{rfc768}
		Internet Engineering Task Force. User Datagram Protocol // RFC 768 --- 1980.

	\book{granneman}{Граннеман С.}{Linux. Карманный справочник}{М}{ООО <<И.Д. Вильямс>>}{2008}

\end{thebibliography}



	\appendix
	
%		\myappendix{Описание прилагаемого DVD-диска} % TODO раскомментировать

%			
На DVD-диске, прилагаемом к записке по курсовому проекту, находятся следующие файлы и каталоги:

\begin{itemize}

	\item Каталог books - электронные версии некоторых источников из списка литературы:

		\begin{itemize}

			\item Файл akinin.pdf - \cite{oskurs};
			\item Файл granneman.djvu - \cite{granneman};
			\item Файл olifer.djvu - \cite{olifer};

		\end{itemize}

	\item Каталог soft - подборка программного обеспечения:

		\begin{itemize}

			\item Файл <<Foxit PDF.exe>> - утилита <<Foxit PDF>>, предназначенная для просмотра PDF-файлов;
			\item Каталог WinDjView и файлы WinDjView.exe и WinDjViewRU.dll, находящиеся в нем - утилита <<WinDjView>>, предназначенная для просмотра DjView-файлов;

		\end{itemize}

	\item Каталог src - файлы исходного кода демонстрационных программ.

		Все файлы, находящиеся в данном каталоге, суть есть текстовые файлы, сохраненные в кодировке UTF-8 с использованием Unix-соглашения по расстановке переносов строк (перенос строки - одиночный
		ASCII-символ с кодом 0xA).

		В каталоге src находятся следующие файлы:

		\begin{itemize}
		
			\item Makefile - файл автоматизации сборки демонстрационных программ;
			\item lab1.c - файл исходного кода демонстрационной программы к лабораторной работе № 1;
			\item lab2.c - файл исходного кода демонстрационной программы к лабораторной работе № 2;
			\item lab3-arp.c и lab3-icmp.c - файлы исходного кода демонстрационных программ к лабораторной работе № 3;
			\item lab4.c - файл исходного кода демонстрационной программы к лабораторной работе № 4;

		\end{itemize}

	\item Каталог vm, его подкаталог vmware и файлы, находящиеся в подкаталоге vmware каталога vm - эталонная виртуальная машина с предустановленной ОС GNU/Linux.

		Для функционирования эталонной виртуальной машины необходим программный комплекс \virtpo\ версии от 7 и выше.

		Имеющиеся пользователи в предустановленной ОС GNU/Linux:

		\begin{itemize}

			\item Пользователь root (суперпользователь) - пароль: toor;
			\item Пользователь user - пароль: user.

		\end{itemize}

		Для расшифровки контейнера с ключом шифрования к корневому разделу в ответ на соответствующий запрос программного комплекса LUKS при загрузке
		системы необходимо ввести пароль: <<djqyf b vbh>> (фраза <<война и мир>>, набранные без переключения клавиатуры в английскую раскладку);

	\item Файл report.pdf - электронная версия в формате PDF записки к курсовому проекту.

\end{itemize}

 % TODO раскомментировать

		\myappendix{Исходный код разработанного программного обеспечения}

			
% TODO раскоментировать

\mysource{../src/}{Makefile}
\mysource{../src/}{lab1.c}
\mysource{../src/}{lab2.c}
\mysource{../src/}{lab3-arp.c}
\mysource{../src/}{lab3-icmp.c}
\mysource{../src/}{lab4.c}

 % TODO раскомментировать

\end{document}

