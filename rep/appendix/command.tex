
Пользователи, существующие в установках ОС GNU/Linux, используемых в настоящем курсе лабораторных работ:

\begin{itemize}

	\item Суперпользователь root (пароль - toor);
	\item Пользователь user (пароль - user).

\end{itemize}

ОС GNU/Linux позволяет пользователю работать с несколькими независимыми друг от друга виртуальными консольными и графическими терминалами. Переключение между консольными виртуальными терминалами осуществляется с помощью сочетаний клавиш <<Alt + FНОМЕР>> и <<Ctrl + Alt + FНОМЕР>>, где <<НОМЕР>> - номер виртуального терминала, на который оператор ОС желает переключится. Так, например, нажатие клавиш <<Alt + F1>> позволяет переключится на первый консольный виртуальный терминал, <<Alt + F2>> - на второй и <<Alt + F3>> - на третий.

Следующие команды могут оказаться полезными при работе с ОС GNU/Linux:

\begin{itemize}

	\item <<telinit 0>> - останов ОС, выключение компьютера (необходимы права суперпользователя);
	\item <<telinit 6>> - перезагрузка ОС (необходимы права суперпользователя);
	\item <<gcc ФАЙЛ.c -o ФАЙЛ\_2>> - компиляция программы, написанной на языке программирования C, исходный код которой сохранен в файле ФАЙЛ.c. На выходе генерируется исполняемый файл ФАЙЛ\_2.
	
		Для запуска на выполнение исполняемого файла ФАЙЛ\_2 оператор должен выполнить команду <<./ФАЙЛ\_2>>;

	\item <<mc>> - запуск файлового менеджера <<Midnight Commander>>, обладающего, кроме всего прочего, удобным текстовым редактором с подсветкой синтаксиса различных языков программирования;
	\item <<nano>> - запуск текстового редактора Nano;
	\item <<su ->> - вход в ОС в качестве суперпользователя;
	\item <<man СТРАНИЦА>> - получение справочной информации (содержимое man-страницы <<СТРАНИЦА>>; например, команда <<man printf>> позволяет получить справочную информацию о функции printf() стандартной библиотеки языка программирования C).
	
		Для завершения выполнения утилиты man необходимо нажать клавишу <<q>>;

	\item <<ps -A>> - получить список всех процессов системы, существующих в ОС на момент вызова утилиты <<ps>>;

	\item <<kill PID>> или <<killall ИМЯ\_ИСПОЛНЯЕМОГО\_ФАЙЛА>> - останов процесса, обладающего указанным идентификатором (PID) или соответствующего указанному исполняемому файлу.

		Другой способ останова некоторого процесса ОС - нажатие сочетания клавиш <<Ctrl + C>> в том виртуальном терминале, в котором процесс запущен. Нажатие указанного сочетания клавиш позволяет останавливать только те процессы ОС, которые были запущены на <<переднем плане>> (то есть не в фоновом режиме); 

	\item <<net\_help>> - получение дополнительной справочной информации.

		Команда <<net\_help>> уникальна для установок ОС GNU/Linux, используемых в данном курсе лабораторных работ.

\end{itemize}

При запуске исполняемых файлов на выполнение в конце командной строки не стоит указывать символ <<\&>> - если означенный символ будет указан, то процесс, соответствующий исполняемому файлу, будет запущен в фоновом режиме и, таким образом, будет, в рамках данного курса лабораторных работ, бесполезен.

