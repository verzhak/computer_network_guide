
На DVD-диске, прилагаемом к записке по курсовому проекту, находятся следующие файлы и каталоги:

\begin{itemize}

	\item Каталог books - электронные версии некоторых источников из списка литературы:

		\begin{itemize}

			\item Файл akinin.pdf - \cite{oskurs};
			\item Файл granneman.djvu - \cite{granneman};
			\item Файл olifer.djvu - \cite{olifer};

		\end{itemize}

	\item Каталог soft - подборка программного обеспечения:

		\begin{itemize}

			\item Файл <<Foxit PDF.exe>> - утилита <<Foxit PDF>>, предназначенная для просмотра PDF-файлов;
			\item Каталог WinDjView и файлы WinDjView.exe и WinDjViewRU.dll, находящиеся в нем - утилита <<WinDjView>>, предназначенная для просмотра DjView-файлов;

		\end{itemize}

	\item Каталог src - файлы исходного кода демонстрационных программ.

		Все файлы, находящиеся в данном каталоге, суть есть текстовые файлы, сохраненные в кодировке UTF-8 с использованием Unix-соглашения по расстановке переносов строк (перенос строки - одиночный
		ASCII-символ с кодом 0xA).

		В каталоге src находятся следующие файлы:

		\begin{itemize}
		
			\item Makefile - файл автоматизации сборки демонстрационных программ;
			\item lab1.c - файл исходного кода демонстрационной программы к лабораторной работе № 1;
			\item lab2.c - файл исходного кода демонстрационной программы к лабораторной работе № 2;
			\item lab3-arp.c и lab3-icmp.c - файлы исходного кода демонстрационных программ к лабораторной работе № 3;
			\item lab4.c - файл исходного кода демонстрационной программы к лабораторной работе № 4;

		\end{itemize}

	\item Каталог vm, его подкаталог vmware и файлы, находящиеся в подкаталоге vmware каталога vm - эталонная виртуальная машина с предустановленной ОС GNU/Linux.

		Для функционирования эталонной виртуальной машины необходим программный комплекс \virtpo\ версии от 7 и выше.

		Имеющиеся пользователи в предустановленной ОС GNU/Linux:

		\begin{itemize}

			\item Пользователь root (суперпользователь) - пароль: toor;
			\item Пользователь user - пароль: user.

		\end{itemize}

		Для расшифровки контейнера с ключом шифрования к корневому разделу в ответ на соответствующий запрос программного комплекса LUKS при загрузке
		системы необходимо ввести пароль: <<djqyf b vbh>> (фраза <<война и мир>>, набранные без переключения клавиатуры в английскую раскладку);

	\item Файл report.pdf - электронная версия в формате PDF записки к курсовому проекту.

\end{itemize}

