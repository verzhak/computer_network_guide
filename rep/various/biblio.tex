
\newcommand{\book}[6]{\bibitem{#1} #2~#3.~---~#4.:~#5~---~#6.}
\newcommand{\journal}[6]{\bibitem{#1} #2~#3.~//~#4~---~#5~---~№~#6.}
\newcommand{\url}[3]{\bibitem{#1} #2.~[Электронный~ресурс]~---~URL:~#3.~Дата~обращения:~\today.}

\begin{thebibliography}{0}

	% \book{nazarov}{Назаров А.С.}{Фотограмметрия: Учебное пособие}{Мн}{ТетраСистемс}{2006}
	% \journal{nikol}{Никольский Д.Б.}{Сравнительный обзор современных радиолокационных систем}{Геоматика}{2008}{1}
	% \url{srtm}{Описание и получение данных SRTM}{http://gis-lab.info/qa/srtm.html}

	\book{beresnev}{Береснев А.Л.}{Администрирование GNU/Linux с нуля}{СПб}{БХВ-Петербург}{2007}

	\url{wikipedia}{Википедия (англоязычный раздел)}{http://en.wikipedia.org}
	\url{wikipedia}{Википедия (русскоязычный раздел)}{http://ru.wikipedia.org}

	\book{olifer}{Олифер В.Г., Олифер Н.А.}{Компьютерные сети. Принципы, технологии, протоколы}{СПб}{Питер}{2010}

	\bibitem{labs}
		Гончаров В.А. Курс лабораторных работ по дисциплине <<Сети ЭВМ и системы телекоммуникаций>> --- Рязань: РГРТА
		--- 2000.

	\bibitem{lection}
		Калинкина Т.И. Курс лекций по дисциплине <<Вычислительные сети>> --- Рязань: РГРТУ --- 2010.

	\bibitem{oskurs}
		Акинин М.В. Системное программирование в \linux. Дистрибутив openSUSE 11.1 \linux. // Курсовой проект по
		дисциплине <<Операционные системы>>; научный руководитель: Засорин С.В. --- Рязань: РГРТУ --- 2009.

	\url{url-tcpdump}{Сниффер tcpdump - официальный сайт}{http://www.tcpdump.org}
	\url{url-wireshark}{Сниффер Wireshark - официальный сайт}{http://www.wireshark.org}
	\book{shildt}{Шилдт Г.}{Справочник программиста по C/C++}{М}{ООО <<И.Д. Вильямс>>}{2006}
	\url{public-data-network-numbers}{IANA IPv4 Address Space Registry}{http://www.iana.org/assignments/ipv4-address-space/ipv4-address-space. xhtml}

	\bibitem{rfc792}
		Internet Engineering Task Force. Internet Control Message Protocol // RFC 792 --- 1981.

	\bibitem{rfc5735}
		Internet Engineering Task Force. Special Use IPv4 Addresses // RFC 5735 --- 2010.

	\bibitem{rfc793}
		Internet Engineering Task Force. Transmission Control Protocol // RFC 793 --- 1981.

	\bibitem{rfc768}
		Internet Engineering Task Force. User Datagram Protocol // RFC 768 --- 1980.

	\book{granneman}{Граннеман С.}{Linux. Карманный справочник}{М}{ООО <<И.Д. Вильямс>>}{2008}

\end{thebibliography}

