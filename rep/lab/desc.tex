
\section{Описание курса лабораторных работ}

	Данный курс лабораторных работ предназначен для студентов 3 - 5-го курсов, обучающихся по специальности <<Компьютерная безопасность>>
	(код специальности: 090102) и может быть также предложен студентам прочих специальностей, уровень подготовки которых удовлетворяет приведенным
	ниже требованиям.

	Данный курс лабораторных работ разработан мною как возможная альтернатива курсу лабораторных работ, рассчитанных на выполнение в ОС семейства
	Windows, поскольку использование ОС Windows для проведения лабораторных работ по дисциплине <<Вычислительные сети>>
	для подготовки специалистов по специальности 090102 имеет ряд недостатков:

	\begin{itemize}

		\item ОС GNU/Linux удобна в использовании, поскольку обладает дружественными и легкими в освоении консольным и графическими интерфейсами,
		в отличии от ОС Windows, консольный интерфейс управления которой существенно ограничивает возможности оператора по управлению ОС,
		а использование графического интерфейса управления вычислительной сетью представляется не всегда разумным;

		\item ОС Windows суть коммерческая ОС, отказ от использования которой при подготовке специалистов может существенным образом сэкономить средства
		ВУЗа;

		\item Развитие привычки работать с ОС Windows у будущих специалистов в области компьютерной безопасности может негативно сказаться на их
		карьерном будущем, так как решения, основанные на ОС Windows, не обладают достаточной степень защищенности и, следовательно, не могут использоваться
		для решения задач обеспечения безопасности информации в вычислительных системах и вычислительных сетях определенного назначения.

		Решения, позволяющие повысить уровень защищенности ОС Windows (SecretNet и прочие), также являются коммерческими, а, следовательно, ведут к
		дополнительным затратам организаций, их использующих, и необходимости их подробного освоения, что, очевидно, не может быть обеспечено в ходе подготовки
		специалистов по специальности 090102.

	\end{itemize}

	В свою очередь, ОС GNU/Linux обладает рядом достоинств:

	\begin{itemize}

		\item Удобные консольный и графические интерфейсы, позволяющие оператору ОС максимально удобным образом настроить интерфейс ОС под свои потребности;
		\item Огромный потенциал автоматизации рутинных действий с помощью разнообразных скриптовых языков, таких, как: Perl, Python, языки командных оболочек
		(Bash, Zsh);
		\item Широкий спектр сетевых возможностей ОС, включающий в себя, кроме всего прочего, поддержку большого количества разнообразного сетевого оборудования;
		\item Бесплатность;
		\item Распространение подавляющего большинства компонент системы в виде пакетов с файлами исходного кода компонент;
		\item Оперативное устранение уязвимостей в основных компонентах ОС таких, как ядро Linux и библиотека GLIBC;
		\item Возможность, для некоторых дистрибутивов, покупки некоторого срока техподдержки;
		\item Большое количество пользователей ОС, настроенных, в основном, дружелюбно и могущих дать ответ на возникший в процессе использования ОС вопрос.

	\end{itemize}

	Кроме того, необходимо отметить также факт того, что ОС GNU/Linux относится к классу Unix-подобных ОС, а, следовательно, навыки, полученные при использовании
	ОС GNU/Linux, студенты могут перенести и в прочие Unix-подобные ОС, такие, как FreeBSD, OpenBSD, NetBSD, Solaris и другие ОС.

	При выполнении предлагаемых лабораторных работ студенты приобретут следующие навыки:

	\begin{itemize}

		\item Навыки работы в командной строке ОС GNU/Linux;
		\item Навыки программирования на языке C в ОС GNU/Linux с применением средств стандартной библиотеки языка C;
		\item Навыки настройки подключения сетевого узла, управляемого ОС GNU/Linux, к вычислительной сети относительно сложной топологии;
		\item Навыки мониторинга состояния вычислительных сетей;
		\item Навыки сетевого программирования с использованием протоколов ARP, ICMP, TCP и UDP стека протоколов TCP/IPv4;
		\item Навыки перехвата сетевого трафика с помощью специальных программных комплексов - снифферов, навыки анализа перехваченного сетевого трафика;
		\item Навыки разработки собственных программных средств перехвата сетевого трафика.

	\end{itemize}

	Предлагаемый курс лабораторных работ состоит из четырех лабораторных работ из расчета один месяц на подготовку к лабораторной работе, на выполнение
	лабораторной работы и на защиту отчета по лабораторной работе. Курс лабораторных работ состоит из следующих работ:

	\begin{enumerate}[1.]

		\item {\bf<<Настройка подключения вычислительной системы к локальной вычислительной сети>>}.

			Данная лабораторная работы знакомит студентов с основными принципами настройки подключения сетевых узлов, управляемых ОС
			GNU/Linux, к вычислительным сетям достаточно сложной топологии. В данной лабораторной работе студенты получат необходимые базовые навыки
			программирования на языке C для ОС GNU/Linux, требующиеся для выполнения последующих лабораторных работ;

		\item {\bf<<Протоколы транспортного уровня стека протоколов TCP/IP>>}.

			Данная лабораторная работа посвящена протоколам TCP и UDP и принципам IP-адресации сетевых узлов и подсетей вычислительных сетей, использующих
			для сетевого обмена протоколы стека протоколов TCP/IPv4. В ходе выполнения данной лабораторной работы студенты ознакомятся с функционалом ОС
			GNU/Linux, позволяющим производить мониторинг сетевого обмена по протоколам TCP и UDP, позволяющим осуществлять простейший обмен между
			сетевыми узлами с использованием указанных протоколов, а также позволяющим решать прочие задачи по управлению сетевым обменом с использованием
			протоколов TCP и UDP. В рамках данной лабораторной работе студенты должны будут разработать многокомпонентный программный комплекс, организующий
			удаленное управление с одного из сетевых узлов вычислительной сети, используемой в ходе выполнения лабораторной работы, другими сетевыми узлами
			означенной вычислительной сети;

		\item {\bf<<Протоколы сетевого уровня стека протоколов TCP/IP>>}.

			Данная лабораторная работа посвящена протоколам IP, ICMP и ARP стека протоколов TCP/IP. В ходе выполнения данной лабораторной работы студенты
			ознакомятся с возможностями ОС GNU/Linux по определению факта доступности удаленного сетевого узла и определению некоторых характеристик возможного
			сетевого обмена с данным сетевым узлом. В ходе выполнения данной лабораторной работы студенты приобретут навыки работы с ARP-кэшами
			сетевых узлов, операторами которых они являются. В рамках данной лабораторной работы студенты должны будут разработать набор программ, использующих
			протокол ICMP для проверки доступности удаленного сетевого узла и протокол ARP для определения соответствия некоторому IP-адресу
			соответствующего MAC-адреса;
		
		\item {\bf<<Перехват сетевого трафика. Межсетевой экран netfilter>>}.

			Данная лабораторная работа посвящена снифферу tcpdump, межсетевому экрану netfilter и консольной утилитой управления межсетевым экраном netfilter - iptables, а также
			средствам ОС GNU/Linux, позволяющим программисту разработать собственный программный комплекс перехвата сетевого трафика.
			В ходе выполнения данной лабораторной работы студенты приобретут навыки использования сниффера tcpdump для интеллектуального перехвата сетевого
			трафика и получения информации о характере трафика и статистических закономерностях, прослеживающихся в содержании перехваченного трафика.
			В ходе выполнения данной лабораторной работы студенты получат навыки использования межсетевого экрана netfilter для оценки характера и объемов
			сетевого трафика, проходящего через сетевой узел. В ходе выполнения данной лабораторной работы студенты получат навыки разработки на
			языке программирования C собственных программных решений перехвата сетевого трафика, что является ценным опытом для будущего специалиста в области
			компьютерной безопасности.

	\end{enumerate}

	Студенты, приступающие к выполнению предлагаемого курса лабораторных работ, должны соответствовать следующим минимальным требованиям:

	\begin{itemize}

		\item Знание языка программирования C - на уровне одного семестра.

			Требования: знание основных компонент языка, умение работать с указателями, умение выделять и освобождать память с использованием
			функционала стандартной	библиотеки языка C, умение организовывать файловый ввод / вывод с использованием функционала стандартной
			библиотеки языка C;

		\item Навыки работы с консольным интерфейсом ОС GNU/Linux, Windows или DOS.

			Требование: знание основных команд (создание каталога, удаление каталога, переход в каталог и тому подобных).

	\end{itemize}

\section{Примеры выполнения заданий практических частей лабораторных работ}

	\subsection{Общая информация}

	В данном разделе приведены описания программ, разработанных мною по заданиям практических частей лабораторных работ.
	Каждая демонстрационная программа реализует, кроме основного задания
	практической части лабораторной работы, дополнительно одно из заданий, предназначенных для защиты отчета по лабораторной работе.
	
	Данный раздел демонстрирует практическую выполнимость заданий к лабораторным работам.
	
	Программа (программы) для каждой из лабораторных работ были
	разработаны (но не закомментированы) в течении сорока минут (лабораторная работа № 2: 52 минуты).
	Таким образом, имеем следующее временное расписание выполнения лабораторной работы студентом:

	\begin{enumerate}

		\item 1 - 2 дня по 4 - 5 часов - подготовка к лабораторной работе.

			Данный этап включает в себя изучение теоретической части методических указаний к лабораторной работе;

		\item Лабораторная работа:

			\begin{enumerate}

				\item 40 минут - выполнение практического части, не связанной с программированием;
				\item 1 час 20 минут - выполнение практической части, связанной с программированием.

					Данный этап подразумевает разработку и отладку программного обеспечения и демонстрацию преподавателю корректного выполнения разработанного программного обеспечения;

				\item 20 минут + 20 минут - защита отчета по предыдущей лабораторной работе.

			\end{enumerate}

	\end{enumerate}

	\subsection{Лабораторная работа № 1}

	В листинге \ref{listing:lab1.c} приведен исходный код программы, реализующей задание практической части лабораторной работы № 1 - получение содержимого
	таблицы маршрутизации сетевого узла с помощью файла /proc/net/route и вывод полученной таблицы в стандартный поток вывода.

	На рисунке \ref{image:demo-lab1} приведен выполнения демонстрационной программы.

	\mimage{demo-lab1}{demo/1}{Процесс выполнения демонстрационной программы}{width=0.95\textwidth}

	\subsection{Лабораторная работа № 2}

	В листинге \ref{listing:lab2.c} приведен исходный код программы, реализующей задание практической части лабораторной работы № 2 - удаленное управление
	серверными компонентами разработанного программного комплекса, запущенными на удаленных сетевых узлах, с помощью управляющей компоненты.

	На рисунке \ref{image:demo-lab2-1} приведен процесс выполнения управляющей компоненты.
	На рисунке \ref{image:demo-lab2-2} приведен процесс выполнения серверной компоненты, осуществляющей обмен с	управляющей компонентой по протоколу UDP.
	На рисунке \ref{image:demo-lab2-3} приведен процесс выполнения серверной компоненты, осуществляющей обмен с	управляющей компонентой по протоколу TCP.

	\mimage{demo-lab2-1}{demo/1000}{Процесс выполнения управляющей компоненты}{width=0.9\textwidth}
	\mimage{demo-lab2-2}{demo/1001}{Процесс выполнения серверной компоненты, осуществляющей обмен с управляющей компонентой по протоколу UDP}{width=\textwidth}
	\mimage{demo-lab2-3}{demo/1002}{Процесс выполнения серверной компоненты, осуществляющей обмен с управляющей компонентой по протоколу TCP}{width=\textwidth}

	\subsection{Лабораторная работа № 3}

	В листинге \ref{listing:lab3-arp.c} приведен исходный код программы, реализующей часть задания практической части лабораторной работы № 3 -
	отправление ARP-запроса и получение ARP-ответа. В листинге \ref{listing:lab3-icmp.c} приведен исходный код программы, реализующей часть
	задания практической части лабораторной работы № 3 - отправление пакетов типа ECHO протокола ICMP и получение ответных пакетов типа ECHO-REPLY
	протокола ICMP.

	На рисунке \ref{image:demo-lab3-arp} приведен выполнения демонстрационной программы, выполняющей отправление ARP-запроса и получение
	ARP-ответа. На рисунке \ref{image:demo-lab3-icmp} приведен выполнения демонстрационной программы, выполняющей отправление пакетов
	типа ECHO протокола ICMP и получение ответных пакетов типа ECHO-REPLY протокола ICMP. На рисунке \ref{image:demo-lab3-tcpdump} приведен вывод сниффера
	tcpdump, демонстрирующий корректность отправления и получения ответных пакетов соответствующих протоколов перечисленными демонстрационными программами.

	\mimage{demo-lab3-arp}{demo/1003}{Отправление ARP-запроса и получение ARP-ответа}{width=\textwidth}
	\mimage{demo-lab3-icmp}{demo/1004}{Отправление пакетов типа ECHO протокола ICMP и получение ответных пакетов типа ECHO-REPLY протокола ICMP}{width=\textwidth}
	\mimage{demo-lab3-tcpdump}{demo/1005}{Результат выполнения сниффера tcpdump}{width=\textwidth}

	\subsection{Лабораторная работа № 4}

	В листинге \ref{listing:lab4.c} приведен исходный код программы, реализующей задание практической части лабораторной работы № 4 - перехват пакетов
	протокола TCP.

	На рисунке \ref{image:demo-lab4} приведен выполнения демонстрационной программы. На рисунке \ref{image:demo-lab4-server}
	приведен процесс выполнения TCP-сервера на сетевом узле 10.0.1.128, на рисунке \ref{image:demo-lab4-client} приведен процесс выполнения клиента на сетевом
	узле 10.0.2.128, выполняющего подключение к TCP-серверу, запущенному на сетевом узле 10.0.1.128.

	\mimage{demo-lab4}{demo/704}{Процесс выполнения демонстрационной программы}{width=\textwidth}
	\mimage{demo-lab4-server}{demo/705}{Процесс выполнения TCP-сервера на сетевом узле 10.0.1.128}{width=\textwidth}
	\mimage{demo-lab4-client}{demo/706}{Процесс выполнения клиента на сетевом узле 10.0.2.128, выполняющего подключение к TCP-серверу, запущенному на сетевом узле 10.0.1.128}{width=\textwidth}

