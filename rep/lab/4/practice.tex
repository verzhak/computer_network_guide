
\subsubsection{Предварительная подготовка}

Для выполнения данной лабораторной работы должна быть использована вычислительная сеть, созданная и настроенная в ходе
выполнения лабораторной работы № 1. Структурная схема используемой вычислительной сети приведена на рисунке
\ref{image:struct4}.

\mimage{struct4}{1/struct}{Структурная схема вычислительной сети}{width=\textwidth}

\subsubsection{Перехват ICMP-трафика}
\label{task:l4t1}

	Для перехвата ICMP-трафика, проходящего через сетевой адаптер \midethf\ центрального сетевого узла,
	на данном узле должна быть запущена утилита tcpdump с аргументами <<-i \midethf\ icmp>>, что проиллюстрировано
	рисунком \ref{image:6} (утилита ping, в данном случае, запущена на периферийном узле \first\ с целевым IP-адресом
	равным \second\ - пакеты типов ECHO и ECHO-REPLY протокола ICMP следуют через центральный сетевой узел).
	
	Ключ <<-i>> позволяет оператору предписать утилите tcpdump осуществлять перехват пакетов, проходящих через строго
	определенный сетевой интерфейс. С помощью аргумента <<icmp>> утилите tcpdump может быть предписано осуществлять
	перехват пакетов только протокола ICMP.

	Для явного указания типа перехватываемых пакетов протокола ICMP утилите tcpdump, запускаемой на центральном сетевом
	узле, необходимо передать следующие аргументы: <<-i \midethf\ 'icmp[icmptype] = icmp-echo'>> - в данном случае
	будут перехвачены только ICMP-ECHO пакеты, что проиллюстрировано рисунком \ref{image:7}.

	Конструкции <<icmp>> и <<icmp[icmptype] = icmp-echo>> описывают маски, в соответствии с которыми выполняется
	перехват пакетов, проходящих через целевой сетевой интерфейс. Одним из достоинств сниффера tcpdump являются широкие
	возможности задания масок перехвата пакетов.
	
	Для завершения выполнения утилиты tcpdump ей необходимо отправить сигнал SIGINT нажатием комбинации клавиш
	<<Ctrl + C>> в терминале, в котором утилита tcpdump была запущена.

	\mimage{6}{4/6}{Перехват пакетов протокола ICMP}{width=\textwidth}
	\mimage{7}{4/7}{Перехват пакетов типа ECHO протокола ICMP}{width=\textwidth}

\subsubsection{Перехват и анализ трафика, порождаемого протоколами транспортного уровня стека протоколов TCP/IP}
\label{task:l4t2}

	Одной из важнейших задач, решаемых снифферами, является задача перехвата пакетов протоколов транспортного уровня,
	так как эти пакеты могут содержать интересующие оператора пользовательские данные.

	Для иллюстрации возможностей утилиты tcpdump по перехвату пакетов протокола TCP на периферийном узле
	\first\ необходимо запустить утилиту ncat с аргументами <<-l 7777>>, а на периферийном узле \second\ - утилиту ncat
	с аргументами <<\first\ 7777>> - таким образом, на сетевом узле \first\ будет запущен TCP-сервер, к которому
	подключится клиент, запущенный на сетевом узле \second.
	Как видно из рисунков \ref{image:10} и \ref{image:11} сервер и клиент выполняют обмен строками символов,
	после чего их выполнение завершается вводом оператором сочетания клавиш <<Ctrl + C>> в терминалах, в которых
	данные сервер и клиент были запущены.

	На рисунках \ref{image:8} и \ref{image:9} приведен вывод утилиты tcpdump, запущенной на центральном сетевом узле
	со следующими аргументами:

		\begin{itemize}

			\item <<-n>> - предписывает утилите tcpdump не преобразовывать никакие численные характеристики процесса
			обмена в их символьные представления (например, не преобразовывать номера портов в названия сервисов с
			помощью файла /etc/protocols);
			\item <<-i \midethf>> - предписывает утилите tcpdump прослушивать сетевой интерфейс \midethf\ центрального
			сетевого узла;
			\item <<-A>> - предписывает утилите tcpdump выводить содержимое полей данных перехваченных пакетов в виде
			последовательности ASCII-символов;
			\item <<tcp>> - предписывает утилите tcpdump перехватывать пакеты только протокола TCP.

		\end{itemize}

	Как видно из рисунков \ref{image:8} и \ref{image:9}, для каждого перехваченного пакета утилита tcpdump вывела:
	
		\begin{itemize}
		
			\item Временную метку перехвата пакета;
			\item IP-адреса отправителя и получателя;
			\item Содержимое отдельных полей заголовка пакета;
			\item Содержимое поля данных пакета.

		\end{itemize}

	\mimage{10}{4/10}{Процесс выполнения TCP-сервера на сетевом узле \first}{width=\textwidth}
	\mimage{11}{4/11}{Процесс выполнения TCP-клиента на сетевом узле \second}{width=\textwidth}
	\mimage{8}{4/8}{Вывод утилиты tcpdump}{width=\textwidth}
	\mimage{9}{4/9}{Вывод утилиты tcpdump}{width=\textwidth}

\subsubsection{Перехват и статистическая обработка трафика}
\label{task:l4t3}

	Утилита tcpdump не обладает сколь-нибудь серьезным функционалом статистической обработки перехваченного трафика.
	Впрочем, консольная неинтерактивная	природа утилиты tcpdump позволяет эффективным образом использовать для
	статистической обработки перехваченного трафика другие утилиты, имеющие консольный неинтерактивный пользовательский
	интерфейс.

	На практике перед системным администратором может быть поставлена задача оценки соотношения объемов пройденного
	через сетевой адаптер служебного и прикладного трафика. Под прикладным трафиком будем понимать трафик, порождаемый
	протоколами TCP и UDP, под служебным трафиком - трафик, порождаемый прочими протоколами, пакеты которых вкладываются
	в пакеты протокола IP. В качестве целевого сетевого адаптера выберем сетевой адаптер \midethf\ центрального сетевого
	узла.

	Для оценки соотношения объемов пройденного через сетевой адаптер служебного и прикладного трафика можно
	использовать межсетевой экран netfilter, включенный в состав ядра Linux.
	Межсетевой экран netfilter выполняет подсчет количеств и суммарных объемов пакетов, к которым были применены
	те или иные правила. Кроме того, netfilter подсчитывает количества и суммарные объемы пакетов, над которыми были
	совершены действия по умолчанию имеющихся цепочек правил.

	На рисунке \ref{image:12} приведен процесс добавления правил netfilter на центральном сетевом узле -
	с помощью данных правил будет проверена правильность рассчета отношения объемов служебного трафика,
	пройденного через сетевой адаптер \midethf\ центрального сетевого узла, к пройденному через тот же сетевой адаптер
	прикладного трафика на основе информации, предоставляемой утилитой tcpdump.
	Добавление правил межсетевого экрана должно быть выполнено с помощью следующих команд:

	\begin{enumerate}

		\item <<iptables -F>> - очистка все цепочек правил таблицы filter межсетевого экрана.
		
		Без явного указания обрабатываемой таблицы с помощью ключа -t утилита iptables работает с таблицей filter
		межсетевого экрана. Таблица filter состоит из следующих цепочек правил: input, forward и output. Правила,
		применяемые к пакетам, пробрасываемым сетевым узлом между сетевыми адаптерами, помещаются в цепочку forward;

		\item <<iptables -P FORWARD ACCEPT>> - определение действия по умолчанию (принимать; ACCEPT) цепочки правил
		forward таблицы filter;

		\item <<iptables -A FORWARD -i \midethf\ -p tcp -j ACCEPT>>,
		<<iptables -A FORWARD -i \midethf\ -p udp -j ACCEPT>>,
		<<iptables -A FORWARD -o \midethf\ -p tcp -j ACCEPT>>, <<iptables -A FORWARD -o \midethf\ -p udp -j ACCEPT>> -
		правила, предписывающие цепочке правил forward (ключ <<-A FORWARD) таблицы filter межсетевого экрана принимать
		(ключ <<-j ACCEPT>>) пакеты протоколов TCP (ключ <<-p tcp>>) и UDP (ключ <<-p udp>>) идующие от (ключ
		<<-i \midethf>>) сетевого адаптера \midethf\ или к (ключ <<-o \midethf>>) сетевому адаптеру \midethf.

	\end{enumerate}

	По умолчанию, после добавления перечисленных правил счетчики количеств и суммарных объемов пакетов, прошедших через
	данные правила, установлены в 0. Для принудительного сброса означенных счетчиков в 0 в процессе отладки правил
	netfilter можно воспользоваться утилитой iptables, передав той ключ <<-Z>>.

	\mimage{12}{4/12}{Добавление правил межсетевого экрана netfilter центрального сетевого узла}{width=\textwidth}

	Для подсчета трафика, проходящего через сетевой адаптер \midethf\ центрального сетевого узла, на данном узле должен
	быть запущен сниффер tcpdump со следующими ключами:

	\begin{itemize}

		\item <<-v>> - предписывает выводить дополнительную информацию о перехватываемых пакетах (в частности,
		информацию о протоколе, пакеты которого вкладываются в пакеты протокола IP);

		\item <<-n>> - предписывает утилите tcpdump не преобразовывать никакие численные характеристики процесса обмена
		в их символьные представления (например, не преобразовывать номера портов в названия сервисов с помощью файла
		/etc/protocols);

		\item <<-i \midethf>> - предписывает утилите tcpdump прослушивать сетевой интерфейс \midethf\ центрального
		сетевого узла;

		\item << > dump>> - предписывает командной оболочке перенаправить стандартный поток вывода сниффера tcpdump
		в файл <<dump>>.

	\end{itemize}

	Процесс запуска сниффера tcpdump на центральном сетевом узле приведен на рисунке \ref{image:13}.

	\mimage{13}{4/13}{Процесс запуска сниффера tcpdump на центральном сетевом узле}{width=\textwidth}

	На рисунках \ref{image:17} и \ref{image:18} приведены:

	\begin{itemize}

		\item Рисунок \ref{image:17} - процесс функционирования TCP-сервера и UDP-клиента на сетевом узле \first\
		и процесс проверки доступности сетевым узлом \first\ сетевого узла \second\ с помощью утилиты ping;
		\item Рисунок \ref{image:18} - процесс функционирования TCP-клиента и UDP-сервера на сетевом узле \second.

	\end{itemize}

	\mimage{17}{4/17}{Процесс функционирования TCP-сервера и UDP-клиента на сетевом узле \first\
	и процесс проверки доступности сетевым узлом \first\ сетевого узла \second\ с помощью утилиты ping}
	{width=0.95\textwidth}
	\mimage{18}{4/18}{Процесс функционирования TCP-клиента и UDP-сервера на сетевом узле \second}{width=0.95\textwidth}

	После того, как обмен между узлами \first\ и \second\ будет завершен, работу сниффера tcpdump необходимо прекратить
	нажатием сочетания клавиш <<Ctrl + C>> в том терминале центрального сетевого узла, на котором tcpdump будет запущен.

	На рисунке \ref{image:19} приведен процесс вычисления отношения объема служебного трафика к объему прикладного
	трафика с использованием информации, собранной сниффером tcpdump. Процесс вычисления означенного отношения состоит
	из следующих этапов:

	\begin{enumerate}

		\item Подсчет суммарного размера пакетов, прошедших через сетевой интерфейс \midethf\ центрального сетевого узла.

		Подсчет суммарного размера пакетов производится с помощью утилит echo, grep, sed и bc следующим образом:

		\begin{lstlisting}{language=bash}
echo `grep "IP.*proto" dump | sed 's/.*length \([0-9]*\).*/\1 + /'` 0 | bc
		\end{lstlisting}
		
		Здесь:

		\begin{itemize}

			\item Утилита grep выбирает все строки файла dump, содержащие подстроки <<IP>> и <<proto>>.
			В конце таковых строк после подстроки <<length>> указан размер в байтах соответствующего пакета;

			\item Утилита sed выделяет из выбранных утилитой grep строк размеры в байтах соответствующих пакетов и
			добавляет справа к выбранным подстрокам	пробел и знак <<+>>;

			\item Утилита echo добавляет справа к выводу утилиты sed цифру <<0>> (чтобы крайний правый оператор <<+>>
			в выводе утилиты sed имел кроме левого операнда, еще и правый операнд) и передает с помощью канала
			получившееся выражение утилите bc;

			\item Утилита bc подсчитывает суммарный размер в байтах пакетов, прошедших через сетевой интерфейс
			\midethf\ центрального сетевого узла, и выводит получившееся число в стандартный поток вывода;

		\end{itemize}

		\item Подсчет суммарного размера пакетов протоколов TCP и UDP, прошедших через сетевой интерфейс
		\midethf\ центрального сетевого узла.

		Подсчет суммарного размера пакетов протоколов TCP и UDP выполняется схожим образом, что и подсчет суммарного
		размера всех пакетов, прошедших через сетевой интерфейс \midethf\ центрального сетевого узла - разница состоит
		лишь в виде регулярного выражения, с помощью которого grep выделяет записи
		об интересующих оператора пакетах. Имеет место быть следующая команда:

		\begin{lstlisting}{language=bash}
echo `grep "IP.*proto.*\(TCP\|UDP\)" dump | sed 's/.*length \([0-9]*\).*/\1 + /'` 0 | bc
		\end{lstlisting}

		Здесь утилита grep выбирает все строки файла dump, содержащие подстроки IP и proto, а также одну из подстрок:
		TCP или UDP (последовательность	символов <<$\backslash|$>> рассматривается как оператор логического ИЛИ;
		последовательности символов <<$\backslash($>> и <<$\backslash)$>>, в данном случае, рассматриваются как
		группирующие скобки);

		\item Подсчет суммарного размера пакетов прочих протоколов, пакеты которых вкладываются в пакеты протокола IP,
		прошедших через сетевой интерфейс \midethf\ центрального сетевого узла.

		Как и в предыдущих двух случаях, подсчет размера пакетов прочих протоколов выполняется с помощью утилит echo,
		grep, sed и bc:

		\begin{lstlisting}{language=bash}
echo `grep -v "\(TCP\|UDP\)" dump | grep "IP.*proto" | sed 's/.*length \([0-9]*\).*/\1 + /'` 0 | bc
		\end{lstlisting}

		В отличие от предыдущих двух случаев, утилита grep применена здесь два раза - первый раз для выбора всех строк
		из файла dump, не содержащих (ключ <<-v>>) одну из двух подстрок: <<TCP>> или <<UDP>> - второй раз -
		для выбора из таковых строк всех строк, содержащих подстроки <<IP>> и <<proto>>;

		\item Подсчет отношения объемов служебного и прикладного трафика, прошедшего через сетевой интерфейс
		\midethf\ центрального сетевого узла.

		Окончательный рассчет отношения объемов служебного и прикладного трафика, прошедшего через сетевой интерфейс
		\midethf\ центрального сетевого узла, осуществляется с помощью утилит echo и bc - утилита echo передает для
		вычисления утилите bc означенное отношение.

	\end{enumerate}

	\mimage{19}{4/19}{Подсчет отношения объемов служебного и прикладного трафика с использованием информации,
	собранной сниффером tcpdump}{width=\textwidth}

	Проверка полученного отношения объема служебного трафика, прошедшего через сетевой адаптер \midethf\ центрального
	сетевого узла, к объему прикладного трафика, прошедшего через тот же сетевой адаптер, выполняется на основании
	информации, собранной межсетевым экраном netfilter, что проиллюстрировано рисунком \ref{image:20}.

	С помощью утилиты iptables, запускаемой с ключами <<-n -v -L>>, оператор может получить сведения о количестве и
	суммарном объеме пакетов, прошедших через цепочки правил таблицы filter - в том числе, прошедших через добавленные
	ранее правила.

	Окончательный подсчет целевого соотношения выполняется, как и в предыдущем случае, с помощью утилит echo и bc.

	\mimage{20}{4/20}{Подсчет отношения объемов служебного и прикладного трафика друг к другу с использованием
	информации, собранной межсетевым экраном netfilter}{width=\textwidth}

\subsubsection{Использование пакетных сокетов для перехвата сетевого трафика}
\label{task:l4t4}

	Необходимо разработать программу, выполняющую перехват пакетов протокола UDP, проходящих через один из
	сетевых адаптеров центрального сетевого узла, на котором программа должна быть запущена.
	Программа должна выводить в стандартный поток вывода для каждого перехваченного пакета содержимое
	полей заголовков IP и UDP пакетов, а также результаты указанной обработки поля <<Data>> перехваченного пакета.
	Для иллюстрации корректности выполнения программы необходимо организовать сетевой обмен между операторами
	переферийных сетевых узлов с помощью утилиты ncat.

	В соответствии с увеличенным на единицу остатком от деления на 7 порядкового номера студента в журнале группы,
	программа каждого студента должна выполнять одну из следующих обработок перехваченных пакетов протокола
	UDP:

	\begin{enumerate}[1.]

		\item Изменять порядок символов строк, передаваемых в полях <<Data>> UDP-пакетов, с прямого на обратный;
		\item Вычислять сумму кодов символов, составляющих строки, передаваемые в полях <<Data>> UDP-пакетов;
		\item Выбирать из строк, передаваемых в полях <<Data>> UDP-пакетов, символы латинского алфавита;
		\item Переводить все символы, передаваемые в полях <<Data>> UDP-пакетов, в верхний регистр;
		\item Удалять из строк, передаваемых в полях <<Data>> UDP-пакетов, все последовательности из трех и более
		согласных букв;
		\item Определять коды символов строк, передаваемых в полях <<Data>> UDP-пакетов, наименьших в соответствующей
		строке по модулю 15;
		\item Разделять строки, передаваемые в полях <<Data>> UDP-пакетов, на слова по традиционным разделителям
		слов в тексте.

	\end{enumerate}

\subsubsection{Подготовка отчета по лабораторной работе}

	Подготовить отчет, в котором необходимо привести подробные описания процессов выполнения заданий пунктов
	\ref{task:l4t1}, \ref{task:l4t2}, \ref{task:l4t3} и \ref{task:l4t4}, проиллюстрированные достаточным
	количеством снимков экрана.

	В отчете по лабораторной работе должен быть приведен исходный код разработанного программного обеспечения
	с соответствующими комментариями.

