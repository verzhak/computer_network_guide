
\subsubsection{Перехват и анализ сетевого трафика}

	Для решения задач обеспечения безопасности передачи информации в вычислительных сетях различных устройства и
	назначения специалист в области компьютерной безопасности может воспользоваться инструментарием перехвата и
	анализа сетевого трафика.

	Инструментарий перехвата и анализа сетевого трафика позволяет администратору сети и лицу, ответственному за
	безопасность сетевого обмена в защищаемой вычислительной сети, решать следующие задачи:

	\begin{itemize}

		\item Отладка организации сети и используемого метода маршрутизации трафика сети;
		\item Оценка пропускной способности сети;
		\item Определение источников паразитного трафика;
		\item Определение типа и содержания трафика, генерируемого различными сетевыми узлами, входящими в
		вычислительную сеть;
		\item Прочие задачи.

	\end{itemize}

	Для решения задач перехвата и анализа сетевого трафика, в том числе трафика, порождаемого протоколами стека
	протоколов TCP/IP, в ОС GNU/Linux можно воспользоваться следующим программным обеспечением:

	\begin{enumerate}

		\item Сниффер tcpdump.

		Сниффер tcpdump представляет собой консольную неинтерактивную утилиту со следующими возможностями:
			
			\begin{itemize}

				\item Перехват трафика, порождаемого протоколами стека протоколов TCP/IP;
				\item Условный перехват трафика (перехват пакетов, удовлетворяющих определенным условиям);
				\item Разбор заголовка перехваченных пакетов;
				\item Вывод содержимого перехваченных пакетов в стандартный поток вывода или в файл;

			\end{itemize}

		\item Сниффер Wireshark\footnote{Предыдущее название - Ethereal}.

		Сниффер Wireshark обладает графическим интерактивным интерфейсом (библиотека GTK+)
		и предоставляет возможности, сходные с возможностями tcpdump. Важным отличием Wireshark от tcpdump, кроме
		графического интерфейса, являются существенно превосходящие	возможности статистического анализа трафика;

		\item Библиотека libpcap для языков программирования C и C++.

		Библиотека libpcap суть есть основа сниффера tcpdump и позволяет программисту разрабатывать собственное
		программное обеспечение, выполняющее перехват и анализ сетевого трафика;

		\item Пакетные сокеты.

		Для решения задач перехвата и анализа сетевого трафика могут быть использованы пакетные сокеты.
		Использование пакетных сокетов является более трудоемкой операцией по сравнению с использованием
		библиотеки libpcap, поскольку в случае использования означенных сокетов программист должен самостоятельно
		реализовывать функционал фильтрации поступающего трафика и функционал декомпозиции заголовков пакетов;

		\item Межсетевой экран netfilter (действия LOG, ULOG, QUEUE и NFQUEUE);

		\item Сетевая подсистема ОС на уровне ядра ОС.

	\end{enumerate}

	При использовании сниффера в ОС GNU/Linux необходимо помнить, что запускать сниффер и управлять его работой
	имеет право только суперпользователь.

\subsubsection{Разработка программного обеспечения, выполняющего перехват и анализ сетевого трафика}

	Процессы ОС, желающие выполнять перехват и анализ сетевого трафика, проходящего через сетевые адаптеры
	вычислительной системы, могут воспользоваться одним из следующих программных средств:

	\begin{itemize}

		\item Библиотека libpcap;
		\item Пакетные сокеты.

	\end{itemize}

	Поскольку библиотека libpcap сама использует пакетные сокеты для перехвата трафика,
	в данной лабораторной работе будут рассмотрены возможности \linux\ по перехвату сетевого трафика
	с использованием пакетных сокетов.

	В общем случае, процесс перехвата сетевого трафика, проходящего через определенный сетевой адаптер вычислительной
	системы, состоит из следующих этапов:

	\begin{enumerate}

		\item Создание пакетного сокета.

			Для создания пакетного сокета необходимо использовать системный вызов socket, передав ему следующие
			аргументы:

			\begin{itemize}

				\item Идентификатор семейства протоколов - AF\_PACKET;
				\item Идентификатор режима передачи пакетов - SOCK\_RAW, для перехвата пакетов протоколов
				канального уровня со вложенными в них пакетами целевого протокола сетевого уровня,
				или SOCK\_DGRAM, для перехвата пакетов целевого протокола сетевого уровня;
				\item Идентификатор протокола сетевого уровня, пакеты которого будут перехватываться.

					В качестве идентификатора протокола сетевого уровня, пакеты которого будут перехватываться,
					в сетях, построенных на основе технологии Ethernet, необходимо использовать значение константы
					с именем вида ETH\_P\_*, преобразованное к своему представлению в сетевом порядке байт. Константы
					с именами вида ETH\_P\_* определены в заголовочном файле <linux/ if\_ether.h>. В качестве
					означенного идентификатора протокола можно использовать значения следующих констант:

					\begin{itemize}

						\item ETH\_P\_IP - протокол IPv4;
						\item ETH\_P\_IPV6 - протокол IPv6;
						\item ETH\_P\_ARP - протокол ARP;
						\item ETH\_P\_RARP - протокол RARP;
						\item ETH\_P\_ALL - протокол не уточняется (любой протокол).

					\end{itemize}

			\end{itemize}

			Для создания пакетного сокета, с помощью которого будет осуществляться перехват пакетов протокола IPv4,
			системному вызову socket необходимо передать параметры AF\_PACKET и ETH\_P\_IP;

		\item Перевод целевого сетевого адаптера в неразборчивый (promiscuous) режим.
		\label{lab4:sprom}

			Обычный режим функционирования сетевых адаптеров предполагает выполнение ими некоторой предварительной
			обработки поступающих пакетов перед передачей пакетов сетевой подсистеме ОС. Так, например, сетевые адаптеры
			отбрасывают пакеты, заведомо не предназначающиеся вычислительной системе, в составе которой данные сетевые
			адаптеры функционируют. Для выполнения перехвата и анализа всего сетевого трафика, проходящего через
			целевой сетевой адаптер, означенный сетевой адаптер необходимо перевести в неразборчивый режим,
			при котором весь сетевой трафик будет передаваться в сетевую подсистему ОС.

			Перевод целевого сетевого адаптера в неразборчивый режим осуществляется в несколько этапов:

			\begin{enumerate}

				\item Определение символьного имени сетевого интерфейса целевого сетевого адаптера;
				\item Создание экземпляра структуры данных ifreq, описанной в заголовочном файле <net/if.h>.

					Экземпляр структуры данных ifreq в дальнейшем будет содержать маску параметров целевого
					сетевого адаптера;

				\item Копирование символьного имени сетевого интерфейса в поле ifr\_name экземпляра структуры
				данных ifreq;
				\item Получение маски параметров целевого сетевого адаптера с помощью системного вызова ioctl
				и действия SIOCGIFFLAGS.

					Пример использования системного вызова ioctl для получения номера сетевого интерфейса
					с помощью действия SIOCGIFINDEX приведен в предыдущей лабораторной работе;

				\item Установ флага IFF\_PROMISC в поле ifr\_flags экземпляра структуры данных ifreq.

					Константа IFF\_PROMISC определена в заголовочном файле <net/if.h>;

				\item Установ маски параметров целевого сетевого адаптера с помощью системного вызова ioctl
				и действия SIOCSIFFLAGS;

			\end{enumerate}

		\item Чтение содержимого пакетов, проходящих через целевой сетевой адаптер.

			Чтение содержимого пакетов, проходящих через целевой сетевой адаптер, выполняется с помощью
			системного вызова recv.
			
			При чтении процессом ОС пакетов, пришедших на сетевой адаптер,
			работающий в неразборчивом режиме, ОС не будет удалять прочитанные пакеты из своей сетевой подсистемы
			сразу после чтения их процессом, а будет лишь помечать данные пакеты как прочитанные данным процессом
			ОС.

			Необходимо помнить, что ОС будет помещать в результирующий буфер заголовки пакетов
			следующих уровней при условии осуществления перехвата пакетов в вычислительной сети, созданной на основе
			технологии Ethernet и протоколов стека протоколов TCP/IP:

			\begin{itemize}

				\item В случае сырого сокета:

					\begin{itemize}

						\item Заголовок пакета протокола канального уровня - заголовок Ethernet-кадра.

							Заголовок Ethernet-кадра может быть представлен с помощью структуры данных ethhdr,
							описанной в заголовочном файле <linux/if\_ether.h> и состоящей из следующих полей:

							\begin{itemize}

								\item h\_dest - MAC-адрес сетевого адаптера - получателя (6 байт);
								\item h\_source - MAC-адрес сетевого адаптера - отправителя (6 байт);
								\item h\_proto - идентификатор протокола, пакет которого вложен в Ethernet-кадр.

									Идентификатор протокола указывается в сетевом порядке байт. Идентификатор протокола
									IPv4, записанный в порядке байт хоста, равен значению константы ETH\_P\_IP;

							\end{itemize}

						\item Заголовок пакета протокола сетевого уровня - например, заголовок IP-пакета;
						\item Заголовок пакета протокола транспортного уровня или пакета протокола сетевого уровня,
						вложенного в IP-пакет - например, заголовок TCP-пакета, заголовок UDP-пакета или заголовок
						ICMP-пакета;

					\end{itemize}

				\item В случае дейтаграммного (SOCK\_DGRAM) сокета:

					\begin{itemize}

						\item Заголовок пакета протокола сетевого уровня - например, заголовок IP-пакета;
						\item Заголовок пакета протокола транспортного уровня или пакета протокола сетевого уровня,
						вложенного в IP-пакет - например, заголовок TCP-пакета, заголовок UDP-пакета или заголовок
						ICMP-пакета;

					\end{itemize}

			\end{itemize}

		\item Анализ содержимого перехваченного пакета;
		
		\item Перевод сетевого адаптера из неразборчивого режима работы в обычный режим работы.
		
			Перевод сетевого адаптера из неразборчивого режима работы в обычный режим работы выполняется
			аналогично обратному переводу, описанному в пункте \ref{lab4:sprom}, с точностью до порядка выполнения
			действий и используемой операции для снятия флага IFF\_PROMISC со значения поля ifr\_flags экземпляра
			структуры данных ifreq.

	\end{enumerate}

\subsubsection{Межсетевой экран netfilter}

	Межсетевой экран суть есть программный комплекс, выполняющий фильтрацию трафика, проходящего через сетевой узел.
	Межсетевые экраны позволяют решать широкий круг задач, к которым можно отнести следующие задачи:

	\begin{itemize}

		\item Отброс и перенаправление пакетов в соответствии с определенными критериями;
		\item Сбор данных о функционировании вычислительной сети;
		\item Сбор данных об активности сетевых узлов;
		\item Сбор и анализ данных о типе трафика, генерируемого и получаемого отдельными сетевыми узлами;
		\item Прочие задачи.

	\end{itemize}

	Межсетевые экраны представляют особый интерес для специалистов в области компьютерной безопасности, так как
	грамотно настроенный межсетевой экран, функционирующий на шлюзе между внешней сетью и внутренней защищаемой
	сетью, позволяет значительно снизить вероятность осуществления целого ряда атак на вычислительные системы,
	составляющие защищаемую сеть. К таковым атакам можно отнести атаки на получение удаленного несанкционированного
	доступа к данным, обмен которыми осуществляется в защищаемой вычислительной сети, атаки на отказ в обслуживании
	защищаемой вычислительной сети и вычислительных систем, составляющих защищаемую вычислительную сеть,
	и прочие атаки на защищаемую вычислительную сеть, могущие быть осуществленными удаленно.

	Межсетевые экраны можно классифицировать по ряду признаков:

	\begin{itemize}

		\item По охвату контролируемых потоков данных выделяют межсетевые экраны следующих типов:

			\begin{itemize}

				\item Традиционный межсетевой экран - межсетевой экран, контролирующих входящие и исходящие потоки
				данных между вычислительными сетями, связанными через межсетевой экран;
				\item Персональный межсетевой экран - межсетевой экран, предназначенный для защиты от
				несанкционированного доступа только той вычислительной системы, в которой данный межсетевой экран
				функционирует;

			\end{itemize}

		\item В зависимости от уровня сетевого взаимодействия (по сетевой модели OSI),
		на котором происходит контроль потока данных, выделяют межсетевые экраны следующих типов:

			\begin{itemize}

				\item Межсетевой экран, работающий на сетевом уровне;
				\item Межсетевой экран, работающий на сеансовом уровне;
				\item Межсетевой экран, работающий на прикладном уровне.

			\end{itemize}

	\end{itemize}

	В состав сетевой подсистемы ядра \linux\ входит межсетевой экран сетевого уровня netfilter, могущий работать как
	в качестве традиционного межсетевого экрана, так и в качестве персонального межсетевого экрана.

	Управление межсетевым экраном netfilter осуществляется с помощью консольной утилиты iptables. Право настраивать
	межсетевой экран netfilter в большинстве вычислительных систем, работающих под управлением \linux,
	имеет исключительно только суперпользователь.

	Межсетевой экран netfilter работает по следующему алгоритму:

	\begin{enumerate}

		\item Очередной пакет поступает к межсетевому экрану;

		\item Пакет последовательно проходит несколько таблиц правил межсетевого экрана в зависимости
		от направления передачи пакета;
		
		\item В каждой из таблиц правил пакет последовательно проходит несколько цепочек правил,
		в каждой из которых он, в конечном счете, либо принимается (действие ACCEPT), либо отбрасывается
		(действие DROP - без извещения отправителя пакета; действие RETURN - с извещением отправителя пакета).

		Одно из действий: ACCEPT, DROP или RETURN - должно быть назначено в качестве действия по умолчанию
		для каждой цепочки правил - действие по умолчанию применяется ко всех пакетам,
		к которым не было явно применено одно из трех перечисленных действий.

		Отброшенный пакет удаляется из сетевой подсистемы ОС, принятый пакет передается очередной цепочке правил;

		\item В каждой из цепочек правил к пакету последовательно применяются правила, составляющие цепочку,
		до тех пор, пока пакет не будет либо принят, либо отброшен.

		Каждое правило состоит из двух частей: условие и действие. Действие применяется к пакетам,
		для которых условие принимает истинное значение;

		\item В том случае, если пакет был принят во всех цепочках правил, пройденных им, он передается сетевой
		подсистеме ОС для дальнейшей обработки.

	\end{enumerate}

