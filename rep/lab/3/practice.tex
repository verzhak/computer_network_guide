
\subsubsection{Предварительная подготовка}

Для выполнения данной лабораторной работы должна быть использована вычислительная сеть, созданная и настроенная в ходе выполнения лабораторной работы № 1.
Структурная схема используемой вычислительной сети приведена на рисунке \ref{image:struct3}.

\mimage{struct3}{1/struct}{Структурная схема вычислительной сети}{width=\textwidth}

\subsubsection{Сетевое программирование с использованием протоколов ICMP и ARP. Утилиты ping и arp. Файл /proc/net/arp}
\label{task:l3t1}

	В рамках данной лабораторной работы каждому студенту группы необходимо выполнить на центральном сетевом узле
	следующие задания:

	\begin{enumerate}

		\item Проверить доступность каждого из периферийных сетевых узлов с помощью N циклов обмена с периферийными
		сетевыми узлами пакетами типа ECHO и ECHO-REPLY протокола ICMP;

		\item Получить содержимое ARP-кэша с помощью файла /proc/net/arp;

		\item Удалить из ARP-кэша записи об обоих периферийных сетевых узлах;

		\item Проверить корректность удаления записей из ARP-кэша с помощью утилиты arp;

		\item Разработать программу, выполняющую ARP-запрос о периферийном сетевом узле \linebreak \first;\label{lab3:pp-arp}

		\item Разработать программу, выполняющую обмен пакетами типов ECHO и ECHO-REPLY протокола ICMP с периферийным
		узлом \second.\label{lab3:pp-icmp}
		
		Программа должна выполнять последовательно N циклов обмена пакетами означенного типа с
		периферийным узлом \second\ и для каждого принятого или отправленного пакета программа должна выводить
		содержимое заголовка пакета;

		\item Запустить на выполнение программы, разработанные в пунктах \ref{lab3:pp-arp} и \ref{lab3:pp-icmp};

		\item По завершению выполнения программ, разработанных в пунктах \ref{lab3:pp-arp} и \ref{lab3:pp-icmp},
		получить содержимое ARP-кэша с помощью файла /proc/net/arp;

		\item Получить содержимое ARP-кэша с помощью утилиты arp;

		\item Сделать выводы о принципах функционирования ARP-кэша.

	\end{enumerate}

	Здесь N суть есть: $N = (M~mod~5 + 1) * 4$, где M - порядковый номер студента в журнале группы.

\subsubsection{Подготовка отчета по лабораторной работе}

	Подготовить отчет, в котором необходимо привести подробное описание процесса выполнения заданий пункта
	\ref{task:l3t1}, проиллюстрированное достаточным количеством снимков экрана.

	В отчете по лабораторной работе должен быть приведен исходный код разработанного программного обеспечения с
	соответствующими комментариями.

