
\subsubsection{Протоколы сетевого уровня стека протоколов TCP/IP}

	Сетевой уровень модели OSI предназначен для образования единой транспортной системы, объединяющей несколько вычислительных сетей.

	К протоколам сетевого уровня стека протоколов TCP/IP относятся следующие протоколы:

	\begin{itemize}

		\item IP - протокол негарантированной доставки данных (в данной лабораторной работе используется версия IPv4 данного протокола);
		\item ICMP - сервисный протокол, предназначенный для выполнения различных служебных задач обеспечения функционирования сети, в частности,
		для передачи сообщений об ошибках;
		\item ARP - протокол, предназначенный для определения по известному адресу сетевого уровня адреса канального уровня (в случае сетей, построенных
		на базе технологии Ethernet и стеке протоколов TCP/IP, адресом канального уровня является MAC-адрес сетевого адаптера, адресом сетевого уровня -
		IP-адрес сетевого узла);
		\item RARP - протокол, предназначенный для определения по известному адресу канального уровня адреса сетевого уровня;
		\item IGMP - протокол управления групповой (multicast) передачей данных в сетях, основанных на протоколе IP;
		\item Прочие протоколы.

	\end{itemize}

	\linux\ предоставляет удобные программные средства работы с протоколами сетевого уровня стека протоколов TCP/IP.

	Использование протоколов сетевого уровня стека протоколов TCP/IP процессами системы для решения задач сетевого взаимодействия состоит в использовании
	сокетов соответствующих типов. Работа процессов ОС с протоколами сетевого уровня стека протоколов TCP/IP, в общем случае,
	подразумевает следущую последовательность действий:

	\begin{enumerate}

		\item Создание сокета с помощью системного вызова socket.

			В качестве идентификатора семейства протоколов данному системному вызову необходимо передать значение одной из следующих констант:

			\begin{itemize}

				\item AF\_INET - для работы с протоколами стека протоколов TCP/IPv4, пакеты которых при пересылке вкладываются в пакеты протокола IPv4;
				\item AF\_INET6 - для работы с протоколами стека протоколов TCP/IPv6, пакеты которых при пересылке вкладываются в пакеты протокола IPv6;
				\item AF\_PACKET - для работы с протоколами канального и сетевого уровней различных семейств протоколов.

					AF\_PACKET-сокеты используются для работы с протоколами сетевого уровня стека протоколов TCP/IP,
					пакеты которых не инкапсулируются в пакеты протоколов IPv4 и IPv6.

			\end{itemize}

			В качестве идентификатора режима передачи пакетов системному вызову socket необходимо передать значение одной из следующих констант:

			\begin{itemize}

				\item SOCK\_RAW - в случае, если в качестве идентификатора семейства протоколов системному вызову socket было передано значение констант
				AF\_INET или AF\_INET6.

					Использование сырых пакетных сокетов (сокетов, при создании которых системному вызову socket были переданы значения констант AF\_PACKET
					и SOCK\_RAW) предполагает работу с протоколами канального уровня;

				\item SOCK\_DGRAM - в случае, если в качестве идентификатора семейства протоколов системному вызову socket было передано значение константы
				AF\_PACKET.

			\end{itemize}

			Использовать пакетные (AF\_PACKET) и (или) сырые (SOCK\_RAW) сокеты имеют право только те процессы, которые обладают характеристикой CAP\_NET\_RAW.
			Характеристика CAP\_NET\_RAW по умолчанию устанавливается на процессы, имеющие эффективным владельцем суперпользователя.
			К процессам, имеющим эффективным владельцем суперпользователя, относятся процессы, запущенные суперпользователем;

		\item Отправление / получение данных с помощью системных вызовов sendto и recv соответственно.

			При отправлении и получении пакетов протоколов сетевого уровня необходимо помнить, что заголовки означенных пакетов
			считываются ОС из пользовательского буфера при отправлении пакетов и помещаются ОС в пользовательский буфер при получении пакетов. Это означает,
			что, в случае отправления пакета процессом ОС, заголовок пакета должен быть сформирован процессом и помещен в начало буфера,
			указатель на который передается в системный вызов sendto. При получении пакета операционная система записывает в пользовательский
			буфер пакет целиком - заголовок пакета в начало буфера и поле данных - следом за заголовком.
			Для корректого формирования и анализа заголовков пакетов необходимо использовать указатели на экземпляры следующих структур данных:

			\begin{itemize}

				\item Протокол IP - структура данных iphdr, описанная в заголовочном файле \linebreak <netinet/ip.h>;
				\item Протокол ICMP - структура данных icmphdr, описанная в заголовочном файле <netinet/ip\_icmp.h>;
				\item Протоколы ARP и RARP - структура данных arphdr, описанная в заголовочном файле <net/if\_arp.h>;
				\item Протокол IGMP - структура данных igmp, описанная в заголовочном файле \linebreak <netinet/igmp.h>.

			\end{itemize}

			Корректное формирование и анализ содержимого пакетов протоколов сетевого уровня состоит, таким образом, в выполнении следующих действий:

			\begin{enumerate}

				\item Описание переменной типа <<указатель на структуру данных, описывающую заголовок пакета>>;
				\item Присвоение указателю адреса первого байта буфера;
				\item Формирование / анализ содержимого заголовка пакета с помощью указателя на соответствующую структуру данных;
				\item Запись / считывание содержимого поля данных пакета, расположенного в буфере по смещению, равному размеру в байтах заголовка пакета.
				Размер в байтах заголовка пакета, в свою очередь, равен размеру в байтах экземпляра соответствующей структуры данных.

			\end{enumerate}

			Необходимо также помнить, что системному вызову sendto передается, кроме всего прочего, описатель адреса сетевого узла - получателя пакета.
			Данный описатель суть есть указатель на экземпляр одной из следующих структур данных, явно приведенный к указателю на экземпляр структуры данных
			sockaddr:

			\begin{itemize}

				\item sockaddr\_in - в случае использования протокола, пакеты которого инкапсулируются в пакеты протокола IPv4;
				\item sockaddr\_ll - в случае использования протокола, пакеты которого не инкапсулируются в пакеты протоколов IPv4 или IPv6.
				
				Структура данных sockaddr\_ll описана в заголовочном файле <netpacket/ packet.h>; 

			\end{itemize}

		\item Уничтожение сокета с помощью системного вызова close.
	
	\end{enumerate}

\subsubsection{Протокол IP}
	
	Протокол IPv4 (в дальнейшем просто протокол IP) - протокол негарантированной доставки данных, использующий формат IPv4 адресации сетевых узлов.

	Протокол IP обеспечивает доставку пользовательских данных на сетевом уровне взаимодействия узлов сети. Пакеты протоколов транспортного уровня
	стека протоколов TCP/IP и некоторые протоколы сетевого уровня стека протоколов TCP/IP вкладываются (инкапсулируются) в пакеты протокола IP.

	Формат пакета протокола IP приведен на рисунке \ref{image:lab3-ip-struct}.

	\vbox
	{
		\begin{center}
			
			\refstepcounter{figure}
			\label{image:lab3-ip-struct}

			\begin{bytefield}[bitwidth=12pt]{32}
				\bitheader{0,3,4,7,8,15,16,31}\\
				\begin{rightwordgroup}{\rotatebox{90}{Заголовок пакета}}
				\bitbox{4}{Version}\bitbox{4}{IHL}\bitbox{8}{ToS}\bitbox{16}{Total}\\
				\bitbox{16}{ID}\bitbox{1}{\rotatebox{90}{\small RF}}\bitbox{1}{\rotatebox{90}{\small DF}}\bitbox{1}{\rotatebox{90}{\small MF}}
				\bitbox{13}{Offset}\\
				\bitbox{8}{TTL}\bitbox{8}{Protocol}\bitbox{16}{Checksum}\\
				\bitbox{32}{Source}\\
				\bitbox{32}{Destination}\\
				\bitbox{32}{Options}\\
				\bitbox{32}{Padding}
				\end{rightwordgroup}\\
				\wordbox{1}{Data}
			\end{bytefield}

			{\noindent Рисунок~\thefigure~---~Формат пакета протокола IP}

		\end{center}
	}

	Заголовок пакета протокола IP состоит из следующих полей:

	\begin{itemize}

		\item <<Version>> - номер версии протокола IP. Для протокола IPv4 значение данного поля устанавливается в 4;
		\item <<IHL>> - размер в двойных словах (32 бита) заголовка пакета (IP Header Length).
		Обычно, значение данного поля равно 5;
		\item <<ToS>> - тип сервиса (Type of Service; ToS; байт дифференцированного обслуживания; DS-байт). Значение данного поля отражает требования отправителя
		к качеству обслуживания пакета;
		\item <<Total>> - размер пакета в байтах;
		\item <<ID>> - идентификатор пакета, используемый для идентификации исходного блока данных, фрагментированного для передачи на несколько IP-пакетов.
		Фрагменты одного и того же блока данных имеют один и тот же идентификатор;
		\item <<RF>> - зарезервированный бит;
		\item <<DF>> - флаг запрета фрагментации пакета (Do not Fragment);
		\item <<MF>> - флаг, установленный в 1 в том случае, если данный пакет не является последним в цепочке пакетов, содержащих фрагменты исходного
		блока данных (More Fragments);
		\item <<Offset>> - смещение в байтах поля <<Data>> (поля данных) пакета от начала заголовка пакета. Значение данного поля должно быть кратно 8;
		\item <<TTL>> - время жизни пакета (Time To Live; TTL) в переходах между маршрутизаторами (в хопах).
		
		Каждый маршрутизатор, пересылающий пакет, вычитает из TTL пакета единицу. В случае, если значение TTL пакета достигло нуля,
		то пакет отбрасывается очередным маршрутизатором;

		\item <<Protocol>> - идентификатор протокола, пакет которого вложен в пакет протокола IP.

		Идентификатор протокола ICMP равен 1, протокола IGMP - 2, протокола TCP - 6, протокола UDP - 17, протокола SCTP - 132;

		\item <<Checksum>> - контрольная сумма заголовка пакета;
		\item <<Source>> - IP-адрес сетевого узла - источника пакета;
		\item <<Destination>> - IP-адрес сетевого узла - получателя пакета;
		\item <<Options>> - произвольное число параметров, используемых при отладке вычислительной сети;
		\item <<Padding>> - заполнение заголовка пакета нулями до размера в байтах, кратного 32-м.

	\end{itemize}

	В поле <<Data>> (поле данных) IP-пакета помещаются данные, пересылаемые с помощью пакета протокола IP. В частности, в поле <<Data>> могут быть помещены
	пакеты протоколов ICMP, TCP, UDP или SCTP.

	При написании программ, работающих непосредственно с пакетами протокола IP, программист может
	воспользоваться структурой данных iphdr, определенной в заголовочном файле <netinet/ip.h> и удобным образом
	описывающей заголовок IP-пакета. Имеют место быть следующие соответствия полей структуры данных iphdr
	полям заголовка пакета протокола IP;

	\begin{itemize}

		\item Поле version - поле <<Version>>;
		\item Поле ihl - поле <<IHL>>;
		\item Поле tos - поле <<ToS>>;
		\item Поле tot\_len - поле <<Total>>;
		\item Поле id - поле <<ID>>;
		\item Поле frag\_off - поля <<RF>>, <<DF>>, <<MF>> и <<Offset>>.

			Для выделения из значения поля frag\_off значений составляющих его полей заголовка пакета протокола IP
			необходимо использовать следующие, определенные в заголовочном файле <netinet/ip.h>, маски:

				\begin{itemize}

					\item IP\_RF - поле <<RF>>;
					\item IP\_DF - поле <<DF>>;
					\item IP\_MF - поле <<MF>>;
					\item IP\_OFFMASK - поле <<Offset>>;

				\end{itemize}

		\item Поле ttl - поле <<TTL>;
		\item Поле protocol - поле <<Protocol>>;
		\item Поле check - поле <<Checksum>>;
		\item Поле saddr - поле <<Source>>;
		\item Поле daddr - поле <<Destination>>.

	\end{itemize}

\subsubsection{Протокол ICMP. Утилита ping}

	Протокол ICMP суть есть сервисный протокол, предназначенный для выполнения различных служебных задач обеспечения функционирования сети, в частности,
	для передачи сообщений об ошибках. При передаче пакеты протокола ICMP вкладываются в пакеты протокола IP.

	Формат пакета протокола ICMP приведен на рисунке \ref{image:lab3-icmp-struct}.

	\vbox
	{
		\begin{center}
			
			\refstepcounter{figure}
			\label{image:lab3-icmp-struct}

			\begin{bytefield}{32}
				\bitheader{0,7,8,15,16,31}\\
				\bitbox{8}{Type}\bitbox{8}{Code}\bitbox{16}{Checksum}\\
				\wordbox{1}{Data}
			\end{bytefield}

			{\noindent Рисунок~\thefigure~---~Формат пакета протокола ICMP}

		\end{center}
	}

	Пакет протокола ICMP состоит из следующих полей:

		\begin{itemize}
		
			\item Поле <<Type>> содержит идентификатор типа пакета и может принимать следующие значения:

				\begin{itemize}

					\item 0 — эхо-ответ (ECHO-REPLY);
					\item 4 — сдерживание источника (отключение источника при переполнении очереди);
					\item 5 — перенаправление;
					\item 8 — эхо-запрос (ECHO);
					\item 11 — превышение временного интервала (для дейтаграммы время жизни истекло);
					\item 12 — неверный параметр (проблема с параметрами дейтаграммы: ошибка в IP-заголовке или отсутствует необходимая опция);
					\item 30 — трассировка маршрута;
					\item Прочие значения из диапазона $[0, 255]$;

				\end{itemize}

			\item Поле <<Code>> заполняется для пакетов определенных типов и содержит, фактически, уточнение типа пакета;
			\item Поле <<Checksum>> содержит контрольную сумму пакета;
			\item Поле <<Data>> содержит дополнительные характеристики ICMP-пакета и данные, пересылаемые в ICMP-пакете.
			Формат данного поля зависит от типа пакета.

		\end{itemize}

	В данной лабораторной работе исследуется возможность применения пакетов типа ECHO и ECHO-REPLY для проверки доступности
	удаленного узла вычислительной сети.

	Алгоритм проверки доступности удаленного узла вычислительной сети с помощью пакетов означенных типов состоит из следующих шагов:

	\begin{enumerate}

		\item Проверяющий сетевой узел отправляет проверяемому сетевому узлу ECHO-пакет;
		\item Проверяющий сетевой узел ожидает получения ответного ECHO-REPLY-пакета от проверяемого сетевого узла.

		Если таковой пакет получен, то проверяемый сетевой узел доступен проверяющему сетевому узлу. Если ответный ECHO-REPLY-пакет не получен проверяющим
		сетевым узлом за определенный промежуток времени, то проверяемый сетевой узел не доступен проверяющему сетевому узлу
		или доступен, но с неудовлетворительными временными характеристиками сетевого обмена.

	\end{enumerate}

	Для проверки доступности удаленного сетевого узла с помощью пакетов типа ECHO и ECHO-REPLY оператор может воспользоваться утилитой ping.

	На рисунке \ref{image:lab3-1} приведен процесс проверки доступности сетевого узла с IP-адресом, равным 127.0.0.1 (что соответствует обратной петле -
	то есть узлу, на котором запущена утилита ping), выполненной с помощью утилиты ping. Без дополнительных ключей утилита ping работает до тех пор,
	пока не будет прервана оператором (на рисунке \ref{image:lab3-1} выполнение ping прервано отправлением ей сигнала SIGINT путем нажатия
	сочетания клавиш <<Ctrl + C>>).

	TTL пакетов на рисунке \ref{image:lab3-1} равен 64, что означает, что маршрут до целевого узла сети может содержать не более 64-х узлов, считая целевой узел.
			
	\mimage{lab3-1}{3/1}{Проверка доступности сетевого узла 127.0.0.1 с помощью утилиты ping (остановлена оператором)}{}

	Для предписания самостоятельного завершения после отправления определенного оператором количества пакетов утилите ping необходимо передать
	ключ <<-c COUNT>>, где COUNT - количество пакетов, после отправления которых утилита ping должна самостоятельно завершить свою работу.

	\mimage{lab3-2}{3/2}{Проверка доступности сетевого узла 127.0.0.1 с помощью утилиты ping (самостоятельно завершила работу после отправления 10-ти пакетов)}{}

	Размер блоков данных в пакетах, отправляемых утилитой ping, можно указать с помощью ключа <<-s SIZE>>, где SIZE - размер в байтах блока данных ECHO-пакета.

	\mimage{lab3-3}{3/3}{Проверка доступности сетевого узла 127.0.0.1 с помощью утилиты ping (размер блоков данных в отправляемых пакетах равен 777-и байтам)}{}

	Для изменения времени ожидания утилитой ping ответных пакетов от целевого узла сети необходимо воспользоваться ключом <<-W TIMEOUT>>, где TIMEOUT -
	максимальное время ожидания ответного пакета в секундах.

	На рисунке \ref{image:lab3-6} приведен результат выполнения утилиты ping к сетевому узлу с IP-адресом равным 192.168.91.128.

	\mimage{lab3-6}{3/6}{Проверка доступности сетевого узла 192.168.91.128 с помощью утилиты ping}{}

	Процессы ОС, использующие пакеты типов ECHO и ECHO-REPLY протокола ICMP для проверки доступности удаленных сетевых узлов,
	должны следовать следующим замечаниям:

	\begin{enumerate}

		\item Системный вызов socket, используемый для создания сокета, должен быть вызван со следующими параметрами:

			\begin{itemize}

				\item AF\_INET - идентификатор семейства протоколов (стек протоколов TCP/ IPv4);
				\item AF\_RAW - идентификатор режима передачи пакетов (сырой сокет);
				\item IPPROTO\_ICMP - идентификатор протокола сетевого взаимодействия (протокол ICMP);

			\end{itemize}

		\item Для отправления и получения пакетов необходимо использовать системные вызовы sendto и recv.
		
		При получении процессом пакета протокола ICMP сетевая подсистема ОС записывает в результирующий буфер содержимое заголовка пакета протокола IP перед
		заголовком пакета протокола ICMP - таким образом, заголовок пакета протокола ICMP находится в результирующем буфере по смещению, равному размеру в
		байтах заголовка пакета протокола IP. Размер в байтах заголовка пакета протокола IP можно считать, с некоторым допущением, равным размеру в байтах
		экземпляра структуры данных iphdr;

		\item Идентификаторы типов пакетов ECHO и ECHO-REPLY равны значениям констант ICMP\_ECHO и ICMP\_ECHOREPLY, определенным в заголовочном файле
		\linebreak <netinet/ip\_icmp.h>;
		\item Значения поля <<Code>> пакетов типов ECHO и ECHO-REPLY устанавливаются в ноль;
		\item Формат поля данных пакетов имеет вид, приведенный на рисунке \ref{image:lab3-icmp-data-struct}.

			\vbox
			{
				\begin{center}
					
					\refstepcounter{figure}
					\label{image:lab3-icmp-data-struct}

					\begin{bytefield}{32}
						\bitheader{0,15,16,31}\\
						\bitbox{16}{Identifier}\bitbox{16}{Sequence number}
					\end{bytefield}

					{\noindent Рисунок~\thefigure~---~Формат поля данных пакетов типа ECHO и ECHO-REPLY протокола ICMP}

				\end{center}
			}

			Здесь поле <<Identifier>> содержит 16-ти битовое беззнаковое целое число - идентификатор последовательности пакетов.
			Данное число идентифицирует процесс проверки доступности удаленного сетевого узла,
			состоящий из нескольких циклов обмена пакетами типов ECHO и ECHO-REPLY.

			Поле <<Sequence number>> содержит порядковый номер пакета в последовательности пакетов, составляющих процесс
			проверки доступности удаленного сетевого узла, считая с нуля.

			Поля <<Identifier>> и <<Sequence number>> записываются в сетевом порядке байт;

		\item Контрольная сумма пакета протокола ICMP вычисляется как дополненная до единицы сумма слов, составляющих содержимое пакета.
		При вычислении контрольной суммы пакета значение поля <<Checksum>> должно быть установлено в ноль.

		В листинге \ref{listing:lab3-checksum} приведен код функции checksum(), вычисляющей контрольную сумму содержимого буфера,
		указатель на который передается функции	в параметре buf, а размер в байтах которого - в параметре buf\_size. Функция checksum() возвращает
		16-ти битовое беззнаковое целое число - контрольную сумму содержимого буфера, указатель на который передан функции в параметре buf.

		\vbox
		{
			\medskip
			\begin{center}
					
				\refstepcounter{figure}
				\label{listing:lab3-checksum}
				\begin{lstlisting}

#include <stdint.h>

uint16_t checksum(uint16_t *buf, uint16_t buf_size)
{
	unsigned u;
	uint32_t sum = 0;
	buf_size /= 2;

	for(u = 0; u < buf_size; u++)
		sum += buf[u];

	return ~ ((sum & 0xFFFF) + (sum >> 16));
}

				\end{lstlisting}

				{\noindent Листинг~\thefigure~---~Функция checksum(), вычисляющая контрольную сумму содержимого целевого буфера}

			\end{center}
		}

		\item Структура данных icmphdr, описывающая пакет протокола ICMP, состоит из следующих полей:

			\begin{itemize}
					
				\item type - идентификатор типа пакета;
				\item code - код пакета (для пакетов типа ECHO и ECHO\_REPLY данное поле не используется и установлено в ноль);
				\item checksum - контрольная сумма пакета;
				\item un.echo.id - идентификатор последовательности пакетов;
				\item un.echo.sequence - номер пакета в последовательности.

			\end{itemize}

	\end{enumerate}

\subsubsection{Протокол ARP. Получение и редактирование содержимого кэша ARP}

	Протокол ARP предназначен для определения по известному адресу сетевого уровня соответствующего адреса канального уровня (в случае сетей, построенных
	на базе технологии Ethernet и стеке протоколов TCP/IP, адресом канального уровня является MAC-адрес сетевого адаптера, адресом сетевого уровня
	- IP-адрес сетевого узла).

	Формат пакета протокола ARP приведен на рисунке \ref{image:lab3-arp-struct}.

	\vbox
	{
		\begin{center}
			
			\refstepcounter{figure}
			\label{image:lab3-arp-struct}

			\begin{bytefield}{32}
				\bitheader{0,7,8,15,16,31}\\
				\bitbox{16}{Hardware type}\bitbox{16}{Protocol type}\\
				\bitbox{8}{Hardware len.}
				\bitbox{8}{Protocol len.}
				\bitbox{16}{Operation}\\
				\begin{rightwordgroup}{Размер полей зависит \\ от значений полей \\ <<Hardware length>> \\ и <<Protocol length>>}
				\bitbox{32}{SHA}\\
				\bitbox{32}{SPA}\\
				\bitbox{32}{THA}\\
				\bitbox{32}{TPA}
				\end{rightwordgroup}
			\end{bytefield}

			{\noindent Рисунок~\thefigure~---~Формат пакета протокола ARP}

		\end{center}
	}

	Пакет протокола ARP состоит из следующих полей:

	\begin{itemize}
			
		\item Поле <<Hardware type> - идентификатор протокола канального уровня;
		\item Поле <<Protocol type>> - идентификатор протокола сетевого уровня;
		\item Поле <<Hardware length>> - размер в байтах сетевого адреса, используемого в протоколе канального уровня;
		\item Поле <<Protocol length>> - размер в байтах сетевого адреса, используемого в протоколе сетевого уровня;
		\item Поле <<Operation>> - тип операции;
		\item Поле <<SHA>> - адрес узла отправителя, используемый в протоколе канального уровня;
		\item Поле <<SPA>> - адрес узла отправителя, используемый в протоколе сетевого уровня;
		\item Поле <<THA>> - адрес узла получателя, используемый в протоколе канального уровня;
		\item Поле <<TPA>> - адрес узла получателя, используемый в протоколе сетевого уровня.

	\end{itemize}

	Размеры полей <<SHA>>, <<SPA>>, <<THA>> и <<TPA>> зависят от протоколов, соответствие для которых устанавливается. 

	Сетевой обмен с использованием протокола ARP может проходить по следующему сценарию:

	\begin{enumerate}

		\item Сетевой узел - отправитель формирует ARP-запрос на установление соответствия MAC-адреса некоторому IP-адресу.

		ARP-запрос представляет из себя ARP-пакет со следующими значениями полей:

			\begin{itemize}
			
				\item Поле <<Hardware type> - 0x0001, что соответствует технологии Ethernet;
				\item Поле <<Protocol type>> - 0x0800, что соответствует протоколу IPv4;
				\item Поле <<Hardware length>> - 6;
				\item Поле <<Protocol length>> - 4;
				\item Поле <<Operation>> - 1;
				\item Поле <<SHA>> - MAC-адрес сетевого интерфейса, с которого сетевой узел - отправитель отправит пакет (размер поля - 6 байт);
				\item Поле <<SPA>> - IPv4-адрес сетевого узла - отправителя (размер поля - 4 байта);
				\item Поле <<THA>> - 0 (размер поля - 6 байт);
				\item Поле <<TPA>> - IPv4-адрес сетевого узла - получателя;

			\end{itemize}

		\item Сетевой узел - отправитель посылает ARP-запрос на широковещательный MAC-адрес (FF:FF:FF:FF:FF:FF);

		\item Удаленный сетевой узел, чей IP-адрес равен IP-адресу, помещенному в ARP-запрос, формирует ARP-ответ и отправляет ARP-ответ
		сетевому узлу - отправителю.

		Заполнение полей пакета ARP-ответа на ARP-запрос совпадает с заполнением полей пакета ARP-запроса со следующими исключениями:

			\begin{itemize}

				\item Поле <<Operation>> - 2;
				\item Поле <<THA>> - MAC-адрес сетевого интерфейса, с которого удаленный сетевой узел отправит ответный пакет (размер поля - 6 байт);

			\end{itemize}

		\item Сетевой узел - отправитель помещает запись о соответствии MAC-адреса IP-адресу в свой кэш ARP и в дальнейшем использует
		данную запись в тех случаях, когда возникает необходимость поставить в соответствие данному IP-адресу некоторый MAC-адрес
		(такая необходимость возникает, например, при выполнении вложения IP-пакета в Ethernet-кадр).

	\end{enumerate}

	Означенный кэш ARP сетевого узла суть есть важный элемент сетевого обмена, так как с помощью данного кэша сетевой узел ставит в соответствие
	IP-адресам (адресам сетевого уровня) MAC-адреса (адреса канального уровня). В случае гипотетической некорректной реализации кэша ARP вычислительная система
	не смогла бы в принципе осуществлять сетевой обмен по протоколам стека протоколов TCP/IP, так как связь между протоколами канального и сетевого уровней
	была бы нарушена.

	Записи о соответствиях IP-адресов MAC-адресам в кэше ARP бывают двух видов:

	\begin{itemize}

		\item Статические записи - добавляются оператором сетевого узла вручную (с помощью утилиты arp) и не выталкиваются из кэша;
		\item Динамические записи - могут быть добавлены как вручную оператором, так и динамически ОС и могут быть вытолкнуты из кэша
		по прошествии определенного промежутка времени (устаревание записи) или при переполнении кэша.

	\end{itemize}

	Получить содержимое кэша ARP можно следующими способами:

	\begin{enumerate}

		\item С помощью утилиты arp, для чего необходимо вызвать ее без аргументов, что проиллюстрировано рисунком \ref{image:lab3-7};
		\item Прочитав файл /proc/net/arp, что проиллюстрировано рисунком \ref{image:lab3-8}.

	\end{enumerate}

	\mimage{lab3-7}{3/7}{Вывод содержимого кэша ARP с помощью утилиты arp}{width=\textwidth}
	\mimage{lab3-8}{3/8}{Вывод содержимого кэша ARP посредством чтения файла /proc/net/arp}{width=\textwidth}

	Для удаления записи из кэша ARP необходимо вызвать утилиту arp с ключом <<-d IP>>, где IP - IP-адрес узла, запись о котором необходимо
	удалить из кэша ARP. Удаление записи из кэша ARP требует прав суперпользователя.
	Для добавления динамической записи в кэш ARP необходимо обратится к целевому узлу (например, с помощью утилиты ping).

	На рисунке \ref{image:lab3-9} приведено состояние кэша ARP до удаления записи об узле 192.168. 91.128.
	На рисунке \ref{image:lab3-10} приведены процесс удаления записи об узле 192.168.91.128 из кэша ARP и состояние кэша ARP после удаления записи
	об узле 192.168.91.128. На рисунке \ref{image:lab3-11} приведен процесс обращения к узлу 192.168.91.128 с помощью утилиты ping.
	И, наконец, на рисунке \ref{image:lab3-12} приведено состояние кэша ARP после добавления в него информации об узле 192.168.91.128 по результатам обращения
	к нему с помощью утилиты ping. Как видно из приведенных рисунков, состояние кэша ARP до удаления/добавления записи об узле 192.168.91.128
	идентично состоянию кэша ARP после удаления/добавления записи об узле 192.168.91.128.

	\mimage{lab3-9}{3/9}{Состояние кэша ARP до удаления записи об узле 192.168.91.128}{width=\textwidth}
	\mimage{lab3-10}{3/10}{Удаление записи об узле 192.168.91.128 из кэша ARP и состояние кэша после удаления записи из него}{width=\textwidth}
	\mimage{lab3-11}{3/11}{Запрос к узлу 192.168.91.128 с помощью утилиты ping}{}
	\mimage{lab3-12}{3/12}{Состояние кэша ARP после добавления в него записи об узле 192.168.91.128}{width=\textwidth}

	При разработке программного обеспечения на языке программирования C для \linux, использующего протокол ARP для определения соответствий IPv4-адресам
	MAC-адресов, необходимо учитывать следующие замечания:

	\begin{enumerate}

		\item Системный вызов socket, используемый для создания сокета, вызывается со следующими параметрами:

			\begin{itemize}

				\item AF\_PACKET - идентификатор семейства протоколов (протоколы канального уровня и протоколы сетевого уровня стека протоколов TCP/IP,
				пакеты которых не вкладываются в пакеты протокола IP);
				\item AF\_DGRAM - идентификатор режима передачи пакетов (негарантированный режим передачи);
				\item ETH\_P\_ARP - идентификатор протокола сетевого взаимодействия (протокол ARP; константа определена в заголовочном файле
				<linux/if\_ether.h>).

				Данный параметр необходимо передавать в системный вызов socket в сетевом порядке байт, тогда как значение константы ETH\_P\_ARP записано в порядке
				байт хоста;

			\end{itemize}

		\item Широковещательный MAC-адрес передается системному вызову sendto с помощью структуры данных sockaddr\_ll, содержащей следующие поля:

			\begin{itemize}
				
				\item sll\_family - идентификатор семейства протоколов (AF\_PACKET);
				\item sll\_protocol - идентификатор протокола сетевого взаимодействия\\(ETH\_P\_ARP), записанный в сетевом порядке байт;
				\item sll\_ifindex - идентификатор сетевого интерфейса, через который необходимо отправить ARP-запрос;
				\item sll\_hatype - (в случае протокола ARP) идентификатор типа операции (1 для ARP-запроса);
				\item sll\_pkttype - идентификатор типа сетевого адреса (PACKET\_BROADCAST для широковещательного адреса);
				\item sll\_halen - размер в байтах адреса (6);
				\item sll\_addr - адрес (размер поля - 8 байт, в случае MAC-адреса используются первые 6 байт; все байты данного поля должны быть установлены
				в 0xFF);

			\end{itemize}

		\item Для получения идентификатора сетевого интерфейса по его имении необходимо воспользоваться системным вызовом ioctl.

		Возможное применение означенного системного вызова для получения идентификатора сетевого интерфейса по его имени
		проиллюстрировано функцией get\_iface\_ id(), исходный код которой приведен в листинге \ref{listing:lab3-get-iface-id}.
		Функция get\_iface\_id() принимает следующие параметры:
		
			\begin{itemize}
			
				\item sock - номер дескриптора сокета, открытого для обмена по протоколу ARP;
				\item device\_name - имя целевого сетевого интерфейса.
				
			\end{itemize}
			
		Функция get\_iface\_id() возвращает идентификатор целевого сетевого интерфейса или -1 в случае ошибки.

		\vbox
		{
			\medskip
			\begin{center}
					
				\refstepcounter{figure}
				\label{listing:lab3-get-iface-id}
				\begin{lstlisting}

#include <string.h>
#include <sys/ioctl.h>
#include <net/if.h>

int get_iface_id(int sock, const char* device_name)
{
	struct ifreq req;
	strcpy(req.ifr_name, device_name);

	if(ioctl(sock, SIOCGIFINDEX, & req))
		return -1;
	
	return req.ifr_ifindex;
}

				\end{lstlisting}

				{\noindent Листинг~\thefigure~---~Функция get\_iface\_id(), определяющая идентификатор целевого сетевого интерфейса}

			\end{center}
		}

		\item Структура данных arphdr, описывающая пакет протокола ARP, состоит из следующих полей:

			\begin{itemize}
					
				\item ar\_hrd - идентификатор протокола канального уровня - 0x0001, что соответствует технологии Ethernet;
				\item ar\_pro - идентификатор протокола сетевого уровня - 0x0800, что соответствует протоколу IPv4;
				\item ar\_hln - размер в байтах сетевого адреса, используемого в протоколе канального уровня - 6;
				\item ar\_pln - размер в байтах сетевого адреса, используемого в протоколе сетевого уровня - 4;
				\item ar\_op - тип операции - 1 (ARP-запрос) или 2 (ARP-ответ).

			\end{itemize}

			Поля ar\_hdr, ar\_pro и ar\_op должны быть записаны в сетевом порядке байт;

		\item Поля пакета протокола ARP, содержащие адреса сетевого узла - отправителя и сетевого узла - получателя, указываются в буфере, содержащем
		означенный пакет, непосредственно после заголовка пакета.

	\end{enumerate}

