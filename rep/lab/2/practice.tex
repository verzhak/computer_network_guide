
\subsubsection{Предварительная подготовка}

Для выполнения данной лабораторной работы должна быть использована вычислительная сеть, созданная и настроенная в ходе выполнения лабораторной работы № 1.
Структурная схема используемой вычислительной сети приведена на рисунке \ref{image:struct2}.

\mimage{struct2}{1/struct}{Структурная схема вычислительной сети}{width=\textwidth}

\subsubsection{Утилиты netstat и ncat. Файлы /etc/protocols и /etc/hosts}
\label{task:l2t1}

	Необходимо выполнить следующие задания:

	\begin{enumerate}

		\item Запустить TCP-сервер на периферийном сетевом узле \first;
		\item Запустить TCP-сервер на центральном сетевом узле;
		\item Выполнить подключения к TCP-серверам, работающим на центральном сетевом узле
		и периферийном сетевом узле \first, с периферийного сетевого узла\\\second\ и центрального сетевого узла
		соответственно;
		\item Получить список всех TCP-портов центрального сетевого узла, по которым осуществляется обмен данными с
		периферийными сетевыми узлами;
		\item Получить список всех TCP-портов центрального сетевого узла, прослушиваемых процессами - серверами,
		запущенными на центральном сетевом узле;
		\item Получить список всех TCP-портов периферийного сетевого узла \second, прослушиваемых процессами -
		серверами, запущенными на периферийном сетевом узле \second;
		\item Завершить выполнение TCP-серверов, работающих на центральном сетевом узле и периферийном сетевом
		узле \first;
		\item Добавить для центрального сетевого узла символьные псевдонимы для IP-адресов обоих периферийных сетевых
		узлов.

		Псевдонимы должны быть образованы от порядкового номера студента в журнале группы следующим образом:
		<<lab-2-N-host-M>>, где N - порядковый номер студента в журнале группы, M - порядковый номер периферийного
		сетевого узла;

		\item Проверить доступность центральному сетевому узлу периферийного сетевого узла \linebreak \first\ с помощью утилиты
		ping, пример использования которой был приведен в первой лабораторной работе.

		При проверки доступности центральному сетевому узлу периферийного сетевого узла \first\ необходимо
		использовать символьный псевдоним IP-адреса периферийного сетевого узла \first;

		\item Добавить для центрального сетевого узла символьные псевдонимы вида\\<<lab-2-N-port-$Q$>> портов с
		номерами $Q_1$ и $Q_2$, где $Q = 1~\text{или}~2$, $Q_1 = 50070 + (N~mod~6 + 1) * 10$, $Q_2 = Q_1 + 1$;

		\item Запустить TCP-сервер на периферийном сетевом узле \second\ на $Q_1$-м порту;
		\item Запустить TCP-сервер на центральном сетевом узле на $Q_2$-м порту;
		\item Выполнить подключения к TCP-серверам, работающим на центральном сетевом узле и периферийном сетевом
		узле \second, с периферийного сетевого узла\\\first\ и центрального сетевого узла соответственно.

		Для подключения к периферийному сетевому узлу \second\ необходимо использовать символьный псевдоним IP-адреса
		данного сетевого узла;

		\item Получить список всех TCP-портов центрального сетевого узла, по которым осуществляется обмен данными с
		периферийными сетевыми узлами;
		\item Получить список всех TCP-портов центрального сетевого узла, прослушиваемых процессами - серверами,
		запущенными на центральном сетевом узле;
		\item Удалить добавленные в ходе выполнения лабораторной работы символьные псевдонимы IP-адресов периферийных
		сетевых узлов и символьные псевдонимы TCP-портов центрального сетевого узла.

	\end{enumerate}

\subsubsection{Сетевое программирование с использованием протоколов TCP и UDP}
\label{task:l2t2}

	Каждому студенту группы необходимо выполнить следующее задание:

	\begin{enumerate}

		\item Разработать программный комплекс, состоящий из следующих компонент:

			\begin{itemize}

				\item Серверная компонента, запускаемая на периферийном сетевом узле\\\first, и выполняющая команды,
				передаваемые ей управляющей компонентой по протоколу UDP.

				После выполнения очередной команды серверная компонента должна выводить соответствующее сообщение
				в стандартный поток вывода;

				\item Серверная компонента, запускаемая на периферийном сетевом узле\\\second, и выполняющая команды,
				передаваемые ей управляющей компонентой по протоколу TCP;

				После выполнения очередной команды серверная компонента должна выводить соответствующее сообщение
				в стандартный поток вывода;

				\item Управляющая компонента, запускаемая на центральном сетевом узле.

				Управляющая компонента должна выполнять обмен с серверными компонентами по указанным протоколам
				в соответствии с командами оператора, должна получать сообщения о результатах выполнения команд
				и выводить соответствующие сообщения в стандартный поток вывода;

			\end{itemize}

		\item Запустить серверные компоненты на периферийных сетевых узлах \first\ и \second;
		\item Запустить управляющую компоненту на центральном сетевом узле;
		\item Продемонстрировать функционирование программного комплекса в соответствии с указаниями преподавателя;
		\item Завершить функционирование серверных компонент программного комплекса по получению управляющей
		компонентой соответствующей команды от оператора;
		\item Завершить выполнение управляющей компоненты программного комплекса.

	\end{enumerate}

	Управляющая компонента разрабатываемого программного комплекса должна принимать следующие команды оператора
	центрального сетевого узла:

	\begin{itemize}

		\item <<halt>> - команда завершения выполнения всех компонент программного комплекса;

		\item <<ping NODE>> - команда запроса ответа от серверной компоненты периферийного узла с номером NODE.
		Серверная компонента должна незамедлительно ответить произвольным ответным сообщением, факт получения которого
		должен быть зафиксирован управляющей компонентой выводом соответствующего сообщения в стандартный поток вывода;

		\item <<rand>> - запрос у обоих серверных компонент случайного 32-х битового целого числа. После получения
		случайных чисел управляющая компонента должна выполнить над ними операцию побитового исключающего и,
		результат которой управляющая компонента должна вывести в стандартный поток вывода;

		\item <<stat>> - вывод статистической информации о количестве выполненных управляющей компонентой команд.

	\end{itemize}

	Управляющая компонента должна считывать команды оператора со стандартного потока ввода в интерактивном режиме.
	На ввод последовательностей символов, не являющихся командами, управляющая компонента должна выводить в стандартный
	поток вывода справочную информацию по использованию управляющей компоненты.

	Протокол пересылки команд и результатов их выполнения между управляющей компонентой и серверными компонентами
	необходимо разработать самостоятельно.

\subsubsection{Подготовка отчета по лабораторной работе}

	Подготовить отчет, в котором необходимо привести подробные описания процессов выполнения заданий пунктов
	\ref{task:l2t1} и \ref{task:l2t2}, проиллюстрированные достаточным количеством снимков экрана.

	В отчете по лабораторной работе должен быть приведен исходный код разработанного программного обеспечения с
	соответствующими комментариями.

