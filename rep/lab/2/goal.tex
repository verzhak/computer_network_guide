
Ознакомление с протоколами транспортного уровня стека протоколов TCP/IP (в частности, с протоколами TCP и UDP).

Ознакомление с принципами IP-адресации сетевых узлов в вычислительных сетях, обмен в которых осуществляется с помощью протоколов стека протоколов TCP/IPv4.

Получение навыков использования утилиты netstat для определения состояния виртуальных TCP и UDP-портов сетевых интерфейсов вычислительной системы.

Получение навыков использования утилиты ncat для организации простейшего обмена данными по протоколам TCP и UDP
между операторами удаленных друг от друга вычислительных систем.

Ознакомление с содержимых файлов /etc/services и /etc/hosts. Получение навыков использования символьных псевдонимов IP-адресов сетевых узлов и
символьных псевдонимов виртуальных портов сетевых интерфейсов.

Получение навыков сетевого программирования с использованием протоколов TCP и UDP.

