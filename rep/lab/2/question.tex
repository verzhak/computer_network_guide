
Одно из следующих заданий может быть предложено в качестве задания для защиты отчета по лабораторной работе:

\begin{enumerate}

	\item Переработка серверной компоненты, осуществлявшей обмен информацией с управляющей компонентой по протоколу
	UDP, для использования при означенном обмене протокола TCP;

	\item Переработка серверной компоненты, осуществлявшей обмен информацией с управляющей компонентой по протоколу
	TCP, для использования при означенном обмене протокола UDP;

	\item Добавление функционала обработки управляющей компонентой одной из следующих команд:

		\begin{itemize}

			\item <<get NODE FILE\_IN FILE\_OUT>> - команда запроса у серверной компоненты, запущенной на периферийном
			сетевом узле с номером NODE, содержимого файла FILE\_IN и сохранение полученной информации на центральном
			сетевом узле в файле FILE\_OUT;

			\item <<put NODE FILE\_IN FILE\_OUT>> - копирование файла FILE\_IN центрального сетевого узла в файл FILE\_OUT
			периферийного сетевого узла с номером NODE;

			\item <<route NODE>> - запрос у серверной компоненты, функционирующей на периферийном сетевом узле с номером
			NODE, таблицы маршрутизации данного периферийного сетевого узла;

		\end{itemize}

	\item Добавление поддержки обмена сообщениями между оператором управляющей компоненты и операторами сетевых узлов,
	на которых запущены серверные компоненты;

	\item Добавление поддержки симметричного шифрования данных, пересылаемых между серверными и управляющей компонентами.

		В качестве простейшего алгоритма симметричного шифрования можно использовать <<затенение>> байт с помощью
		операции побитового исключающего или.

		Ключи шифрования должны быть сгенерированы управляющей компонентой и разосланы серверным компонентам
		непосредственно после установа соединения между серверными и управляющей компонентами.

\end{enumerate}

