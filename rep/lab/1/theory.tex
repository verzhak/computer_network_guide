
\subsubsection{Общие сведения}

	\linux\ является сетевой ОС и позволяет вычислительной системе, работающей под ее управлением,
	взаимодействовать с прочими вычислительными системами в сетях самой разнообразной организации.

	Так, например, \linux\ позволяет организовать функционирование вычислительной системы в составе локальной вычислительной
	сети, построенной по одной из вариаций технологии Ethernet и использующей протоколы стека протоколов
	TCP/IP\footnote{Допускается использование как протокола IPv4, так и протокола IPv6.}
	для взаимодействия узлов сети на сетевом и транспортном уровнях сетевой модели OSI.

	В общем случае, процесс функционирования вычислительной системы, работающей под управлением \linux,
	в качестве сетевого узла локальной вычислительной сети, основанной на технологии Ethernet и стеке протоколов
	TCP/IPv4, состоит из следующих этапов:

	\begin{enumerate}

		\item Физическое подключение сетевого адаптера к материнской плате вычислительной системы;

		\item Физическое подключение сетевого адаптера к каналу связи локальной вычислительной сети;

		\item Загрузка ОС вычислительной системы;

		\item Вход в ОС оператора в качестве суперпользователя;

		\item Настройка подключения сетевого адаптера к локальной вычислительной сети - создание сетевого узла на базе вычислительной системы. 
		
		Вычислительная система обеспечивает функционирование сетевого узла, подключенного через сетевой адаптер к локальной вычислительной сети.
		Сетевой адаптер доступен процессам пользовательского уровня ОС через соответствующий сетевой интерфейс.

		Данный этап предполагает:

			\begin{itemize}

				\item Задание IPv4-адреса настраиваемого сетевого узла;
				\item Задание маски сети\footnote{Маска сети суть есть целое беззнаковое число, позволяющее определить адрес сети и широковещательный адрес
				сети по адресу любого сетевого узла, состоящего в сети.} и широковещательного адреса сети\footnote{Широковещательный адрес сети используется
				при сетевом обмене для массовой рассылки пакетов всем сетевым узлам, состоящим в сети.}, в которую включен настраиваемый сетевой узел;
				\item Настройка дополнительных параметров сетевого узла.
				
				К числу дополнительных параметров сетевого узла относится, например, Maximum Transmission Unit
				(MTU) - максимальный размер пакета, могущего быть переданным через данный сетевым узлом или данному сетевому узлу
				по протоколу канального уровня сетевой модели OSI;

			\end{itemize}

		\item Настройка таблицы маршрутизации.

		С помощью таблицы маршрутизации вычислительная система принимает решение, по какому маршруту
		(какому сетевому узлу через какой сетевой интерфейс) необходимо передать очередной пакет данных, чтобы
		означенный пакет достиг сетевого узла - приемника;

		\item Запуск и настройка дополнительных сервисов.

		В число дополнительных сервисов входят, например, DNS-сервер, DHCP-сервер, WEB-сервер, FTP-сервер и прочие сервисы;

		\item Обмен данными между процессами вычислительной системы и процессами удаленных вычислительных систем;

		\item Останов дополнительных сервисов;

		\item Очистка таблицы маршрутизации;

		\item Сброс настроек сетевого интерфейса;

		\item Останов ОС.

	\end{enumerate}

	В данной лабораторной работе будет рассмотрен способ подключения вычислительной системы, работающей
	под управлением \linux, к локальной вычислительной сети, основанной на технологии Ethernet и
	стеке протоколов TCP/IPv4.

\subsubsection{Настройка подключения вычислительной системы к локальной вычислительной сети}

	Настройка подключения вычислительной системы к локальной вычислительной сети может быть произведена
	различными способами. Каждый из способов предполагает определенную степень автоматизации процесса подключения -
	от полностью ручного способа, рассматриваемого в данной лабораторной работе, до полностью автоматического,
	заключающегося в запуске операционной системой определенного набора программ с определенными ключами по
	составленному администратором ОС скрипту на некотором языке программирования\footnote{Чаще всего в качестве языка автоматизации
	выбирается язык командного интерпретатора, используемого администратором.}.

	Ручной способ настройки подключения вычислительной системы к локальной вычислительной сети предполагает следующую последовательность
	действий:

	\begin{enumerate}

		\item Вход оператора ОС в ОС в качестве суперпользователя;

		\item Настройка ядра ОС.

		Настройка ядра ОС предполагает загрузку в пространство ядра кода сетевой подсистемы ядра
		и драйвера используемого сетевого адаптера.
		
		Как правило, хотя и необязательно, код сетевой подсистемы ядра находится в файле ядра и,
		соответственно, загружается в ядерное пространство непосредственно в ходе загрузки ядра ОС на
		ранней стадии загрузки ОС. В том случае, если код сетевой подсистемы ядра скомпилирован в виде отдельных
		модулей ядра, подгружаемых в ходе работы ОС, данные модули загружаются в ходе запуска графической подсистемы
		ОС, поскольку она не может функционировать без сетевой подсистемы, или загружаются ОС автоматически,
		при выполнении попыток настройки сетевого адаптера. В крайне редких случаях, когда код сетевой подсистемы
		отсутствует в ядре, требуется перекомпилировать ядро ОС, что выходит за рамки данной лабораторной работы.

		Драйвер используемого сетевого адаптера может находится в файле ядра ОС, в отдельном модуле ядра,
		загружаемым ядром при выполнении попытки настройки сетевого адаптера, или отсутствовать в ядре вовсе.

		Определить, присутствует ли драйвер сетевого адаптера в пространстве ядра ОС, можно с помощью утилиты
		ifconfig - для этого необходимо запустить данную утилиту с ключом <<-a>>, который предписывает утилите
		ifconfig вывести в стандартный поток вывода информацию о всех присутствующих в вычислительной системе
		(но, возможно, ненастроенных) сетевых адаптерах.
		На рисунке \ref{image:lab1-5} приведен возможный вывод утилиты ifconfig, запущенной с ключом <<-a>>.

		\mimage{lab1-5}{1/5}{Список сетевых адаптеров, присутствующих в вычислительной системе}{}

		Как видно из рисунка \ref{image:lab1-5} к вычислительной системе подключены следующие сетевые адаптеры:

		\begin{itemize}

			\item eth0 - Ethernet-адаптер. IP-адрес отсутствует => адаптер не настроен;
			\item lo - виртуальный интерфейс обратной петли;
			\item vboxnet0 - виртуальный Ethernet-адаптер, используемый виртуальными машинами VirtualBox
				  (ни одна из таковых не запущена => vboxnet0 не настроен (IP-адрес отсутствует));
			\item vmnet1, vmnet8 - виртуальные Ethernet-адаптеры, используемые виртуальными машинами VMWare.
			Оба адаптера настроены и имеют следующие IP-адреса:

				\begin{itemize}

					\item vmnet1 - 192.168.126.1;
					\item vmnet8 - 192.168.91.1;

				\end{itemize}

			\item wlan0 - Wi-Fi карта (не настроена);

		\end{itemize}

		\item Настройка подключения сетевого интерфейса сетевого адаптера, с помощью которого выполняется подключение к локальной
		вычислительной сети, к локальной вычислительной сети.

		Означенная настройка предполагает:

		\begin{itemize}

			\item Задание IP-адреса настраиваемого сетевого узла;
			\item Задание маски сети и широковещательного адреса сети, в которую включен сетевой узел;
			\item Настройка дополнительных параметров сетевого узла.

		\end{itemize}

		Настройка сетевого интерфейса выполняется с помощью утилиты ifconfig и предполагает следующую последовательность действий.

		\begin{enumerate}

			\item Определение имени сетевого интерфейса.

			Интерфейсы Ethernet-адаптеров именуются в \linux\ в виде последовательности символов <<ethN>>, где N -
			порядковый номер сетевого адаптера (считая с 0), зависящий от его физического подключения к
			материнской плате вычислительной системы.

			Узнать имя сетевого интерфейса можно с помощью утилиты ifconfig, запущенной с ключом <<-a>>;

			\item Сброс настроек сетевого интерфейса.

			Для сброса настроек сетевого интерфейса необходимо выполнить команду <<ifconfig ethN down>>;

			\item Настройка сетевого интерфейса.

			Для настройки сетевого интерфейса необходимо выполнить команду <<ifconfig ethN IP broadcast BROADCAST netmask NETMASK>>, где:
			
			\begin{itemize}
			
				\item IP - IP-адрес сетевого узла;
				\item BROADCAST - широковещательный адрес сети;
				\item NETMASK - маска сети.

			\end{itemize}

			Передав утилите ifconfig дополнительные ключи, можно произвести настройку прочих параметров
			сетевого адаптера. Так, например, с помощью ключа <<mtu MTU\_SIZE>> оператор может настроить
			максимальный размер пакета (MTU\_SIZE; указывается в байтах),
			могущего быть переданным через сетевой адаптер по протоколу канального уровня сетевой модели OSI,
			что полезно для некоторых сетевых адаптеров отдельных производителей, требующих для своего корректного функционирования
			значение MTU меньшее, чем традиционное значение MTU, равное 1500 байтам.

		\end{enumerate}

		В случае успешного завершения процесса настройки сетевого интерфейса, запись о данном сетевом интерфейсе появится
		в списке корректно настроенных сетевых интерфейсов, могущего быть полученным с помощью утилиты ifconfig, запущенной без ключей,
		что проиллюстрировано рисунком \ref{image:lab1-4};

		\mimage{lab1-4}{1/4}{Список корректно настроенных сетевых интерфейсов вычислительной системы}{width=\textwidth}

		\item Настройка таблицы маршрутизации.

		После осуществления настройки сетевого интерфейса необходимо выполнить настройку таблицы маршрутизации, для чего используется утилита route.

		Одна запись, связанная с настраиваемым сетевым интерфейсом, в таблице маршрутизации уже есть -
		это указание направлять все пакеты, следующие в соответствующую сеть, через настроенный сетевой интерфейс.

		Необходимо добавить в таблицу маршрутизации запись о маршруте по умолчанию -
		то есть запись о маршруте, по которому будут отправляться пакеты в случае, если других маршрутов,
		по которым можно было бы отправить пакет, нет.
		
		Для добавления записи в таблицу маршрутизации о маршруте по умолчанию необходимо выполнить следующие действия:

		\begin{enumerate}

			\item Удалить запись о маршруте по умолчанию из таблицы маршрутизации.

			Для удаления записи о маршруте по умолчанию из таблицы маршрутизации необходимо выполнить команду
			<<route del default>>;
			
			\item Добавить запись о маршруте по умолчанию в таблицу маршрутизации
			с предписанием использовать целевой сетевой узел в качестве шлюза между сетями.

			Для добавления записи о маршруте по умолчанию в таблицу маршрутизации необходимо выполнить команду
			<<route add default gw IP ethN>>, где IP - IP-адрес сетевого узла - шлюза.

		\end{enumerate}

		Содержимое таблицы маршрутизации можно получить с помощью утилиты route, запустив данную утилиту без каких бы то ни было ключей.
		Пример получения содержимого таблицы маршрутизации с помощью утилиты route приведен на рисунке \ref{image:lab1-route}.

		\mimage{lab1-route}{1/route}{Состояние таблицы маршрутизации}{width=\textwidth}

	\end{enumerate}

\subsubsection{Проверка корректности функционирования локальной вычислительной сети}

	По завершению настройки вычислительной сети разумно проверить корректность ее функционирования,
	для чего можно использовать утилиту ping.
		
	Утилита ping предназначена для проверки доступности целевого сетевого узла и позволяет оценить
	некоторые параметры процесса обмена данными с целевым удаленным сетевым узлом.

	Для проверки доступности сетевого узла, объединенного с сетевым узлом, на котором производится
	запуск утилиты ping, в вычислительную сеть, утилите ping необходимо передать IP-адрес сетевого узла,
	доступность которого проверяется.

	На рисунке \ref{image:lab1-ping} приведен результат выполнения утилиты ping,
	запущенной на некотором сетевом узле последовательно с ключами <<-c 5 10.0.1.128>> и
	<<-c 5 10.0.2.128>>\footnote{Ключ <<-c 5>> суть есть указание использовать при проверки доступности
	целевого сетевого узла только 5 циклов обмена пакетами типов ECHO и ECHO-REPLY протокола ICMP.}.
	Как видно из рисунка \ref{image:lab1-ping}, сетевые узлы 10.0.1.128 и 10.0.2.128 доступны сетевому узлу, на котором запущена утилита ping,
	что позволяет говорить о корректности настроек локальной вычислительной сети.

	\mimage{lab1-ping}{1/ping}{Проверка корректности функционирования вычислительной сети}{width=\textwidth}

