
\subsubsection{Физическая организация локальной вычислительной сети, используемой в данной лабораторной работе}
	\label{lab1:lab1-net}

	Для выполнения данной лабораторной работы используется вычислительная сеть, состоящая из трех сетевых узлов.

	Сетевой узел, в дальнейшем называемый <<центральным>> сетевым узлом, и два сетевых узла, в дальнейшем называемые <<периферийными>> сетевыми узлами,
	должны быть организованы на основе виртуальных вычислительных систем, функционирующих с помощью программного комплекса
	виртуализации вычислительных систем \virtpo.

	Вычислительные системы, на которых должны быть организованы центральный и периферийные сетевые узлы,
	обладают следующими характеристиками:

	\begin{itemize}

		\item Аппаратная часть - вычислительная система на базе микропроцессора архитектуры ia32;
		\item Программная часть - \linux\ с ядром Linux версии 2.6.20 и выше и библиотекой GLIBC версии 2.11 и выше.

	\end{itemize}

	Центральный сетевой узел должен быть объединен с каждым из периферийных сетевых узлов в вычислительную сеть
	с помощью отдельного виртуального сетевого Ethernet-адаптера, функционирование которого обеспечивает \virtpo.
	Таким образом, вычислительная сеть, объединяющая три указанных сетевых узла, должна состоять из двух подсетей,
	каждая из которых должна состоять из центрального сетевого узла и одного из периферийных сетевых узлов.
	Структурная схема вычислительной сети приведена на рисунке \ref{image:lab1-struct}.

	\mimage{lab1-struct}{1/struct}{Структурная схема используемой вычислительной сети}{width=\textwidth}

\subsubsection{Настройка вычислительной сети}
	\label{lab1:lab1-p1}

	В соответствии с пунктом \ref{lab1:lab1-net} необходимо создать и настроить вычислительную сеть, состоящую из трех сетевых узлов,
	организованных на основе вычислительных систем, виртуализируемых с помощью программного комплекса \virtpo\ и
	использующих \linux, соответствующую заявленным в пункте \ref{lab1:lab1-net} требованиям.
	
	Для получения виртуальных машин, объединяемых в вычислительную сеть, необходимо выполнить следующие действия:

	\begin{enumerate}

		\item Получить у преподавателя эталонную виртуальную машину с предустановленной на ней \linux, соответствующей
		приведенным в пункте \ref{lab1:lab1-net} требованиям;

		\item Сделать две копии эталонной виртуальной машины.

	\end{enumerate}

	В дальнейшем, при запуске копий эталонной виртуальной машины на возможный запрос программного комплекса виртуализации о
	подтверждении получения виртуальной машины путем копирования эталонной виртуальной машины необходимо отвечать утвердительно.

	Полученные три идентичные виртуальные машины необходимо использовать следующим образом:

	\begin{enumerate}

		\item Эталонную виртуальную машину - в качестве вычислительной системы, на базе которой организуется
		центральный сетевой узел;

		\item Копии эталонной виртуальной машины - в качестве вычислительных систем, на базе которых
		организуются периферийные сетевые узлы.

	\end{enumerate}

	Перед выполнением непосредственной настройки вычислительной сети, необходимо выполнить следующие действия с виртуальными сетевыми адаптерами \virtpo:

	\begin{enumerate}

		\item Убедиться, что центральный сетевой узел физически подключен к обеим подсетям вычислительной сети,
		для чего необходимо открыть окно настроек виртуальной машины, на базе которой организуется
		центральный сетевой узел, и убедиться в том, что в данном окне присутствуют записи о
		подключенных к виртуальной машине двух виртуальных Ethernet-адаптерах;

		\item Для каждой из двух виртуальных машин, на базе которых организуются периферийные сетевые узлы,
		в окне настроек виртуальной машины требуется удалить одно из подключений к виртуальным сетевым адаптерам,
		для чего в окне настроек виртуальной машины необходимо выбрать
		соответствующее подключение и щелкнуть по кнопке <<Удалить>> (<<Delete>>) означенного окна.
		В конце концов, окно настроек виртуальной машины, на базе которой организуется периферийный сетевой узел,
		с точностью до имени сетевого адаптера примет вид, приведенный на рисунке \ref{image:lab1-vmsetup}.

		Необходимо помнить, что удаляемые подключения виртуальных машин к сетевым адаптерам
		не должны соответствовать одному и тому же виртуальному сетевому адаптеру,
		иначе создать сеть, состоящую из двух физически разделенных сетей, не получится.

		\mimage{lab1-vmsetup}{1/vmsetup}{Окно настроек виртуальной машины, на базе которой должен быть организован
		периферийный сетевой узел}{width=\textwidth}

	\end{enumerate}

	Непосредственная настройка вычислительной сети заключается в выполнении следующих действий:

	\begin{enumerate}

		\item На центральном сетевом узле:

			\begin{enumerate}

				\item Настройка сетевых интерфейсов в соответствии со схемой, приведенной на рисунке
				\ref{image:lab1-struct}.

				При выполнении настройки сетевых интерфейсов указывать широковещательные адреса сетей не нужно.
				В качестве масок сетей необходимо указывать 255.255.255.0;

				\item Настройка проброса пакетов между сетями.

				Для включения проброса пакетов между сетями необходимо выполнить команду <<echo 1 > /proc/sys/net/ipv4/ip\_forward>>.
				Данная команда запишет единицу в файл /proc/sys/net/ipv4/ip\_forward, содержимое которого (0 или 1) выступает в роли флага при
				принятии решения сетевым узлом о возможности проброса пакетов между сетями;

			\end{enumerate}

		\item На каждом из периферийных сетевых узлов:

			\begin{enumerate}

				\item Настройка сетевого интерфейса в соответствии со схемой, приведенной на рисунке
				\ref{image:lab1-struct}.

				При выполнении настройки сетевых интерфейсов указывать широковещательные адреса сетей не нужно.
				В качестве масок сетей необходимо указывать 255.255.255.0;

				\item Добавление в таблицу маршрутизации записи о центральном сетевом узле, как о шлюзе по умолчанию.

			\end{enumerate}

	\end{enumerate}

	Для проверки корректности настройки вычислительной сети необходимо, используя утилиту ping, выполнить следующие действия:

	\begin{enumerate}

		\item Проверить доступность центральному сетевому узлу каждого из периферийных сетевых узлов,
		послав каждому из периферийных сетевых узлов 10 пакетов типа ECHO протокола ICMP;

		\item Проверить доступность одному из периферийных узлов другого периферийного узла,
		послав тому 10 пакетов типа ECHO протокола ICMP.

	\end{enumerate}

\subsubsection{Получение содержимого таблицы маршрутизации}
	\label{lab1:lab1-p2}

	Необходимо разработать программу на языке C, которая, будучи запущена на центральном сетевом узле, выведет в стандартный поток вывода
	список маршрутов, присутствующих в таблице маршрутизации центрального сетевого узла, в виде строк вида: <<IFACE DEST GATEWAY MASK>>, где:

	\begin{itemize}

		\item IFACE - имя сетевого интерфейса, через который отправляются пакеты по соответствующему маршруту;
		\item DEST - IP-адрес целевой сети, записанный в традиционной нотации (четыре числа в десятичной системе счисления, разделенные точками);
		\item GATEWAY - IP-адрес шлюза, записанный в традиционной нотации;
		\item MASK - маска сети, записанная в традиционной нотации.

	\end{itemize}

	Существуют несколько способов получения содержимого таблицы маршрутизации в \linux\ в программах, написанных на языке программирования C.
	Одним из таковых способов является чтение содержимого файла /proc/net/route, в котором текущая таблица маршрутизации представлена в виде ASCII-текста.
	Пример содержимого файла \linebreak /proc/net/route некоторой вычислительной системы приведен на рисунке
	\ref{image:lab1-flroute}.

	\mimage{lab1-flroute}{1/100}{Содержимое файла /proc/net/route}{width=\textwidth}

	Содержимое файла /proc/net/route представляет из себя таблицу, состоящую из нескольких строк,
	каждая из которых, кроме первой строки и пустых строк, описывает один из доступных маршрутов.
	Первая строка содержит заголовок таблицы. Колонки таблицы, хранимой в файле /proc/net/route,
	описывают определенные параметры маршрутов:

	\begin{itemize}
		
		\item Iface - имя сетевого интерфейса;
		\item Destination - IP-адрес целевой сети;
		\item Gateway - IP-адрес шлюза;
		\item Mask - маска сети.

	\end{itemize}

	Все IP-адреса, записанные в файле /proc/net/route, представлены в виде 32-х битовых шестнадцатеричных чисел в сетевом порядке байт.
	Необходимо помнить, что порядок байт хоста в случае вычислительных систем, построенных на базе процессов архитектур ia32 и ia64,
	обратен сетевому порядку байт.

	Для редактирования файла исходного кода программы необходимо воспользоваться консольным текстовым редактором nano:

	\begin{enumerate}

		\item Открыть файл исходного кода на редактирование.
		
		Для открытия файла исходного кода на редактирование необходимо запустить на выполнение текстовый редактор nano,
		передав тому в качестве аргумента командной строки имя файла исходного кода. В том случае, если на момент открытия файл исходного кода не существовал,
		он будет создан, если оператор в процессе редактирования файла сохранит изменения, внесенные в файл.

		Файл исходного кода, очевидно, должен иметь расширение <<.c>>;

		\item Записать в файл исходного кода исходный код программы.

		Для переключения между раскладками (английской и русской) можно воспользоваться сочетанием клавиш <<Shift + Ctrl>>;

		\item Выйти из текстового редактора с сохранением изменений, внесенных в файл исходного кода.

		Для выхода из текстового редактора необходимо нажать сочетание клавиш <<Ctrl + X>>.
		Для подтверждения сохранения изменений при выходе из текстового редактора оператор должен ввести с
		клавиатуры символ <<Y>> в ответ на соответствующий запрос.
		
		Символ <<\verb|^|>>, присутствующий в большинстве указаний сочетаний клавиш для ввода команд управления редактором, означает клавишу <<Ctrl>>.
		Например, строка <<\verb|^|X>> соответствует сочетанию клавиш <<Ctrl + X>>.

	\end{enumerate}

	Компиляцию программы необходимо осуществить с помощью компилятора GNU C Compiler, выполнив в каталоге с файлом исходного кода следующую команду:
	<<gcc FILE -o program.out>>. Здесь:

	\begin{itemize}

		\item FILE - имя файла исходного кода;
		\item <<-o program.out>> - указание компилятору генерировать на выходе исполняемый файл program.out,
		сохраняемый в текущем каталоге.

	\end{itemize}

	Для запуска на выполнение полученного исполняемого файла необходимо выполнить команду <<PATH/program.out>>,
	где PATH - относительный или абсолютный путь к каталогу, содержащему исполняемый файл.
	PATH равен <<.>>, если исполняемый файл находится в текущем каталоге.

	Для принудительного завершения программы ее главному процессу необходимо отправить сигнал SIGINT нажатием сочетания клавиш <<Ctrl + C>> в том терминале
	вычислительной системы, в котором программа запущена.

\subsubsection{Подготовка отчета по лабораторной работе}

	Отчет по лабораторной работе должен содержать подробные описания процессов выполнения заданий пунктов
	\ref{lab1:lab1-p1} и \ref{lab1:lab1-p2}, проиллюстрированные достаточным количеством снимков экрана.

	В отчете по лабораторной работе должен быть приведен исходный код разработанного программного обеспечения с
	соответствующими комментариями.

